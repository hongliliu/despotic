% Generated by Sphinx.
\def\sphinxdocclass{report}
\documentclass[letterpaper,10pt,english]{sphinxmanual}
\usepackage{iftex}

\ifPDFTeX
  \usepackage[utf8]{inputenc}
\fi
\ifdefined\DeclareUnicodeCharacter
  \DeclareUnicodeCharacter{00A0}{\nobreakspace}
\fi
\usepackage{cmap}
\usepackage[T1]{fontenc}
\usepackage{amsmath,amssymb,amstext}
\usepackage{babel}
\usepackage{times}
\usepackage[Bjarne]{fncychap}
\usepackage{longtable}
\usepackage{sphinx}
\usepackage{multirow}
\usepackage{eqparbox}


\addto\captionsenglish{\renewcommand{\figurename}{Fig.\@ }}
\addto\captionsenglish{\renewcommand{\tablename}{Table }}
\SetupFloatingEnvironment{literal-block}{name=Listing }

\addto\extrasenglish{\def\pageautorefname{page}}

\setcounter{tocdepth}{1}


\title{DESPOTIC Documentation}
\date{Feb 16, 2017}
\release{2.0}
\author{Mark R. Krumholz}
\newcommand{\sphinxlogo}{}
\renewcommand{\releasename}{Release}
\makeindex

\makeatletter
\def\PYG@reset{\let\PYG@it=\relax \let\PYG@bf=\relax%
    \let\PYG@ul=\relax \let\PYG@tc=\relax%
    \let\PYG@bc=\relax \let\PYG@ff=\relax}
\def\PYG@tok#1{\csname PYG@tok@#1\endcsname}
\def\PYG@toks#1+{\ifx\relax#1\empty\else%
    \PYG@tok{#1}\expandafter\PYG@toks\fi}
\def\PYG@do#1{\PYG@bc{\PYG@tc{\PYG@ul{%
    \PYG@it{\PYG@bf{\PYG@ff{#1}}}}}}}
\def\PYG#1#2{\PYG@reset\PYG@toks#1+\relax+\PYG@do{#2}}

\expandafter\def\csname PYG@tok@gd\endcsname{\def\PYG@tc##1{\textcolor[rgb]{0.63,0.00,0.00}{##1}}}
\expandafter\def\csname PYG@tok@gu\endcsname{\let\PYG@bf=\textbf\def\PYG@tc##1{\textcolor[rgb]{0.50,0.00,0.50}{##1}}}
\expandafter\def\csname PYG@tok@gt\endcsname{\def\PYG@tc##1{\textcolor[rgb]{0.00,0.27,0.87}{##1}}}
\expandafter\def\csname PYG@tok@gs\endcsname{\let\PYG@bf=\textbf}
\expandafter\def\csname PYG@tok@gr\endcsname{\def\PYG@tc##1{\textcolor[rgb]{1.00,0.00,0.00}{##1}}}
\expandafter\def\csname PYG@tok@cm\endcsname{\let\PYG@it=\textit\def\PYG@tc##1{\textcolor[rgb]{0.25,0.50,0.56}{##1}}}
\expandafter\def\csname PYG@tok@vg\endcsname{\def\PYG@tc##1{\textcolor[rgb]{0.73,0.38,0.84}{##1}}}
\expandafter\def\csname PYG@tok@vi\endcsname{\def\PYG@tc##1{\textcolor[rgb]{0.73,0.38,0.84}{##1}}}
\expandafter\def\csname PYG@tok@mh\endcsname{\def\PYG@tc##1{\textcolor[rgb]{0.13,0.50,0.31}{##1}}}
\expandafter\def\csname PYG@tok@cs\endcsname{\def\PYG@tc##1{\textcolor[rgb]{0.25,0.50,0.56}{##1}}\def\PYG@bc##1{\setlength{\fboxsep}{0pt}\colorbox[rgb]{1.00,0.94,0.94}{\strut ##1}}}
\expandafter\def\csname PYG@tok@ge\endcsname{\let\PYG@it=\textit}
\expandafter\def\csname PYG@tok@vc\endcsname{\def\PYG@tc##1{\textcolor[rgb]{0.73,0.38,0.84}{##1}}}
\expandafter\def\csname PYG@tok@il\endcsname{\def\PYG@tc##1{\textcolor[rgb]{0.13,0.50,0.31}{##1}}}
\expandafter\def\csname PYG@tok@go\endcsname{\def\PYG@tc##1{\textcolor[rgb]{0.20,0.20,0.20}{##1}}}
\expandafter\def\csname PYG@tok@cp\endcsname{\def\PYG@tc##1{\textcolor[rgb]{0.00,0.44,0.13}{##1}}}
\expandafter\def\csname PYG@tok@gi\endcsname{\def\PYG@tc##1{\textcolor[rgb]{0.00,0.63,0.00}{##1}}}
\expandafter\def\csname PYG@tok@gh\endcsname{\let\PYG@bf=\textbf\def\PYG@tc##1{\textcolor[rgb]{0.00,0.00,0.50}{##1}}}
\expandafter\def\csname PYG@tok@ni\endcsname{\let\PYG@bf=\textbf\def\PYG@tc##1{\textcolor[rgb]{0.84,0.33,0.22}{##1}}}
\expandafter\def\csname PYG@tok@nl\endcsname{\let\PYG@bf=\textbf\def\PYG@tc##1{\textcolor[rgb]{0.00,0.13,0.44}{##1}}}
\expandafter\def\csname PYG@tok@nn\endcsname{\let\PYG@bf=\textbf\def\PYG@tc##1{\textcolor[rgb]{0.05,0.52,0.71}{##1}}}
\expandafter\def\csname PYG@tok@no\endcsname{\def\PYG@tc##1{\textcolor[rgb]{0.38,0.68,0.84}{##1}}}
\expandafter\def\csname PYG@tok@na\endcsname{\def\PYG@tc##1{\textcolor[rgb]{0.25,0.44,0.63}{##1}}}
\expandafter\def\csname PYG@tok@nb\endcsname{\def\PYG@tc##1{\textcolor[rgb]{0.00,0.44,0.13}{##1}}}
\expandafter\def\csname PYG@tok@nc\endcsname{\let\PYG@bf=\textbf\def\PYG@tc##1{\textcolor[rgb]{0.05,0.52,0.71}{##1}}}
\expandafter\def\csname PYG@tok@nd\endcsname{\let\PYG@bf=\textbf\def\PYG@tc##1{\textcolor[rgb]{0.33,0.33,0.33}{##1}}}
\expandafter\def\csname PYG@tok@ne\endcsname{\def\PYG@tc##1{\textcolor[rgb]{0.00,0.44,0.13}{##1}}}
\expandafter\def\csname PYG@tok@nf\endcsname{\def\PYG@tc##1{\textcolor[rgb]{0.02,0.16,0.49}{##1}}}
\expandafter\def\csname PYG@tok@si\endcsname{\let\PYG@it=\textit\def\PYG@tc##1{\textcolor[rgb]{0.44,0.63,0.82}{##1}}}
\expandafter\def\csname PYG@tok@s2\endcsname{\def\PYG@tc##1{\textcolor[rgb]{0.25,0.44,0.63}{##1}}}
\expandafter\def\csname PYG@tok@nt\endcsname{\let\PYG@bf=\textbf\def\PYG@tc##1{\textcolor[rgb]{0.02,0.16,0.45}{##1}}}
\expandafter\def\csname PYG@tok@nv\endcsname{\def\PYG@tc##1{\textcolor[rgb]{0.73,0.38,0.84}{##1}}}
\expandafter\def\csname PYG@tok@s1\endcsname{\def\PYG@tc##1{\textcolor[rgb]{0.25,0.44,0.63}{##1}}}
\expandafter\def\csname PYG@tok@ch\endcsname{\let\PYG@it=\textit\def\PYG@tc##1{\textcolor[rgb]{0.25,0.50,0.56}{##1}}}
\expandafter\def\csname PYG@tok@m\endcsname{\def\PYG@tc##1{\textcolor[rgb]{0.13,0.50,0.31}{##1}}}
\expandafter\def\csname PYG@tok@gp\endcsname{\let\PYG@bf=\textbf\def\PYG@tc##1{\textcolor[rgb]{0.78,0.36,0.04}{##1}}}
\expandafter\def\csname PYG@tok@sh\endcsname{\def\PYG@tc##1{\textcolor[rgb]{0.25,0.44,0.63}{##1}}}
\expandafter\def\csname PYG@tok@ow\endcsname{\let\PYG@bf=\textbf\def\PYG@tc##1{\textcolor[rgb]{0.00,0.44,0.13}{##1}}}
\expandafter\def\csname PYG@tok@sx\endcsname{\def\PYG@tc##1{\textcolor[rgb]{0.78,0.36,0.04}{##1}}}
\expandafter\def\csname PYG@tok@bp\endcsname{\def\PYG@tc##1{\textcolor[rgb]{0.00,0.44,0.13}{##1}}}
\expandafter\def\csname PYG@tok@c1\endcsname{\let\PYG@it=\textit\def\PYG@tc##1{\textcolor[rgb]{0.25,0.50,0.56}{##1}}}
\expandafter\def\csname PYG@tok@o\endcsname{\def\PYG@tc##1{\textcolor[rgb]{0.40,0.40,0.40}{##1}}}
\expandafter\def\csname PYG@tok@kc\endcsname{\let\PYG@bf=\textbf\def\PYG@tc##1{\textcolor[rgb]{0.00,0.44,0.13}{##1}}}
\expandafter\def\csname PYG@tok@c\endcsname{\let\PYG@it=\textit\def\PYG@tc##1{\textcolor[rgb]{0.25,0.50,0.56}{##1}}}
\expandafter\def\csname PYG@tok@mf\endcsname{\def\PYG@tc##1{\textcolor[rgb]{0.13,0.50,0.31}{##1}}}
\expandafter\def\csname PYG@tok@err\endcsname{\def\PYG@bc##1{\setlength{\fboxsep}{0pt}\fcolorbox[rgb]{1.00,0.00,0.00}{1,1,1}{\strut ##1}}}
\expandafter\def\csname PYG@tok@mb\endcsname{\def\PYG@tc##1{\textcolor[rgb]{0.13,0.50,0.31}{##1}}}
\expandafter\def\csname PYG@tok@ss\endcsname{\def\PYG@tc##1{\textcolor[rgb]{0.32,0.47,0.09}{##1}}}
\expandafter\def\csname PYG@tok@sr\endcsname{\def\PYG@tc##1{\textcolor[rgb]{0.14,0.33,0.53}{##1}}}
\expandafter\def\csname PYG@tok@mo\endcsname{\def\PYG@tc##1{\textcolor[rgb]{0.13,0.50,0.31}{##1}}}
\expandafter\def\csname PYG@tok@kd\endcsname{\let\PYG@bf=\textbf\def\PYG@tc##1{\textcolor[rgb]{0.00,0.44,0.13}{##1}}}
\expandafter\def\csname PYG@tok@mi\endcsname{\def\PYG@tc##1{\textcolor[rgb]{0.13,0.50,0.31}{##1}}}
\expandafter\def\csname PYG@tok@kn\endcsname{\let\PYG@bf=\textbf\def\PYG@tc##1{\textcolor[rgb]{0.00,0.44,0.13}{##1}}}
\expandafter\def\csname PYG@tok@cpf\endcsname{\let\PYG@it=\textit\def\PYG@tc##1{\textcolor[rgb]{0.25,0.50,0.56}{##1}}}
\expandafter\def\csname PYG@tok@kr\endcsname{\let\PYG@bf=\textbf\def\PYG@tc##1{\textcolor[rgb]{0.00,0.44,0.13}{##1}}}
\expandafter\def\csname PYG@tok@s\endcsname{\def\PYG@tc##1{\textcolor[rgb]{0.25,0.44,0.63}{##1}}}
\expandafter\def\csname PYG@tok@kp\endcsname{\def\PYG@tc##1{\textcolor[rgb]{0.00,0.44,0.13}{##1}}}
\expandafter\def\csname PYG@tok@w\endcsname{\def\PYG@tc##1{\textcolor[rgb]{0.73,0.73,0.73}{##1}}}
\expandafter\def\csname PYG@tok@kt\endcsname{\def\PYG@tc##1{\textcolor[rgb]{0.56,0.13,0.00}{##1}}}
\expandafter\def\csname PYG@tok@sc\endcsname{\def\PYG@tc##1{\textcolor[rgb]{0.25,0.44,0.63}{##1}}}
\expandafter\def\csname PYG@tok@sb\endcsname{\def\PYG@tc##1{\textcolor[rgb]{0.25,0.44,0.63}{##1}}}
\expandafter\def\csname PYG@tok@k\endcsname{\let\PYG@bf=\textbf\def\PYG@tc##1{\textcolor[rgb]{0.00,0.44,0.13}{##1}}}
\expandafter\def\csname PYG@tok@se\endcsname{\let\PYG@bf=\textbf\def\PYG@tc##1{\textcolor[rgb]{0.25,0.44,0.63}{##1}}}
\expandafter\def\csname PYG@tok@sd\endcsname{\let\PYG@it=\textit\def\PYG@tc##1{\textcolor[rgb]{0.25,0.44,0.63}{##1}}}

\def\PYGZbs{\char`\\}
\def\PYGZus{\char`\_}
\def\PYGZob{\char`\{}
\def\PYGZcb{\char`\}}
\def\PYGZca{\char`\^}
\def\PYGZam{\char`\&}
\def\PYGZlt{\char`\<}
\def\PYGZgt{\char`\>}
\def\PYGZsh{\char`\#}
\def\PYGZpc{\char`\%}
\def\PYGZdl{\char`\$}
\def\PYGZhy{\char`\-}
\def\PYGZsq{\char`\'}
\def\PYGZdq{\char`\"}
\def\PYGZti{\char`\~}
% for compatibility with earlier versions
\def\PYGZat{@}
\def\PYGZlb{[}
\def\PYGZrb{]}
\makeatother

\renewcommand\PYGZsq{\textquotesingle}

\begin{document}

\maketitle
\tableofcontents
\phantomsection\label{index::doc}


Contents:


\chapter{License and Citations}
\label{license:welcome-to-despotic-s-documentation}\label{license:license-and-citations}\label{license::doc}
DESPOTIC is distributed under the terms of the \href{http://www.gnu.org/copyleft/gpl.html}{GNU General Public License version 3.0}. The text of the license is included in the main directory of the repository as \code{GPL-3.0.txt}.

If you cite DESPOTIC in any published work, please cite the code paper: \href{http://adsabs.harvard.edu/abs/2014MNRAS.437.1662K}{DESPOTIC -- A New Software Library to Derive the Energetics and SPectra of Optically Thick Interstellar Clouds, Krumholz, M. R. 2014, Monthly Notices of the Royal Astronomical Society, 437, 1662}.


\chapter{Installing DESPOTIC}
\label{installation::doc}\label{installation:installing-despotic}
DESPOTIC is distributed in two ways.


\section{Installing from git}
\label{installation:installing-from-git}
The full source distribution, including example Python programs using it and sample cloud descriptor files, is available from \href{https://bitbucket.org/krumholz/despotic/}{bitbucket}, and can be obtained via git by doing:

\begin{Verbatim}[commandchars=\\\{\}]
git clone git@bitbucket.org:krumholz/despotic.git
\end{Verbatim}

The package can then be installed by doing:

\begin{Verbatim}[commandchars=\\\{\}]
python setup.py install
\end{Verbatim}

in the \code{despotic} directory. You may need to preface this with
\code{sudo} if you want to install in a way that is accessible for all
users on the system.


\section{Installing from pip}
\label{installation:installing-from-pip}
DESPOTIC is also available through the python package index. If you
have pip installed, you can just type:

\begin{Verbatim}[commandchars=\\\{\}]
pip install despotic
\end{Verbatim}

As with the setup via \code{git}, you may need to preface this with
\code{sudo} to install globally.


\section{Setting Up the Environment}
\label{installation:setting-up-the-environment}
DESPOTIC creates a local cache of atomic and molecular data files. You
can specify the location of this cache by setting the
\code{\$DESPOTIC\_HOME} environment variable; the data cache will be
created in \code{\$DESPOTIC\_HOME/LAMDA} If you are using a \code{bash}-like
shell, the syntax to set the location of \code{\$DESPOTIC\_HOME} is:

\begin{Verbatim}[commandchars=\\\{\}]
export DESPOTIC\PYGZus{}HOME = /path/to/despotic
\end{Verbatim}

while for a \code{csh}-like shell, it is:

\begin{Verbatim}[commandchars=\\\{\}]
setenv DESPOTIC\PYGZus{}HOME /path/to/despotic
\end{Verbatim}

If \code{\$DESPOTIC\_HOME} is not set, then DESPOTIC will attempt to create
the cache in the directory from which it is run.


\section{Requirements and Dependencies}
\label{installation:requirements-and-dependencies}
DESPOTIC requires
\begin{itemize}
\item {} 
\titleref{scipy \textgreater{}= 0.11.0}

\item {} 
\titleref{cython \textgreater{}= 0.20.x}

\item {} 
\titleref{matplotlib \textgreater{}= 1.3.x}

\end{itemize}


\section{Installing the \texttt{despotic.winds} Module}
\label{installation:ssec-winds-installation}\label{installation:installing-the-despotic-winds-module}
The \code{despotic.winds} module (see {\hyperref[winds:sec\string-winds]{\crossref{\DUrole{std,std-ref}{Winds}}}}) is written in
C++, and makes use of the \href{http://www.gnu.org/software/gsl/}{GNU scientific library}. Users who wish to make use of
this package will need to compile the C++ portions of the
software. This process should be automatic for users who have an
installed C++ compiler and standard build tools (\code{make}), and who
have the \href{http://www.gnu.org/software/gsl/}{GSL} headers and library
installed in their \code{CXX\_INCLUDE\_PATH} and \code{LIBRARY\_PATH},
respectively.

If this does not work, users should be able to build the required C++
library by changing into the \code{despotic/winds} directory and doing
\code{make}. Users can edit the \code{Makefile} in \code{despotic/winds} as
necessary to tailor the build for their environment.

The \code{winds} module is not required for the operation of the rest of
the code, so users who are not planning to make use of this capability
can safely ignore it.


\chapter{Quickstart}
\label{quickstart::doc}\label{quickstart:quickstart}

\section{Introduction}
\label{quickstart:introduction}
DESPOTIC is a tool to Derive the Energetics and SPectra of
Optically Thick Interstellar Clouds. It can perform a
variety of calculations regarding the chemical and thermal state of
interstellar clouds, and predict their observable line
emission. DESPOTIC treats clouds in a simple one-zone model, and is
intended to allow rapid, interactive exploration of parameter space.

In this Quickstart, we will walk through a basic interactive python
session using DESPOTIC. This will work equally well from an ipython
shell or in an ipython notebook.


\section{The \texttt{cloud} Class}
\label{quickstart:the-cloud-class}
The basic object in DESPOTIC, which provides an interface to most of
its functionality, is the class \code{cloud}. This class stores the basic
properties of an interstellar cloud, and provides methods to perform
calculations on those properties. The first step in most DESPOTIC
sessions is to import this class:

\begin{Verbatim}[commandchars=\\\{\}]
from despotic import cloud
\end{Verbatim}

The next step is generally to input data describing a cloud upon which
computations are to be performed. The input data describe the cloud's
physical properties (density, temperature, etc.), the bulk composition
of the cloud, what emitting species it contains, and the radiation
field around it. While it is possible to enter the data manually, it
is usually easier to read the data from a file, using the
{\hyperref[cloudfiles:sec\string-cloudfiles]{\crossref{\DUrole{std,std-ref}{Cloud Files}}}} format. For this Quickstart, we'll use one of
the configuration files that ship with DESPOTIC and that are
included in the \code{cloudfiles} subdirectory of the DESPOTIC
distribution. To create a cloud whose properties are as given in
a particular cloud file, we simply invoke the constructor with
the \code{fileName} optional argument set equal to a string containing
the name of the file to be read:

\begin{Verbatim}[commandchars=\\\{\}]
gmc = cloud(fileName=\PYGZdq{}cloudfiles/MilkyWayGMC.desp\PYGZdq{}, verbose=True)
\end{Verbatim}

Note that, if you're not running DESPOTIC from the directory where you
installed it, you'll need to include the full path to the \code{cloudfiles}
subdirectory in this command. Also note the optional argument
\code{verbose}, which we have set to \code{True}. Most DESPOTIC methods
accept the \code{verbose} argument, which causes them to produce printed
output containing a variety of information. By default DESPOTIC
operations are silent.


\section{Computing Temperatures}
\label{quickstart:computing-temperatures}
At this point most of the calculations one could want to do on a cloud
are provided as methods of the \code{cloud} class. One of the most basic is
to set the cloud to its equilibrium dust and gas temperatures. This is
accomplished via the \code{setTempEq} method:

\begin{Verbatim}[commandchars=\\\{\}]
gmc.setTempEq(verbose=True)
\end{Verbatim}

With \code{verbose} set to \code{True}, this command will produce variety of
output as it iterates to calculate the equilibrium gas and dust
temperatures, before finally printing \code{True}. This illustrates
another feature of DESPOTIC commands: those that iterate return a
value of \code{True} if they converge, and \code{False} if they do not.

To see the gas and dust temperatures to which the cloud has been set,
we can simply print them:

\begin{Verbatim}[commandchars=\\\{\}]
print gmc.Tg
print gmc.Td
\end{Verbatim}

This shows that DESPOTIC has calculated an equilibrium gas temperature
of 10.2 K, and an equilibrium dust temperature of 14.4 K.


\section{Line Emission}
\label{quickstart:line-emission}
Next we might wish to compute the CO line emission emerging from the
cloud. We do this with the \code{cloud} method \code{lineLum}:

\begin{Verbatim}[commandchars=\\\{\}]
lines = gmc.lineLum(\PYGZdq{}co\PYGZdq{})
\end{Verbatim}

The argument \code{co} specifies that we are interested in the emission
from the CO molecule. This method returns a \code{list} of \code{dict}, each
of which gives information about one of the CO lines. The \code{dict}
contains a variety of fields, but one of them is the
velocity-integrated brightness temperature of the line. Again, we can
just print the values we want. The first element in the list is the
\(J = 1 \rightarrow 0\) line, and the velocity-integrated
brightness temperature is listed as \code{intTB} in the \code{dict}. Thus to
get the velocity-integrated brightness temperature of the first line,
we just do:

\begin{Verbatim}[commandchars=\\\{\}]
print lines[0][’intTB’]
\end{Verbatim}

This shows that the velocity-integrated brightness temperature of the
CO \(J = 1 \rightarrow 0\) line is 79 K km/s.


\section{Heating and Cooling Rates}
\label{quickstart:heating-and-cooling-rates}
Finally, we might wish to know the heating and cooling rates produced
by various processes, which lets us determined what sets the thermal
balance in the cloud. This may be computed using the method \code{dEdt},
as follows:

\begin{Verbatim}[commandchars=\\\{\}]
rates = gmc.dEdt()
\end{Verbatim}

This method returns a \code{dict} that contains all the heating and
cooling terms for gas and dust. For example, we can print the rates of
cosmic ray heating and CO line cooling via:

\begin{Verbatim}[commandchars=\\\{\}]
print rates[\PYGZdq{}GammaCR\PYGZdq{}]
print rates[\PYGZdq{}LambdaLine\PYGZdq{}][\PYGZdq{}co\PYGZdq{}]
\end{Verbatim}


\chapter{Functional Guide to DESPOTIC Capabilities}
\label{functions:functional-guide-to-despotic-capabilities}\label{functions::doc}
This section covers the most common tasks for which DESPOTIC can be
used. They are not intended to provide a comprehensive overview of the
library's capabilities, and users who wish to understand every
available option should refer to {\hyperref[fulldoc:sec\string-fulldoc]{\crossref{\DUrole{std,std-ref}{Full Documentation of All DESPOTIC Classes and Functions}}}}. The routines used in
this section are all described in full detail there. For all of these
operations, the user should first have imported the basic DESPOTIC
class \code{cloud} by doing:

\begin{Verbatim}[commandchars=\\\{\}]
from despotic import cloud
\end{Verbatim}

In the examples below we will also assume that \code{matplotlib} and
\code{numpy} have both been imported, via:

\begin{Verbatim}[commandchars=\\\{\}]
import matplotlib.pyplot as plt
import numpy as np
\end{Verbatim}


\section{Unit Conventions}
\label{functions:unit-conventions}
In this section, and in general when using DESPOTIC, there are two
important conventions to keep in mind.
\begin{enumerate}
\item {} 
All quantities are in CGS units unless otherwise specified. The
main exceptions are quantities that are normalized to Solar or Solar
neighborhood values (e.g. metallicity and interstellar radiation
field strength) and quantities where the conventional unit is
non-CGS (e.g. integrated brightness temperatures are expressed in
the usual units of K km/s).

\item {} 
All rates are expressed per H nucleus, rather than per unit mass or
per unit volume. This includes chemical abundances. Thus for
example a heating rate of \(\Gamma=10^{-26}\) should be
understood as \(10^{-26}\) erg/s/H nucleus. An abundance
\(x_{\mathrm{He}}=0.1\) should be understood as 1 He atom per
10 H nuclei.

\end{enumerate}


\section{Line Emission}
\label{functions:ssec-line-emission}\label{functions:line-emission}
The most basic task in DESPOTIC is computing the line emission
emerging from a cloud of specified physical properties. The first step
in any computation of line emission is to create a cloud object with
the desired properties. This is often most easily done by creating a
DESPOTIC cloud file (see {\hyperref[cloudfiles:sec\string-cloudfiles]{\crossref{\DUrole{std,std-ref}{Cloud Files}}}}), but the user can also
create a cloud with the desired properties manually. The properties
that are important for line emission are the gas volume density,
column density, temperature, velocity dispersion, and chemical
composition; these are stored in the cloud class and the composition
class within it. For example, the following code snippet:

\begin{Verbatim}[commandchars=\\\{\}]
mycloud = cloud()
mycloud.nH = 1.0e2
mycloud.colDen = 1.0e22
mycloud.sigmaNT = 2.0e5
mycloud.Tg = 10.0
mycloud.comp.xoH2 = 0.1
mycloud.comp.xpH2 = 0.4
mycloud.comp.xHe = 0.1
\end{Verbatim}

creates a cloud with all its parameters set to default values, a
volume density of H nuclei \(n_{\mathrm{H}} =
10^2\,\mathrm{cm}^{-3}\), a column density of H nuclei
\(N_{\mathrm{H}} = 10^{22}\,\mathrm{cm}^{-2}\), a non-thermal
velocity dispersion of \(\sigma_{\mathrm{NT}} = 2.0 \times
10^5\,\mathrm{cm}\,\mathrm{s}^{-1}\), a gas temperature of \(T_g =
10\,\mathrm{K}\), and a composition that is 0.1
ortho-\(\mathrm{H}_2\) molecules per H nucleus, 0.4
para-\(\mathrm{H}_2\) molecules, and 0.1 He atoms per H nucleus.

The next step is to specify the emitting species whose line emission
is to be computed. As with the physical properties of the cloud, this
is often most easily done by creating a cloud file. However, it can
also be done manually by using the cloud.addEmitter method, which
allows users to add emitting species to clouds. The following code
snippet adds CO as an emitting species, at an abundance of
\(10^{-4}\) CO molecules per H nucleus:

\begin{Verbatim}[commandchars=\\\{\}]
mycloud.addEmitter(\PYGZdq{}CO\PYGZdq{}, 1.0e\PYGZhy{}4)
\end{Verbatim}

The first argument is the name of the emitting species, and the second
is the abundance. The requisite molecular data file will be read from
disk if it is available, or automatically downloaded from the Leiden
Atomic and Molecular Database if it is not (see {\hyperref[data:sec\string-data]{\crossref{\DUrole{std,std-ref}{Atomic and Molecular Data}}}}).

Once an emitter has been added, only a single call is required to
calculate the luminosity of its lines:

\begin{Verbatim}[commandchars=\\\{\}]
lines = mycloud.lineLum(\PYGZdq{}CO\PYGZdq{})
\end{Verbatim}

The argument is just the name of the species whose line emission
should be computed. Note that it must match the name of an existing
emitter, and that emitter names are case-sensitive. The value returned
by this procedure, which is stored in the variable \code{lines}, is a
\code{list} of \code{dict}, with each \code{dict} describing the properties of
a single line. Lines are ordered by frequency, from lowest to
highest. Each \code{dict} contains the following keys-value pairs
\begin{itemize}
\item {} 
\code{freq} is the line frequency in Hz

\item {} 
\code{intIntensity} is the frequency-integrated intensity of the line
after subtracting off the CMB contribution, in
\(\mathrm{erg}\,\mathrm{cm}^{-2}\,\mathrm{s}^{-1}\,\mathrm{sr}^{-1}\)

\item {} 
\code{intTB} is the velocity-integrated brightness temperature (again
subtracting off the CMB) in
\(\mathrm{K}\,\mathrm{km}\,\mathrm{s}^{-1}\)

\item {} 
\code{lumPerH} is the rate of energy emission in the line per H nucleus
in the cloud, in \(\mathrm{erg}\,\mathrm{s}^{-1}\).

\end{itemize}

This is a partial list of what the \code{dict} contains; see
{\hyperref[fulldoc:sssec\string-full\string-cloud]{\crossref{\DUrole{std,std-ref}{cloud}}}} for a complete listing.

Once the data been obtained, the user can do what he or she wishes
with them. For example, to plot velocity-integrated brightness
temperature versus line frequency, the user might do:

\begin{Verbatim}[commandchars=\\\{\}]
freq = [l[\PYGZdq{}freq\PYGZdq{}] for l in lines]
TB = [l[\PYGZdq{}intTB\PYGZdq{}] for l in lines]
plt.plot(freq, TB, \PYGZdq{}o\PYGZdq{})
\end{Verbatim}


\section{Heating and Cooling Rates}
\label{functions:heating-and-cooling-rates}
To use DESPOTIC's capability to calculate heating and cooling rates,
in addition to the quantities specified for a calculation of line
emission one must also add the quantities describing the dust and the
radiation field. As before, this is most easily accomplished by
creating a DESPOTIC cloud file (see {\hyperref[cloudfiles:sec\string-cloudfiles]{\crossref{\DUrole{std,std-ref}{Cloud Files}}}}), but the
data can also be input manually. The code snippet below does so:

\begin{Verbatim}[commandchars=\\\{\}]
mycloud.dust.alphaGD   = 3.2e\PYGZhy{}34    \PYGZsh{} Dust\PYGZhy{}gas coupling coefficient
mycloud.dust.sigma10   = 2.0e\PYGZhy{}25    \PYGZsh{} Cross section for 10K thermal radiation
mycloud.dust.sigmaPE   = 1.0e\PYGZhy{}21    \PYGZsh{} Cross section for photoelectric heating
mycloud.dust.sigmaISRF = 3.0e\PYGZhy{}22    \PYGZsh{} Cross section to the ISRF
mycloud.dust.beta      = 2.0        \PYGZsh{} Dust spectral index
mycloud.dust.Zd        = 1.0        \PYGZsh{} Abundance relative to Milky Way
mycloud.Td             = 10.0       \PYGZsh{} Dust temperature
mycloud.rad.TCMB       = 2.73       \PYGZsh{} CMB temperature
mycloud.rad.TradDust   = 0.0        \PYGZsh{} IR radiation field seen by the dust
mycloud.rad.ionRate    = 2.0e\PYGZhy{}17    \PYGZsh{} Primary ionization rate
mycloud.rad.chi        = 1.0        \PYGZsh{} ISRF normalized to Solar neighborhood
\end{Verbatim}

These quantities specify the dust-gas coupling constant, the dust
cross section to 10 K thermal radiation, the dust cross section to the
8 - 13.6 eV photons the dominate photoelectric heating, the dust cross
section to the broader interstellar radiation field responsible for
heating the dust, the dust spectral index, the dust abundance relative
to the Milky Way value, the dust temperature, the cosmic microwave
background temperature, the infrared radiation field that heats the
dust, the primary ionization rate due to cosmic rays and x-rays, and
the ISRF strength normalized to the Solar neighborhood value. All of
the numerical values shown in the code snippet above are in fact the
defaults, and so none of the above commands are strictly
necessary. However, it is generally wise to set quantities explicitly
rather than relying on default values.

Once these data have been input, one may compute all the heating and
cooling terms that DESPOTIC includes using the \code{cloud.dEdt} routine:

\begin{Verbatim}[commandchars=\\\{\}]
rates = mycloud.dEdt()
\end{Verbatim}

This call returns a dict which contains the instantaneous rates of
heating and cooling. The entries in the dict are: \code{GammaPE}, the gas
photoelectric heating rate, \code{GammaCR}, the gas heating rate due to
cosmic ray and X-ray ionization, \code{GammaGrav}, the gas heating rate due
to gravitational compression, \code{GammaDustISRF}, the dust heating rate
due to the ISRF, \code{GammaDustCMB}, the dust heating rate due to the CMB,
\code{GammaDustIR}, the dust heating rate due to the IR field,
\code{GammaDustLine}, the dust heating rate due to absorption of line
photons, \code{PsiGD}, the gas-dust energy exchange rate (positive means
gas heating, dust cooling), \code{LambdaDust}, the dust cooling rate via
thermal emission, and \code{LambdaLine}, the gas cooling rate via line
emission. This last quantity is itself a dict, with one entry per
emitting species and the dictionary keys corresponding to the emitter
names. Thus in the above example, one could see the cooling rate via
CO emission by doing:

\begin{Verbatim}[commandchars=\\\{\}]
print rates[\PYGZdq{}LambdaLine\PYGZdq{}][\PYGZdq{}CO\PYGZdq{}]
\end{Verbatim}


\section{Temperature Equilibria}
\label{functions:ssec-temp-eq}\label{functions:temperature-equilibria}
Computing the equilibrium temperature requires exactly the same
quantities as computing the heating and cooling rates; indeed, the
process of computing the equilibrium temperature simply amounts to
searching for values of \(T_g\) and \(T_d\) such that the sum
of the heating and cooling rates returned by \code{cloud.dEdt} are zero. One
may perform this calculation using the \code{cloud.setTempEq} method:

\begin{Verbatim}[commandchars=\\\{\}]
mycloud.setTempEq()
\end{Verbatim}

This routine iterates to find the equilibrium gas and dust
temperatures, and returns True if the iteration converges. After this
call, the computed dust and gas temperatures may simply be read off:

\begin{Verbatim}[commandchars=\\\{\}]
print mycloud.Td, mycloud.Tg
\end{Verbatim}

The \code{cloud.setTempEq} routine determines the dust and gas
temperatures simultaneously. However, there are many situations where
it is preferable to solve for only one of these two, while leaving the
other fixed. This may be accomplished by the calls:

\begin{Verbatim}[commandchars=\\\{\}]
mycloud.setDustTempEq()
mycloud.setGasTempEq()
\end{Verbatim}

These routines, respectively, set \code{mycloud.Td} while leaving
\code{mycloud.Tg} fixed, or vice-versa. Solving for one temperature at a
time is often faster, and if dust-gas coupling is known to be
negligible will produce nearly identical results as solving for the
two together.


\section{Time-Dependent Temperature Evolution}
\label{functions:ssec-temp-evol}\label{functions:time-dependent-temperature-evolution}
To perform computations of time-dependent temperature evolution,
DESPOTIC provides the method \code{cloud.tempEvol}. In its most basic
form, this routine simply accepts an argument specifying the amount of
time for which the cloud is to be integrated, and returning the
temperature as a function of time during this evolution (note that
executing this command may take a few minutes, depending on your
processor):

\begin{Verbatim}[commandchars=\\\{\}]
mycloud.Tg = 50.0         \PYGZsh{} Start the cloud out of equilibrium
tFinal = 20 * 3.16e10     \PYGZsh{} 20 kyr
Tg, t = mycloud.tempEvol(tFinal)
\end{Verbatim}

The two values returned are arrays, the second of which gives a series
of 100 equally-spaced times between 0 and \code{tFinal}, and the first of
which gives the temperatures at those times. The number of output
times, the spacing between them, and their exact values may all be
controlled by optional arguments -- see {\hyperref[fulldoc:sssec\string-full\string-cloud]{\crossref{\DUrole{std,std-ref}{cloud}}}} for
details. At
the end of this evolution, the cloud temperature \code{mycloud.Tg}
will be changed to its value at the end of 20 kyr of evolution, and
the dust temperature \code{mycloud.Tg} will be set to its thermal
equilibrium value at that cloud temperature.

If one wishes to examine the intermediate states in more detail, one
may also request that the full state of the cloud be saved at every
time:

\begin{Verbatim}[commandchars=\\\{\}]
clouds, t = mycloud.tempEvol(tFinal, fullOutput=True)
\end{Verbatim}

The \code{fullOutput} optional argument, if \code{True}, causes the routine
to return a full copy of the state of the cloud at each output time,
instead of just the gas temperature \code{Tg}. In this case, \code{clouds}
is a sequence of 100 \code{cloud} objects, and one may interrogate their
states (e.g. calculating their line emission) using the usual
routines.


\section{Chemical Equilibria}
\label{functions:chemical-equilibria}\label{functions:ssec-chem-eq}
DESPOTIC can also compute the chemical state of clouds from a chemical
network. Full details on chemical networks are given in
{\hyperref[chemistry:sec\string-chemistry]{\crossref{\DUrole{std,std-ref}{Chemistry and Chemical Networks}}}}, but for this example we will use a simple network
that DESPOTIC ships with, that of \href{http://adsabs.harvard.edu/abs/1999ApJ...524..923N}{Nelson \& Langer (1999, ApJ,
524, 923)}. This
network computes the chemistry of carbon and oxygen in a region where
the hydrogen is fully molecular. For more details see
{\hyperref[chemistry:sssec\string-nl99]{\crossref{\DUrole{std,std-ref}{NL99}}}}.

To perform computations with this network, one must first import the
class that defines it:

\begin{Verbatim}[commandchars=\\\{\}]
from despotic.chemistry import NL99
\end{Verbatim}

One can set the equilibrium abundances of a cloud to the equilibrium
values determined by the network via the command:

\begin{Verbatim}[commandchars=\\\{\}]
mycloud.setChemEq(network=NL99)
\end{Verbatim}

The argument \code{network} specifies that the calculation should use the
\code{NL99} class. This call sets the abundances of all the emitters that
are included in the network to their equilibrium values. In this case,
the network includes CO, and thus it sets the CO abundance to a new
value:

\begin{Verbatim}[commandchars=\\\{\}]
print mycloud.emitters[\PYGZdq{}CO\PYGZdq{}].abundance
\end{Verbatim}

One can also see the abundances of all the species included in the
network, including those that do not correspond to emitters in the
cloud, by printing the chemical network property \code{abundances}:

\begin{Verbatim}[commandchars=\\\{\}]
print mycloud.chemnetwork.abundances
\end{Verbatim}

Once the chemical network is associated with the cloud, subsequent
calls to \code{setChemEq} need not include the \code{network}
keyword. DESPOTIC assumes that all subsequent chemical calculations
are to be performed with the same chemical network unless it is
explicitly told otherwise via a call to \code{setChemEq} or
\code{chemEvol} (see {\hyperref[functions:ssec\string-chem\string-time]{\crossref{\DUrole{std,std-ref}{Time-Dependent Chemical Evolution}}}}) that specifies a different
chemical network.


\section{Simultaneous Chemical and Thermal Equilibria}
\label{functions:simultaneous-chemical-and-thermal-equilibria}
The \code{setChemEq} routine (see {\hyperref[functions:ssec\string-chem\string-eq]{\crossref{\DUrole{std,std-ref}{Chemical Equilibria}}}}) called with no
extra arguments leaves the gas temperature fixed. However, it is also
possible to compute a simultaneous equilibrium for the temperature and
the thermal state. To do so, we first import a chemical network to
be used, in this case the Nelson \& Langer (1999) network (see
{\hyperref[functions:ssec\string-chem\string-eq]{\crossref{\DUrole{std,std-ref}{Chemical Equilibria}}}}):

\begin{Verbatim}[commandchars=\\\{\}]
from despotic.chemistry import NL99
\end{Verbatim}

We then call \code{cloud.setChemEq} with an optional keyword
\code{evolveTemp}

\begin{Verbatim}[commandchars=\\\{\}]
mycloud.setChemEq(network=NL99, evolveTemp=\PYG{l+s}{{}`{}`}\PYG{l+s}{iterate}\PYG{l+s}{{}`{}`})
\end{Verbatim}

The \code{network} keyword specifies that the computation should use the
NL99 network, while \code{evolveTemp} specifies how to handle the
simultaneous thermal and chemical equilibrium calculation. The options
available are
\begin{itemize}
\item {} 
\code{fixed}: gas temperature is held fixed

\item {} 
\code{iterate}: calculation iterates between computing chemical and
gas thermal equilibria, i.e., chemical equilibrium is computed at
fixed temperature, equilibrium gas temperature (see
{\hyperref[functions:ssec\string-temp\string-eq]{\crossref{\DUrole{std,std-ref}{Temperature Equilibria}}}}) is computed for fixed abundances, and the
process is repeated until the temperature and abundances converge;
dust temperature is held fixed

\item {} 
\code{iterateDust}: same as \code{iterate}, except the dust temperature is
iterated as well

\item {} 
\code{gasEq}: gas temperature is always set to its instantaneous
equilibrium value as the chemical state is evolved toward
equilibrium; dust temperature is held fixed

\item {} 
\code{fullEq}: same as \code{gasEq}, except that both gas and dust
temperatures are set to their instantaneous equilibrium values

\item {} 
\code{evol}: chemical state and gas temperature are evolved in time
together, while dust temperature is always set to its instantaneous
equilibrium value; evolution stops once gas temperature and
abundances stop changing significantly

\end{itemize}

Note that, while in general the different evolution methods will
converge to the same answer, there is no guarantee that they will do
in systems where multiple equilibria exist.


\section{Time-Dependent Chemical Evolution}
\label{functions:ssec-chem-time}\label{functions:time-dependent-chemical-evolution}
DESPOTIC can also calculate time-dependent chemical evolution. This is
accomplished through the method cloud.chemEvol. At with
\code{cloud.tempEvol} (see {\hyperref[functions:ssec\string-temp\string-evol]{\crossref{\DUrole{std,std-ref}{Time-Dependent Temperature Evolution}}}}), this routine accepts
an argument specifying the amount of time for which the cloud is to be
integrated, and returning the chemical abundances as a function of
time during this evolution:

\begin{Verbatim}[commandchars=\\\{\}]
mycloud.rad.ionRate = 2.0e\PYGZhy{}16 \PYGZsh{} Raise the ionization rate a lot
tFinal = 0.5 * 3.16e13 \PYGZsh{} 0.5 Myr
abd, t = mycloud.chemEvol(tFinal, network=NL99)
\end{Verbatim}

Note that the \code{network=NL99} option may be omitted if one has
previously assigned that network to the cloud (for example by
executing the examples in {\hyperref[functions:ssec\string-chem\string-eq]{\crossref{\DUrole{std,std-ref}{Chemical Equilibria}}}}).

The output quantity abd here is an object of class \code{abundanceDict},
which is a specialized dict for handling chemical abundances -- see
{\hyperref[chemistry:sssec\string-abundancedict]{\crossref{\DUrole{std,std-ref}{abundanceDict}}}}. One can examine the abundances of specific
species just by giving their chemical names. For example, to see
the time-dependent evolution of the abundances of CO, C, and
\(\mathrm{C}^+\), one could do:

\begin{Verbatim}[commandchars=\\\{\}]
plt.plot(t, abd[\PYGZdq{}CO\PYGZdq{}])
plt.plot(t, abd[\PYGZdq{}C\PYGZdq{}])
plt.plot(t, abd[\PYGZdq{}C+\PYGZdq{}])
\end{Verbatim}

As with \code{setChemEq}, this routine modifies the abundances of
emitters in the cloud to the values they achieve at the end of the
evolution, so to see the final CO abundance one could do:

\begin{Verbatim}[commandchars=\\\{\}]
print mycloud.emitters[\PYGZdq{}CO\PYGZdq{}].abundance
\end{Verbatim}


\section{Multi-Zone Clouds}
\label{functions:multi-zone-clouds}
While most DESPOTIC functionality is provided through the \code{cloud}
class, which represents a single cloud, it is sometimes useful to have
a cloud that contains zones of different optical depths. This
functionality is provided through the \code{zonedcloud} class. A
\code{zonedcloud} is just a collection of \code{cloud} objects that are
characterized by having different column densities (and optionally
volume densities), and on which all the operations listed above can be
performed in a batch fashion.

One can create a \code{zonedcloud} in much the same way as a \code{cloud},
but reading from an input file:

\begin{Verbatim}[commandchars=\\\{\}]
from despotic import zonedcloud
zc = zonedcloud(fileName=\PYGZdq{}cloudfiles/MilkyWayGMC.desp\PYGZdq{})
\end{Verbatim}

A \code{zonedcloud} is characterized by column densities for each of its
zones, which can be accessed through the \code{colDen} property:

\begin{Verbatim}[commandchars=\\\{\}]
print zc.colDen
\end{Verbatim}

The column densities of all zones, and the number of zones, can be
controlled when the \code{zonedcloud} is created using the keywords
\code{nZone} and \code{colDen}; see {\hyperref[fulldoc:sssec\string-full\string-zonedcloud]{\crossref{\DUrole{std,std-ref}{zonedcloud}}}} for the
full list of keywords.

Once a \code{zonedcloud} exists, all of the functions described above in
this section are available for it, and will be applied zone by
zone. For example, one can do:

\begin{Verbatim}[commandchars=\\\{\}]
zc.setTempEq()
print zc.Tg
\end{Verbatim}

to set and then print the temperature in each zone. Commands the
report observable quantities or abundances will return
appropriately-weighted sums over the entire cloud. For example:

\begin{Verbatim}[commandchars=\\\{\}]
zc.lineLum(\PYGZsq{}co\PYGZsq{})[0]
\end{Verbatim}

returns a dict describing the \(J=1\rightarrow 0\) line of CO. The
quantities \code{intTB} and \code{intIntensity} that are part of the dict
and contain the velocity-integrated brightness temperature and
frequency-integrated intensity, respectively (see
{\hyperref[functions:ssec\string-line\string-emission]{\crossref{\DUrole{std,std-ref}{Line Emission}}}}), are sums over all zones, while ones
like \code{Tex} (the excitation temperature) that do not make sense to
sum are returned as an array giving zone-by-zone values.


\section{Computing Line Profiles}
\label{functions:computing-line-profiles}
Line profile computation operates somewhat differently then the
previous examples, because it is provided through a stand-alone
procedure rather than through the cloud class. This procedure is
called lineProfLTE, and may be imported directly from the DESPOTIC
package. The routine also requires emitter data stored in an
\code{emitterData} object. The first step in a line profile calculation is
therefore to import these two objects into the python environment:

\begin{Verbatim}[commandchars=\\\{\}]
from despotic import lineProfLTE
from despotic import emitterData
\end{Verbatim}

The second step is to read in the emitter data. The interface to read
emitter data is essentially identical to the one used to add an
emitter to a cloud. One simply declares an \code{emitterData} object,
giving the name of the emitter as an argument:

\begin{Verbatim}[commandchars=\\\{\}]
csData = emitterData(’CS’) \PYGZsh{} Reads emitter data for the CS molecule
\end{Verbatim}

Alternately, emitter data may be obtained from a \code{cloud}, since clouds
store emitter data for all their emitters. Using the examples from the
previous sections:

\begin{Verbatim}[commandchars=\\\{\}]
coData = mycloud.emitters[\PYGZdq{}CO\PYGZdq{}].data
\end{Verbatim}

copies the emitter data for CO to the variable \code{coData}.

The third step is to specify the radius of the cloud, and the profiles
of any quantities within the cloud that are to change with radius,
including density, temperature, radial velocity, and non-thermal
velocity dispersion. Each of these can be constant, but the most
interesting applications are when one or more of them are not, in
which case they must be defined by functions. These function each take
a single argument, the radius in units where the outer radius of the
cloud is unity, and return a single floating point value, giving the
quantity in question in CGS units. For example, to compute line
profiles through a cloud of spatially-varying temperature and infall
velocity, one might define the functions:

\begin{Verbatim}[commandchars=\\\{\}]
R = 0.02 * 3.09e18 \PYGZsh{} 0.2 pc
def TProf(r):
    return 8.0 + 12.0\PYG{g+ge}{*np.exp(\PYGZhy{}r*}\PYG{g+ge}{*2/(2.0*}0.5**2))
def vProf(r):
    return \PYGZhy{}4.0e4*r
\end{Verbatim}

The first function sets a temperature that varies from 20 K in the
center of close to 8 K at the outer edge, and the second defines a
velocity that varies from 0 in the center to \(-0.4\) km
\(\mathrm{s}^{-1}\) (where negative indicates infall) at the outer
edge. Similar functions can be defined by density and non-thermal
velocity dispersion if the user so desires. Alternately, the user can
simply define them as constants:

\begin{Verbatim}[commandchars=\\\{\}]
ncs = 0.1       \PYGZsh{} CS density 0.1 cm\PYGZca{}\PYGZhy{}3
sigmaNT = 2.0e4 \PYGZsh{} Non\PYGZhy{}thermal velocity dispersion 0.2 km s\PYGZca{}\PYGZhy{}1
\end{Verbatim}

The final step is to use the \code{lineProfLTE} routine to compute the
brightness temperature versus velocity:

\begin{Verbatim}[commandchars=\\\{\}]
TB, v = lineProfLTE(cs, 2, 1, R, ncs, TProf, vProf, sigmaNT).
\end{Verbatim}

Here the first argument is the emitter data, the second and third are
the upper and lower quantum states between which the line is to be
computed (ordered by energy, with ground state = 0), followed by the
cloud radius, the volume density, the temperature, the velocity, and
the non-thermal velocity dispersion. Each of these quantities can be
either a float or a callable function of one variable, as in the
example above. If it is a float, that quantity is taken to be
constant, independent of radius. This routine returns two arrays, the
first of which is the brightness temperature and the second of which
is the velocity at which that brightness temperature is computed,
relative to line center. These can be examined in any of the usual
numpy ways, for example by plotting them:

\begin{Verbatim}[commandchars=\\\{\}]
plt.plot(v, TB)
\end{Verbatim}

By default the velocity is sampled at 100 values. The routine attempts
to guess a reasonable range of velocities based on the input values of
radial velocity and velocity dispersion, but these defaults may be
overridden by the optional argument \code{vLim}, which is a sequence of
two values giving the lower and upper limits on the velocity:

\begin{Verbatim}[commandchars=\\\{\}]
TB, v = lineProfLTE(cs, 2, 1, R, ncs, TProf, vProf, sigmaNT,
                    vLim=[\PYGZhy{}2e5,2e5]).
\end{Verbatim}

A variety of other optional arguments can be used to control the
velocities at which the brightness temperature is computed. It is also
possible to compute line profiles at positions offset from the
geometric center of the cloud, using the optional argument offset --
see {\hyperref[fulldoc:sssec\string-full\string-lineproflte]{\crossref{\DUrole{std,std-ref}{lineProfLTE}}}}.


\section{Escape Probability Geometries}
\label{functions:escape-probability-geometries}
DESPOTIC supports three possible geometries that can be used when
computing escape probabilities, and which are controlled by the
\code{escapeProbGeom} optional argument. This argument is accepted by all
DESPOTIC functions that use the escape probability formalism,
including all those involving computation of line emission. This
optional argument, if included, must be set equal to one of the three
strings \code{sphere} (the default), \code{slab}, or \code{LVG}. These choices
correspond to spherical geometry, slab geometry, and the large
velocity gradient approximation, respectively.


\chapter{Cloud Files}
\label{cloudfiles:sec-cloudfiles}\label{cloudfiles::doc}\label{cloudfiles:cloud-files}

\section{File Structure}
\label{cloudfiles:file-structure}
DESPOTIC cloud files contain descriptions of clouds that can be read
by the \code{cloud} or \code{zonedcloud} classes, using either the class
constructor or the read method; see {\hyperref[fulldoc:sssec\string-full\string-cloud]{\crossref{\DUrole{std,std-ref}{cloud}}}} for
details. This section contains a description of the format for these
files. It is recommended but not required that cloud files have names
that end in the extension \code{.desp}.

Each line of a cloud file must be blank, contain a comment starting
with the character \code{\#}, or contain a key-value pair formatted as:

\begin{Verbatim}[commandchars=\\\{\}]
\PYG{n}{key} \PYG{o}{=} \PYG{n}{value}
\end{Verbatim}

The line may also contain comments after \code{value}, again beginning
with \code{\#}. Any content after \code{\#} is treated as a comment and is
ignored. DESPOTIC keys are case-insensitive, and whitespace around
keys and values are ignored. For the full list of keys, see
{\hyperref[cloudfiles:tab\string-cloudfiles]{\crossref{\DUrole{std,std-ref}{Cloud file keys and their meanings}}}}. All quantities must
be in CGS units. Key-value pairs may be placed in any order, with the
exception of the key \code{H2opr}, which provides a means of specify the
ratio of ortho-to-para-\(\mathrm{H}_2\), instead of directly
setting the ortho-and para-\(\mathrm{H}_2\) abundances. If this
key is specified, it must precede the key \code{xH2}, which gives the
total \(\mathrm{H}_2\) abundance including both ortho- and para-
species. Not all keys are required to be present. If left unspecified,
most quantities default to a fiducial Milky Way value (if a reasonable
one exists, e.g., for the gas-dust coupling constant and
ISRF strength) or to 0 (if it does not, e.g., for densities and
chemical abundances).


\begin{threeparttable}
\capstart\caption{Cloud file keys and their meanings}\label{cloudfiles:tab-cloudfiles}\label{cloudfiles:id1}
\noindent\begin{tabulary}{\linewidth}{|L|L|L|}
\hline
\textsf{\relax 
Key
} & \textsf{\relax 
Units
} & \textsf{\relax 
Description
}\\
\hline \multicolumn{3}{|l|}{}\\
\hline \multicolumn{3}{|l|}{
Physical Properties
}\\
\hline
nH
 & 
\(\mathrm{cm}^{-3}\)
 & 
Volume density of H nuclei
\\
\hline
colDen
 & 
\(\mathrm{cm}^{-2}\)
 & 
Column density of H nuclei, averaged over area
\\
\hline
sigmaNT
 & 
cm \(\mathrm{s}^{-1}\)
 & 
Non-thermal velocity dispersion
\\
\hline
Tg
 & 
K
 & 
Gas temperature
\\
\hline
Td
 & 
K
 & 
Dust temperature
\\
\hline \multicolumn{3}{|l|}{}\\
\hline \multicolumn{3}{|l|}{
Dust Properties
}\\
\hline
alphaGD
 & 
erg \(\mathrm{cm}^3\) \(\mathrm{K}^{-3/2}\)
 & 
Gas-dust collisional coupling coefficient
\\
\hline
sigmaD10
 & 
\(\mathrm{cm}^2\) \(\mathrm{H}^{-1}\)
 & 
Dust cross section per H to 10 K thermal radiation
\\
\hline
sigmaDPE
 & 
\(\mathrm{cm}^2\) \(\mathrm{H}^{-1}\)
 & 
Dust cross section per H to 8-13.6 eV photons
\\
\hline
sigmaDISRF
 & 
\(\mathrm{cm}^2\) \(\mathrm{H}^{-1}\)
 & 
Dust cross section per H averaged over ISRF
\\
\hline
betaDust
 & 
Dimensionless
 & 
Dust IR spectral index
\\
\hline
Zdust
 & 
Dimensionless
 & 
Dust abundance normalized to Milky Way value
\\
\hline \multicolumn{3}{|l|}{}\\
\hline \multicolumn{3}{|l|}{
Radiation Field Properties
}\\
\hline
TCMB
 & 
K
 & 
CMB temperature
\\
\hline
TradDust
 & 
K
 & 
Temperature of the dust-trapped IR radiation field
\\
\hline
ionRate
 & 
\(\mathrm{s}^{-1}\)
 & 
Primary cosmic ray / x-ray ionization rate
\\
\hline
chi
 & 
Dimensionless
 & 
ISRF strength, normalized to Solar neighborhood
\\
\hline \multicolumn{3}{|l|}{}\\
\hline \multicolumn{3}{|l|}{
Chemical composition
}\\
\hline
emitter
 &  & 
See {\hyperref[cloudfiles:ssec\string-emitters]{\crossref{\DUrole{std,std-ref}{Emitters}}}}
\\
\hline\end{tabulary}

\end{threeparttable}



\section{Emitters}
\label{cloudfiles:emitters}\label{cloudfiles:ssec-emitters}
The \code{emitter} key is more complex than most, and requires special
mention. Lines describing emitters follow the format:

\begin{Verbatim}[commandchars=\\\{\}]
\PYG{n}{emitter} \PYG{o}{=} \PYG{n}{name} \PYG{n}{abundance} \PYG{p}{[}\PYG{n}{noextrap}\PYG{p}{]} \PYG{p}{[}\PYG{n}{energySkip}\PYG{p}{]} \PYG{p}{[}\PYG{n}{file}\PYG{p}{:}\PYG{n}{FILE}\PYG{p}{]} \PYG{p}{[}\PYG{n}{url}\PYG{p}{:}\PYG{n}{URL}\PYG{p}{]}
\end{Verbatim}

Here the brackets indicate optional items, and the optional items may
appear in any order, but must be after the two mandatory ones.

The first mandatory item, \code{name}, gives name of the emitting
molecule or atom. Note that molecule and atom names are case
sensitive, in the sense that DESPOTIC will not assume that \code{co} and
\code{CO} describe the same species. Any string is acceptable for
\code{name}, but if the file or URL containing the data for that species
is not explicitly specified, the name is used to guess the
corresponding file name in the Leiden Atomic and Molecular Database
(LAMDA) -- see {\hyperref[data:sec\string-data]{\crossref{\DUrole{std,std-ref}{Atomic and Molecular Data}}}}. It is therefore generally advisable
to name a species following LAMDA convention, which is that molecules
are specified by their chemical formula, with a number specifying the
atomic weight preceding the if the species is not the most common
isotope. Thus LAMDA refers to \(^{28}\mathrm{Si}^{16}\mathrm{O}\)
(the molecule composed of the most common isotopes) as \code{sio},
\(^{29}\mathrm{Si}^{16}\mathrm{O}\) as \code{29sio}, and
\(^{12}\mathrm{C}^{18}\mathrm{O}\) as \code{c18o}. The automatic
search for files in LAMDA also includes common variants of
the file name used in LAMDA. The actual file name from
which DESPOTIC reads data for a given emitter is stored in
the emitterData class -- see {\hyperref[fulldoc:sssec\string-full\string-emitterdata]{\crossref{\DUrole{std,std-ref}{emitterData}}}}.

The second mandatory item, \code{abundance}, gives the abundance of that
species relative to H nuclei. For example, an abundance of 0.1 would
indicate that there is 1 of that species per 10 H nuclei.

The optional items \code{noextrap} and \code{energySkip} change how DESPOTIC
performs computations with that species. If \code{noextrap} is set,
DESPOTIC will raise an error if any attempt is made to calculate a
collision rate coefficient between that species and one of the bulk
components (H, He, etc.) that is outside the range tabulated in the
data file. If not, DESPOTIC will instead handle temperatures outside
the tabulated range by substituting the closest temperature inside the
tabulated range. Note that this behavior can be altered within a
DESPOTIC program by using the \code{extrap} property of the
\code{emitterData} class -- see {\hyperref[fulldoc:sec\string-fulldoc]{\crossref{\DUrole{std,std-ref}{Full Documentation of All DESPOTIC Classes and Functions}}}}.

The optional item \code{energySkip} specifies that a species should be
ignored when computing heating and cooling rates via the
\code{cloud.dEdt} method, and by extension whenever thermal equilibria or
thermal evolution are computed for that cloud. However, line emission
from that species can still be computed using the \code{cloud.lineLum}
method. This option is therefore useful for including species for
which the line emission is an interesting observable, but which are
irrelevant to the thermal balance and thus can be omitted when
calculating cloud thermal properties in order to save computational
time.

Finally, the optional items \code{file:FILE} and \code{url:URL} specify
locations of atomic and molecular data files, either on the local file
system or on the web. This capability is useful in part because some
LAMDA files do not follow the usual naming convention, or because for
some species LAMDA provides more than one version of the data for that
species (e.g., two versions of the data file for atomic C exist, one
with only the low-lying IR levels, and another including the
higher-energy UV levels). File specifications must be of the form
\code{file:FILE} with \code{FILE} replaced by a file name, which can include
both absolute and relative paths. If no path or a relative path is
given, DESPOTIC searches for the file first in the current directory,
and then in the directory \code{\$DESPOTIC\_HOME/LAMDA}, where
\code{\$DESPOTIC\_HOME} is an environment variable. If it is not specified,
DESPOTIC just looks for a directory called LAMDA relative to the
current directory.

The \code{url:URL} option can be used to specify the location of a file
on the web, usually somewhere on the LAMDA website. It must be
specified as \code{url:URL}, where \code{URL} is replaced by an absolute or
relative URL. If an absolute URL is given, DESPOTIC attempts to
download the file from that location. If a relative URL is given,
DESPOTIC attempts to download the file from at
\code{http://\$DESPOTIC\_LAMDAURL/datafiles/URL}, where
\code{\$DESPOTIC\_LAMDAURL} is an environment variable. If this environment
variable is not specified, DESPOTIC searches for the file at
\code{http://home.strw.leidenuniv.nl/\textasciitilde{}moldata/URL}.


\chapter{Atomic and Molecular Data}
\label{data:sec-data}\label{data:atomic-and-molecular-data}\label{data::doc}
DESPOTIC requires atomic and molecular data to work. This section
describes how it handles these data, both on disk and in its internal
workings.


\section{The Local Database}
\label{data:the-local-database}
DESPOTIC uses atomic and molecular data in the format specified by the
\href{http://home.strw.leidenuniv.nl/~moldata/}{Leiden Atomic and Molecular Database}. The user can manually
supply the required data files, but the more common use case is to
access the data directly from LAMDA. When emitter data is required,
DESPOTIC will attempt to guess the name of the required data file and
download it automatically -- see {\hyperref[cloudfiles:ssec\string-emitters]{\crossref{\DUrole{std,std-ref}{Emitters}}}}. When DESPOTIC
downloads a file from \href{http://home.strw.leidenuniv.nl/~moldata/}{LAMDA}, it caches a local copy
for future use. The next time the same emitter is used, unless
DESPOTIC is given an explicit URL from which the file should be
fetched, it will use the local copy instead of re-downloading the file
from LAMDA. (However, see {\hyperref[data:ssec\string-database\string-updates]{\crossref{\DUrole{std,std-ref}{Keeping the Atomic and Molecular Data Up to Date}}}}.)

The location of the database is up to the user, and is specified
through the environment variable \code{\$DESPOTIC\_HOME}. If this
environment variable is set, LAMDA files will be places in the
directory \code{\$DESPOTIC\_HOME/LAMDA}, and that is the default location
that will be searched when a file is needed. If the environment
variable \code{\$DESPOTIC\_HOME} is not set, DESPOTIC looks for files in a
subdirectory LAMDA of the current working directory, and caches files
in that directory if they are downloaded. It is recommended that users
set a \code{\$DESPOTIC\_HOME} environment variable when working with
DESPOTIC, so as to avoid downloading and caching multiple copies of
LAMDA for different projects in different directories.


\section{Keeping the Atomic and Molecular Data Up to Date}
\label{data:keeping-the-atomic-and-molecular-data-up-to-date}\label{data:ssec-database-updates}
The data in LAMDA are updated regularly as new calculations or
laboratory experiments are published. Some of these updates add new
species, but some also provide improved data on species that are
already in the database. DESPOTIC attempts to ensure that its locally
cached data are up to date by putting an expiration date on them. By
default, if DESPOTIC discovers that a given data file is more than six
months old, it will re-download that file from LAMDA. This behavior
can be overridden by manually specifying a file name, either in the
cloud file (see {\hyperref[cloudfiles:sec\string-cloudfiles]{\crossref{\DUrole{std,std-ref}{Cloud Files}}}}) or when invoking
the \code{cloud.addEmitter} or \code{emitter.\_\_init\_\_} methods. Users
can also force updates of the local database more frequently using the
\code{refreshLamda} function -- see {\hyperref[fulldoc:sssec\string-full\string-refreshlamda]{\crossref{\DUrole{std,std-ref}{refreshLamda}}}}.


\section{DESPOTIC's Internal Model for Atomic and Molecular Data}
\label{data:ssec-database-internal}\label{data:despotic-s-internal-model-for-atomic-and-molecular-data}
When it is running, DESPOTIC maintains a list of emitting species for
which data have been read within the \code{emitter} module (see
{\hyperref[fulldoc:sssec\string-full\string-emitter]{\crossref{\DUrole{std,std-ref}{emitter}}}}). Whenever a new emitter is created, either for an
existing cloud, for a new cloud being created, or as a free-standing
object of the emitter class, DESPOTIC checks the emitter name against
the central list. If the name is found in the list, DESPOTIC will
simply use the stored data for that object rather than re-reading the
file containing the data. This is done as an efficiency measure, and
also to ensure consistency between emitters of the same species
associated with different clouds. However, this model has some
important consequences of which the user should be aware.
\begin{enumerate}
\item {} 
Since data on level structure, collision rates, etc. (everything
stored in the \code{emitterData} class -- see {\hyperref[fulldoc:sssec\string-full\string-emitterdata]{\crossref{\DUrole{std,std-ref}{emitterData}}}}) is
shared between all emitters of the same name, and any alterations
made to the data for one emitter will affect all others of the same
name.

\item {} 
It is not possible to have two emitters of the same name but with
different data. Should a user desire to achieve this for some
reason (e.g., to compare results computed using an older LAMDA file
and a newer one), the way to achieve this is to give the two
emitters different names, such as \code{co\_ver1} and \code{co\_ver2}.

\item {} 
Maintenance of a central emitter list affects how deepcopy and
pickling operations operate on emitters. See
{\hyperref[fulldoc:sssec\string-full\string-emitterdata]{\crossref{\DUrole{std,std-ref}{emitterData}}}} for details.

\end{enumerate}


\chapter{Chemistry and Chemical Networks}
\label{chemistry::doc}\label{chemistry:chemistry-and-chemical-networks}\label{chemistry:sec-chemistry}
The chemistry capabilities in DESPOTIC are located in the
\code{despotic.chemistry} module.


\section{Operations on Chemical Networks}
\label{chemistry:operations-on-chemical-networks}\label{chemistry:ssec-operations}
The central object in a chemical calculation is a chemical network,
which consists of a set of chemical species and a set of reactions
that change the concentrations of each of them. Formally, a network is
defined by a set of \(N\) species with abundances
\(\mathbf{x}\) (defined as number density of each species per H
nucleus), and a function \(d\mathbf{x}/dt = \mathbf{f}(\mathbf{x},
\mathbf{p})\) that gives the time rate of change of the abundances. The
vector \(\mathbf{p}\) specifies a set of parameters on which the
reaction rates depend, and it may include quantities such as the
ambient radiation field (for photoreactions), the density, the
temperature, etc. The numerical implementation of chemical networks is
described in {\hyperref[chemistry:ssec\string-chemnetworks]{\crossref{\DUrole{std,std-ref}{Defining New Chemical Networks: the chemNetwork Class}}}}. Given any chemical network,
DESPOTIC is capable of two operations:
\begin{itemize}
\item {} 
Given an initial set of abundances at \(\mathbf{x}(t_0)\) at
time \(t_0\), compute the abundances at some later time
\(t_1\). This is simply a matter of numerically integrating the
ordinary differential equation \(d\mathbf{x}/dt =
\mathbf{f}(\mathbf{x},\mathbf{p})\) from \(t_0\) to \(t_1\),
where \(\mathbf{f}\) is a known function. In DESPOTIC, this
capability is implemented by the routine \code{chemEvol} in the
\code{despotic.chemistry.chemEvol} module. The \code{cloud} class
contains a wrapper around this routine, which allows it to be
called to operate on a specific instance of \code{cloud} or
\code{zonedcloud}.

\item {} 
Find an equilibrium set of abundances
\(\mathbf{x}_{\mathrm{eq}}\) such that \(d\mathbf{x}/dt =
\mathbf{f}(\mathbf{x}_{\mathrm{eq}}, \mathbf{p}) = 0\). Note that
these is in general no guarantee that such an equilibrium exists, or
that it is unique, and there are no general techniques for
identifying such equilibria for arbitrary vector functions
\(\mathbf{f}\). DESPOTIC handles this problem by explicitly
integrating the ODE \(d\mathbf{x}/dt =
\mathbf{f}(\mathbf{x},\mathbf{p})\) until \(\mathbf{x}\) reaches
constant values (within some tolerance) or until a specified maximum
time is reached. In DESPOTIC, this capability is implemented by
the routine \code{setChemEq} in the \code{despotic.chemistry.setChemEq}
module. The \code{cloud} and \code{zonedcloud} classed contains a wrapper
around this routine, which allows it to be called to operate on a
specific instance of \code{cloud} or \code{zonedcloud}.

\end{itemize}


\section{Predefined Chemical Networks}
\label{chemistry:predefined-chemical-networks}\label{chemistry:ssec-predefined-networks}
DESPOTIC ships with two predefined chemical networks, described below.


\subsection{\texttt{NL99}}
\label{chemistry:nl99}\label{chemistry:sssec-nl99}
The \code{NL99} network implements the reduced C-O network introduced by
\href{http://adsabs.harvard.edu/abs/1999ApJ...524..923N}{Nelson \& Langer (1999, Astrophysical Journal, 524, 923; hereafter
NL99)}. Readers
are refereed to that paper for a full description of the network and
the physical approximations on which it relies. To summarize briefly
here, the network is intended to capture the chemistry of carbon and
oxygen as it occurs at moderate densities and low temperatures in
\(\mathrm{H}_2\) dominated clouds. It includes the species C,
\(\mathrm{C}^+\), CHx, CO, \(\mathrm{HCO}^+\),
\(\mathrm{H}^+_3\), \(\mathrm{He}^+\), O, OHx, M, and
\(\mathrm{M}^+\). Several of these are ``super-species'' that
agglomerate several distinct species with similar reaction rates and
pathways, including CHx (where \(x = 1-4\)), OHx (where \(x =
1-2\)), and M and \(\mathrm{M}^+\) (which are stand-ins for metals
such as iron and nickel). The network involves two-body reactions
among these species, as well as photochemical reactions induces by UV
from the ISRF and reactions initiated by cosmic ray ionizations. In
addition to the initial abundances of the various species, the network
depends on the ISRF, the ionization rate, and the total abundances of
C and O nuclei.

In implementing the NL99 network in DESPOTIC there are a three design
choices to be made. First, photochemical and ionization reactions
depend on the UV radiation field strength and the ionization
rate. When performing computations on a cloud, DESPOTIC takes these
from the parameters \code{chi} and \code{ionRate} that are part of the radiation
class attached to the cloud.

Second, photochemical reactions depend on the amount of shielding
against the ISRF provided by dust, and, in the case of the reaction
\(\mathrm{CO}\rightarrow\mathrm{C}+\mathrm{O}\), line shielding by
CO and \(\mathrm{H}_2\). Following its usual approximation for
implementing such shielding in a one-zone model, DESPOTIC takes the
relevant column density to be \(N_{\mathrm{H}}/2\), where
\(N_\mathrm{H}\) is the column density colDen of the cloud, so
that the typical amount of shielding is assumed to correspond to
half the area-averaged column density. For the dust shielding,
NL99 express the shielding in terms of the V-band extinction
\(A_V\); unless instructed otherwise, DESPOTIC computes this via
\begin{equation*}
\begin{split}A_V = 0.4 \sigma_{\mathrm{PE}}(N_{\mathrm{H}}/2).\end{split}
\end{equation*}
This ratio of V-band to 100 nm extinction is intermediate
between the values expected for Milky Way and SMC dust opacity curves,
as discussed in \href{http://adsabs.harvard.edu/abs/2011ApJ...731...25K}{Krumholz, Leroy, \& McKee (2011, Astrophysical
Journal, 731, 25)}. However, the
user may override this choice. For line shielding, DESPOTIC computes
the \(\mathrm{H}_2\) and CO column densities
\begin{align*}\!\begin{aligned}
N_{\mathrm{H}_2} = x_{\mathrm{H}_2} N_{\mathrm{H}}/2\\
N_\mathrm{CO} = x_\mathrm{CO} N_\mathrm{H} /2,\\
\end{aligned}\end{align*}
which amounts to assuming that the CO and \(\mathrm{H}_2\) are
uniformly distributed. Note that the NL99 network explicitly assumes
\(x_{\mathrm{H}_2} = 0.5\), as no reactions involving atomic H are
included -- see {\hyperref[chemistry:sssec\string-nl99\string-gc]{\crossref{\DUrole{std,std-ref}{NL99\_GC}}}} for a network that does include
hydrogen chemistry. These column densities are then used to find a
shielding factor by interpolating the tabulated values of \href{http://adsabs.harvard.edu/abs/1988ApJ...334..771V}{van Dishoeck
\& Black (1988, Astrophysical Journal, 334, 771)}.

The third choice is how to map between the species included in the
chemistry network and the actual emitting species that are required
to compute line emission, cooling, etc. This is non-trivial both
because the chemical network includes super-species, because the
chemical network does not distinguish between ortho- and para-
sub-species while the rest of DESPOTIC does, and because the network
does not distinguish between different isotopomers of the same
species, while the rest of DESPOTIC does. This does not create
problems in mapping from cloud emitter abundances to the chemical
network, since the abundances can simply be summed, but it does create
a question about how to map output chemical abundances computed by
the network into the abundances of emitters that can be operated on by
the remainder of DESPOTIC. In order to handle this issue, DESPOTIC
makes the following assumptions:
\begin{enumerate}
\item {} 
OHx is divided evenly between OH and \(\mathrm{OH}_2\)

\item {} 
The ratio of ortho- to para- for all species is the same as that of
\(\mathrm{H}_2\)

\item {} 
The abundances ratios of all isotopomers of a species remain fixed
as reactions occur, so, for example, the final abundance ratio of
\(\mathrm{C}^{18}\mathrm{O}\) to
\(\mathrm{C}^{16}\mathrm{O}\) as computed by the chemical
network is always the same as the initial one.

\end{enumerate}


\subsection{\texttt{NL99\_GC}}
\label{chemistry:sssec-nl99-gc}\label{chemistry:nl99-gc}
The \code{NL99\_GC} network is an extension of the \code{NL99} network to
handle hydrogen chemistry as well as carbon and oxygen chemistry,
following the recipe described in \href{http://adsabs.harvard.edu/abs/2012MNRAS.421..116G}{Glover \& Clark (2012, MNRAS,
421, 116)}. This
network effectively uses \code{NL99} for carbon and oxygen (with some
updates to the rate coefficients) and the network of \href{http://adsabs.harvard.edu/abs/2007ApJS..169..239G}{Glover \& Mac Low
(2007, ApJS, 169, 239)}
for the hydrogen chemistry. Self-shielding of hydrogen is handled via
the approximate analytic shielding function of \href{http://adsabs.harvard.edu/abs/1996ApJ...468..269D}{Draine \& Bertoldi
(1996, ApJ, 468, 269)}. Effective
column densities for shielding are computed as in \code{NL99}.
Calculations of hydrogen chemistry are assumed to leave the ratio of
ortho- to para- \(\mathrm{H}_2\) unchanged from the initial value,
or set it to 0.25 if the initial value is undefined (e.g., because the
calculation started with a composition that was all H and no
\(\mathrm{H}_2\)). All other assumptions are completely analogous
to \code{NL99}.


\section{Defining New Chemical Networks: the \texttt{chemNetwork} Class}
\label{chemistry:defining-new-chemical-networks-the-chemnetwork-class}\label{chemistry:ssec-chemnetworks}
DESPOTIC implements chemical networks through the abstract base class
\code{chemNetwork}, which is defined by module
\code{despotic.chemistry.chemNetwork} -- see {\hyperref[fulldoc:sec\string-fulldoc]{\crossref{\DUrole{std,std-ref}{Full Documentation of All DESPOTIC Classes and Functions}}}} for full
details. This class defines the required elements that all chemistry
networks must contain; users who wish to implement their own chemistry
networks must derive them from this class, and must override the class
methods listed below. Users are encouraged to examine the two
{\hyperref[chemistry:ssec\string-predefined\string-networks]{\crossref{\DUrole{std,std-ref}{Predefined Chemical Networks}}}} for examples of how to derive a
chemical network class from \code{chemNetwork}.

For any class \code{cn} derived from \code{chemNetwork}, the user is
required to define the following non-callable traits:
\begin{itemize}
\item {} 
\code{cn.specList}: a list of strings that describes the chemical
species included in the network. The names in \code{cn.specList} can be
arbitrary, and are not used for any purpose other than providing
human-readable labels on outputs. However, it is often convenient to
match the names to the names of emitters, as this makes it
convenient to add the emitters back to the cloud later.

\item {} 
\code{cn.x}: a numpy array of rank 1, with each element specifying the
abundance of a particular species in the network. The number of
elements in the array must match the length of \code{cn.specList}. As
with all abundances in DESPOTIC, abundances must be specified
relative to H nuclei, so that if, for example, \code{x{[}3{]}} is \code{0.1},
this means that there is 1 particle of species 3 per 10 H nuclei.

\item {} 
\code{cn.cloud}: an instance of the \code{cloud} class to which the
chemical network is attached. This can be \code{None}, as chemical
networks can be free-standing at not attached to specified instances
of \code{cloud}. However, much of the functionality of chemical
networks is based around integration with the \code{cloud} class.

\end{itemize}

In addition, a class \code{cn} derived from \code{chemNetwork} must define
the following callable attributes:
\begin{itemize}
\item {} 
\code{cn.\_\_init\_\_(self, cloud=None, info=None)}: this is the
initialization method. It must accept two keyword arguments. The
first, \code{cloud}, is an instanced of the \code{cloud} class to which
this chemical network will be attached. This routine should set
\code{cn.cloud} equal to the input \code{cloud} instance, and it may also
extract information from the input \code{cloud} instances in order to
initialize \code{cn.x} or any other required quantities. The second
keyword argument, \code{info}, is a dict containing any additional
initialization parameters that the chemical network can be or must
be passed upon instantiation.

\item {} 
\code{cn.dxdt(self, xin, time)}: this is a method that computes the
time derivative of the abundances. Given an input numpy array of
abundances \code{xin} (which is the same shape as \code{cn.x}) and a time
\code{time}, it must return a numpy array giving the time derivative of
all abundances in units of \(\mathrm{s}^{-1}.\)

\item {} 
\code{cn.applyAbundances(self, addEmitters=False)}: this is a method to
take the abundances stored in the chemical network and use them to
update the abundances of the corresponding emitters in the \code{cloud}
instance associated with this chemical network. This method is
called at the end of every chemical calculation, and is responsible
for copying information from the chemical network back into the
cloud. The optional Boolean argument \code{addEmitters}, if \code{True},
specifies that the method should attempt to not only alter the
abundances of any emitters associated with the cloud, it should also
attempt to add new emitters that are included in the chemical
network, and whose abundances are to be determined from it. It is up
to the user whether or not to honor this request and implement this
behavior. Failure to do so will not prevent the chemical network
from operating, but should at least be warned so that other users
are not surprised.

\end{itemize}

Finally, the following is an optional attribute of the derived class
\code{cn}:
\begin{itemize}
\item {} 
\code{cn.abundances}: this is a property that returns the abundance
information defined in \code{cn.x} as a object of class
{\hyperref[chemistry:sssec\string-abundancedict]{\crossref{\DUrole{std,std-ref}{abundanceDict}}}}. This class is a utility class that
provides an interface to the chemical abundances in a network
that operates like a Python dict. The property abundances is defined
in the base \code{chemNetwork} class, so it is available by inheritance
regardless of whether the user defines \code{cn.abundances}. However,
the user may find it convenient to override
\code{chemNetwork.abundances} to provide more information, e.g., to
provide abundances of species that are not explicitly evolved in the
network, and are instead derived via conservation relations. Both
the {\hyperref[chemistry:sssec\string-nl99]{\crossref{\DUrole{std,std-ref}{NL99}}}} and {\hyperref[chemistry:sssec\string-nl99\string-gc]{\crossref{\DUrole{std,std-ref}{NL99\_GC}}}} classes do this.

\end{itemize}

Once a chemical network class that defines the above methods has been
defined, that class can be passed as an argument associated with the
\code{network} keyword to the \code{cloud.setChemEq} and \code{cloud.chemEvol}
methods (and their equivalents in \code{zonedcloud}), and these methods
will automatically perform chemical calculations using the input
network.

In setting up chemical networks, it often convenient to make use of
the {\hyperref[chemistry:ssec\string-chemhelpers]{\crossref{\DUrole{std,std-ref}{Helper Modules for Chemical Networks}}}} that are provided.


\section{Helper Modules for Chemical Networks}
\label{chemistry:ssec-chemhelpers}\label{chemistry:helper-modules-for-chemical-networks}
The modules below, implemented in \code{despotic.chemistry}, are intended
as helpers for defining and working with chemical networks.


\subsection{\texttt{abundanceDict}}
\label{chemistry:sssec-abundancedict}\label{chemistry:abundancedict}
The \code{abundanceDict} class provides a dict-like interface to numpy
arrays containing species abundances. The motivation for its existence
is that it is desirable to keep lists of chemical species abundances
in a numpy array for speed of computation, but for humans it is much
more convenient to be able to query and print chemical species by
name, with a dict-like interface. The \code{abundanceDict} class overlays
a dict-like structure on a numpy array, making it possible to combine
the speed of a numpy array with the convenience of a dict.

Usage is simple. To define an \code{abundanceDict} one simply provides a
set of chemical species names and a numpy array containing abundances
of those species:

\begin{Verbatim}[commandchars=\\\{\}]
\PYG{k+kn}{from} \PYG{n+nn}{despotic}\PYG{n+nn}{.}\PYG{n+nn}{chemistry} \PYG{k}{import} \PYG{n}{abundanceDict}

\PYG{c+c1}{\PYGZsh{} specList = list of strings containing species names}
\PYG{c+c1}{\PYGZsh{} x = numpy array containing species abundances}
\PYG{n}{abd} \PYG{o}{=} \PYG{n}{abundanceDict}\PYG{p}{(}\PYG{n}{specList}\PYG{p}{,} \PYG{n}{x}\PYG{p}{)}
\end{Verbatim}

Once created, an \code{abundanceDict} can be handled much like a dict, in
which the species names in \code{specList} are the keys:

\begin{Verbatim}[commandchars=\\\{\}]
\PYG{c+c1}{\PYGZsh{} Print abundance of the species CO}
\PYG{n+nb}{print} \PYG{n}{abd}\PYG{o}{.}\PYG{n}{abundances}\PYG{p}{[}\PYG{l+s+s2}{\PYGZdq{}}\PYG{l+s+s2}{CO}\PYG{l+s+s2}{\PYGZdq{}}\PYG{p}{]}

\PYG{c+c1}{\PYGZsh{} Set the C+ abundance to 1e\PYGZhy{}10}
\PYG{n}{abd}\PYG{p}{[}\PYG{l+s+s2}{\PYGZdq{}}\PYG{l+s+s2}{C+}\PYG{l+s+s2}{\PYGZdq{}}\PYG{p}{]} \PYG{o}{=} \PYG{l+m+mi}{1}\PYG{n}{e}\PYG{o}{\PYGZhy{}}\PYG{l+m+mi}{10}
\end{Verbatim}

These operations will alter the underlying numpy array appropriately;
the array may also be accessed directly, as
\code{abundanceDict.x}. However, the performance penalty for referring to
objects by name rather than by index in \code{abundanceDict.x} is
negligibly small in almost all cases, so it is recommended to use
keys, as this makes for much more human-readable code.

Mathematical operations on the abundances will pass through and be
applied to the underlying numpy array:

\begin{Verbatim}[commandchars=\\\{\}]
\PYG{c+c1}{\PYGZsh{} Double the abundance of every species}
\PYG{n}{abd} \PYG{o}{=} \PYG{l+m+mi}{2}\PYG{o}{*}\PYG{n}{abd}

\PYG{c+c1}{\PYGZsh{} Add the abundances of two different abundanceDict\PYGZsq{}s, abd1 and abd2}
\PYG{n}{abd3} \PYG{o}{=} \PYG{n}{abd1} \PYG{o}{+} \PYG{n}{abd2}
\end{Verbatim}

Finally, \code{abundanceDict} instances differ from regular dict objects
in that, once they are created, their species lists are immutable;
species abundances can be changed, but species cannot be added or
removed.

For full details on \code{abundanceDict}, see {\hyperref[fulldoc:sec\string-fulldoc]{\crossref{\DUrole{std,std-ref}{Full Documentation of All DESPOTIC Classes and Functions}}}}.


\subsection{\texttt{reactions}}
\label{chemistry:reactions}\label{chemistry:sssec-reactions}
The \code{reactions} module provides a generic way to describe chemical
reactions and compute their rates. It contains one basic class and
several specialized derived classes to compute particualr types of
reactions. The goal of all the classes in \code{reactions} is to allow
users to describe the reactions in a chemical network in
human-readable forms, but then compute their rates using efficient
numpy operations at high speed.


\subsubsection{\texttt{reaction\_matrix}}
\label{chemistry:reaction-matrix}\label{chemistry:ssssec-reaction-matrix}
The most general class in \code{reactions} is called
\code{reaction\_matrix}. To instantiate it, one provides a list of species
and a list of reactions between them:

\begin{Verbatim}[commandchars=\\\{\}]
\PYG{k+kn}{from} \PYG{n+nn}{despotic}\PYG{n+nn}{.}\PYG{n+nn}{chemistry}\PYG{n+nn}{.}\PYG{n+nn}{reactions} \PYG{k}{import} \PYG{n}{reaction\PYGZus{}matrix}
\PYG{n}{rm} \PYG{o}{=} \PYG{n}{reaction\PYGZus{}matrix}\PYG{p}{(}\PYG{n}{specList}\PYG{p}{,} \PYG{n}{reactions}\PYG{p}{)}
\end{Verbatim}

Here \code{specList} is a list of strings giving the names of the
chemical species, while \code{reactions} is a list of dict's, one dict
per reaction, describing each reaction. Each dict in \code{reactions}
contains two keys: \code{reactions{[}"spec"{]}} gives the list of species
involved in the reaction, and \code{reactions{[}"stoich"{]}} gives the
corresponding stoichiometric factors, with reactants on the left hand
side having negative factors (indicating that they are destroyed by
the reaction) and those on the right having positive factors
(indicating that they are created). Thus for example to include the
reactions
\begin{equation*}
\begin{split}\mathrm{C} + \mathrm{O} & \rightarrow \mathrm{CO} \\
\mathrm{H} + \mathrm{H} & \rightarrow \mathrm{H}_2\end{split}
\end{equation*}
in a network, one could define the reactions as:

\begin{Verbatim}[commandchars=\\\{\}]
\PYG{n}{reactions} \PYGZbs{}
   \PYG{o}{=} \PYG{p}{[} \PYG{p}{\PYGZob{}} \PYG{l+s+s2}{\PYGZdq{}}\PYG{l+s+s2}{spec}\PYG{l+s+s2}{\PYGZdq{}} \PYG{p}{:} \PYG{p}{[}\PYG{l+s+s2}{\PYGZdq{}}\PYG{l+s+s2}{C}\PYG{l+s+s2}{\PYGZdq{}}\PYG{p}{,} \PYG{l+s+s2}{\PYGZdq{}}\PYG{l+s+s2}{O}\PYG{l+s+s2}{\PYGZdq{}}\PYG{p}{,} \PYG{l+s+s2}{\PYGZdq{}}\PYG{l+s+s2}{CO}\PYG{l+s+s2}{\PYGZdq{}}\PYG{p}{]}\PYG{p}{,} \PYG{l+s+s2}{\PYGZdq{}}\PYG{l+s+s2}{stoich}\PYG{l+s+s2}{\PYGZdq{}} \PYG{p}{:} \PYG{p}{[}\PYG{o}{\PYGZhy{}}\PYG{l+m+mi}{1}\PYG{p}{,} \PYG{o}{\PYGZhy{}}\PYG{l+m+mi}{1}\PYG{p}{,} \PYG{l+m+mi}{1}\PYG{p}{]} \PYG{p}{\PYGZcb{}}\PYG{p}{,}
       \PYG{p}{\PYGZob{}} \PYG{l+s+s2}{\PYGZdq{}}\PYG{l+s+s2}{spec}\PYG{l+s+s2}{\PYGZdq{}} \PYG{p}{:} \PYG{p}{[}\PYG{l+s+s2}{\PYGZdq{}}\PYG{l+s+s2}{H}\PYG{l+s+s2}{\PYGZdq{}}\PYG{p}{,} \PYG{l+s+s2}{\PYGZdq{}}\PYG{l+s+s2}{H2}\PYG{l+s+s2}{\PYGZdq{}}\PYG{p}{]}\PYG{p}{,} \PYG{l+s+s2}{\PYGZdq{}}\PYG{l+s+s2}{stoich}\PYG{l+s+s2}{\PYGZdq{}} \PYG{p}{:} \PYG{p}{[}\PYG{o}{\PYGZhy{}}\PYG{l+m+mi}{2}\PYG{p}{,} \PYG{l+m+mi}{1}\PYG{p}{]} \PYG{p}{\PYGZcb{}} \PYG{p}{]}
\end{Verbatim}

Once a \code{reaction\_matrix} has been defined, one can compute the rates
of change of all species using the method \code{reaction\_matrix.dxdt}:

\begin{Verbatim}[commandchars=\\\{\}]
\PYG{n}{dxdt} \PYG{o}{=} \PYG{n}{rm}\PYG{o}{.}\PYG{n}{dxdt}\PYG{p}{(}\PYG{n}{x}\PYG{p}{,} \PYG{n}{n}\PYG{p}{,} \PYG{n}{ratecoef}\PYG{p}{)}
\end{Verbatim}

Here \code{x} is a numpy array giving the instantaneous abundances of all
species, \code{n} is the number density of H nuclei, and \code{ratecoeff} is
a numpy array giving the rate coefficient for all reactions. The
quantity returned is the instantaneous rate of change of all
abundances \code{x}. The density-dependence of the reaction rates implied
by the stoichiometric factors in the reaction list is computed
automatically.


\subsubsection{\texttt{cr\_reactions}}
\label{chemistry:cr-reactions}
The \code{cr\_reactions} class is a specialized class derived from
{\hyperref[chemistry:ssssec\string-reaction\string-matrix]{\crossref{\DUrole{std,std-ref}{reaction\_matrix}}}} that is intended to handle cosmic
ray-induced reactions -- those where the rate is proportional to
the cosmic ray ionization rate.

Instantiation of a \code{cr\_reactions} object takes the same two
arguments as {\hyperref[chemistry:ssssec\string-reaction\string-matrix]{\crossref{\DUrole{std,std-ref}{reaction\_matrix}}}}, but \code{reactions}, the
list of reactions to be passed, is altered in that each reaction takes
an additional key, \code{rate}, that gives the proportionality between
the reaction rate and the cosmic ray ionization rate. Thus for example
if we wished to include the reactions
\begin{equation*}
\begin{split}\mathrm{CR} + \mathrm{H} & \rightarrow \mathrm{H}^+ +
\mathrm{e}^- \\
\mathrm{CR} + \mathrm{H}_2 & \rightarrow \mathrm{H}_2^+ +
\mathrm{e}^-\end{split}
\end{equation*}
in a network, with the former occurring at a rate per particle equal to
the rate of primary cosmic ray ionizations, and the other at a rate
per particle that is twice the rate of primary ionizations, we would
define:

\begin{Verbatim}[commandchars=\\\{\}]
\PYG{k+kn}{from} \PYG{n+nn}{despotic}\PYG{n+nn}{.}\PYG{n+nn}{chemistry}\PYG{n+nn}{.}\PYG{n+nn}{reactions} \PYG{k}{import} \PYG{n}{cr\PYGZus{}reactions}
\PYG{n}{specList} \PYG{o}{=} \PYG{p}{[}\PYG{l+s+s2}{\PYGZdq{}}\PYG{l+s+s2}{H}\PYG{l+s+s2}{\PYGZdq{}}\PYG{p}{,} \PYG{l+s+s2}{\PYGZdq{}}\PYG{l+s+s2}{H2}\PYG{l+s+s2}{\PYGZdq{}}\PYG{p}{,} \PYG{l+s+s2}{\PYGZdq{}}\PYG{l+s+s2}{H+}\PYG{l+s+s2}{\PYGZdq{}}\PYG{p}{,} \PYG{l+s+s2}{\PYGZdq{}}\PYG{l+s+s2}{H2+}\PYG{l+s+s2}{\PYGZdq{}}\PYG{p}{,} \PYG{l+s+s2}{\PYGZdq{}}\PYG{l+s+s2}{e\PYGZhy{}}\PYG{l+s+s2}{\PYGZdq{}}\PYG{p}{]}
\PYG{n}{reactions} \PYGZbs{}
   \PYG{o}{=} \PYG{p}{[} \PYG{p}{\PYGZob{}} \PYG{l+s+s2}{\PYGZdq{}}\PYG{l+s+s2}{spec}\PYG{l+s+s2}{\PYGZdq{}} \PYG{p}{:} \PYG{p}{[}\PYG{l+s+s2}{\PYGZdq{}}\PYG{l+s+s2}{H}\PYG{l+s+s2}{\PYGZdq{}}\PYG{p}{,} \PYG{l+s+s2}{\PYGZdq{}}\PYG{l+s+s2}{H+}\PYG{l+s+s2}{\PYGZdq{}}\PYG{p}{,} \PYG{l+s+s2}{\PYGZdq{}}\PYG{l+s+s2}{e\PYGZhy{}}\PYG{l+s+s2}{\PYGZdq{}}\PYG{p}{]}\PYG{p}{,} \PYG{l+s+s2}{\PYGZdq{}}\PYG{l+s+s2}{stoich}\PYG{l+s+s2}{\PYGZdq{}} \PYG{p}{:} \PYG{p}{[}\PYG{o}{\PYGZhy{}}\PYG{l+m+mi}{1}\PYG{p}{,} \PYG{l+m+mi}{1}\PYG{p}{,} \PYG{l+m+mi}{1}\PYG{p}{]}\PYG{p}{,}
         \PYG{l+s+s2}{\PYGZdq{}}\PYG{l+s+s2}{rate}\PYG{l+s+s2}{\PYGZdq{}} \PYG{p}{:} \PYG{l+m+mf}{1.0} \PYG{p}{\PYGZcb{}}\PYG{p}{,}
       \PYG{p}{\PYGZob{}} \PYG{l+s+s2}{\PYGZdq{}}\PYG{l+s+s2}{spec}\PYG{l+s+s2}{\PYGZdq{}} \PYG{p}{:} \PYG{p}{[}\PYG{l+s+s2}{\PYGZdq{}}\PYG{l+s+s2}{H2}\PYG{l+s+s2}{\PYGZdq{}}\PYG{p}{,} \PYG{l+s+s2}{\PYGZdq{}}\PYG{l+s+s2}{H2+}\PYG{l+s+s2}{\PYGZdq{}}\PYG{p}{,} \PYG{l+s+s2}{\PYGZdq{}}\PYG{l+s+s2}{e\PYGZhy{}}\PYG{l+s+s2}{\PYGZdq{}}\PYG{p}{]}\PYG{p}{,} \PYG{l+s+s2}{\PYGZdq{}}\PYG{l+s+s2}{stoich}\PYG{l+s+s2}{\PYGZdq{}} \PYG{p}{:} \PYG{p}{[}\PYG{o}{\PYGZhy{}}\PYG{l+m+mi}{1}\PYG{p}{,} \PYG{l+m+mi}{1}\PYG{p}{,} \PYG{l+m+mi}{1}\PYG{p}{]}\PYG{p}{,}
         \PYG{l+s+s2}{\PYGZdq{}}\PYG{l+s+s2}{rate}\PYG{l+s+s2}{\PYGZdq{}} \PYG{p}{:} \PYG{l+m+mf}{2.0} \PYG{p}{\PYGZcb{}} \PYG{p}{]}
\PYG{n}{cr} \PYG{o}{=} \PYG{n}{cr\PYGZus{}reactions}\PYG{p}{(}\PYG{n}{specList}\PYG{p}{,} \PYG{n}{reactions}\PYG{p}{)}
\end{Verbatim}

Once a \code{cr\_reactions} object is instantiated, it can be used to
compute the rates of change all species using the
\code{cr\_reactions.dxdt} routine:

\begin{Verbatim}[commandchars=\\\{\}]
\PYG{n}{dxdt} \PYG{o}{=} \PYG{n}{cr}\PYG{o}{.}\PYG{n}{dxdt}\PYG{p}{(}\PYG{n}{x}\PYG{p}{,} \PYG{n}{n}\PYG{p}{,} \PYG{n}{ionrate}\PYG{p}{)}
\end{Verbatim}

Here \code{x} is a numpy array giving the current abundances, \code{n} is
the number density of H nuclei, and \code{ionrate} is the cosmic ray
primary ionization rate.

Note that, in most cases, the variable \code{n} will not be used, because
it is needed only if there is more than one reactant on the left hand
side of a reaction. It is provided to enable the case where a cosmic
ray ionization is followed immediately by another reaction, and the
network combines the two steps for speed. For example, the
{\hyperref[chemistry:sssec\string-nl99]{\crossref{\DUrole{std,std-ref}{NL99}}}} network combines the two reactions
\begin{equation*}
\begin{split}\mathrm{CR} + \mathrm{H}_2 & \rightarrow \mathrm{H}_2^+ +
\mathrm{e}^- \\
\mathrm{H}_2^+ + \mathrm{H}_2 & \rightarrow \mathrm{H}_3^+ +
\mathrm{H} + \mathrm{e}^-\end{split}
\end{equation*}
into the single super-reaction
\begin{equation*}
\begin{split}\mathrm{CR} + 2\mathrm{H}_2 \rightarrow \mathrm{H}_3^+ +
\mathrm{H} + \mathrm{e}^-\end{split}
\end{equation*}
and the rate of this super-reaction does depend on density.


\subsubsection{\texttt{photoreactions}}
\label{chemistry:photoreactions}
The \code{photoreactions} class is a specialized class derived from
{\hyperref[chemistry:ssssec\string-reaction\string-matrix]{\crossref{\DUrole{std,std-ref}{reaction\_matrix}}}} that is intended to handle reactions
that are driven by FUV photons, and thus have rates proportional to
the interstellar radiation field (ISRF) strength, modified by dust and
gas shielding, and possibly also by line shielding.

Instantiation of \code{photoreactions} takes the same two arguments as
{\hyperref[chemistry:ssssec\string-reaction\string-matrix]{\crossref{\DUrole{std,std-ref}{reaction\_matrix}}}}: a list of species, and a list of
reactions, each element of which is a dict giving the reactants and
stoichiometric factors. The dict also contains three additional keys
that are specific to photoreactions: \code{rate} gives the reaction rate
per reactant per second in an ISRF equal to the unshielded Solar
neighborhood value (formally, with strength characterized by
\(\chi = 1\) -- see {\hyperref[cloudfiles:tab\string-cloudfiles]{\crossref{\DUrole{std,std-ref}{Cloud file keys and their meanings}}}}). The key \code{av\_fac}
describes the optical depth per unit visual extinction provided by
dust for the photons driving the reaction. That is, reaction rates
will be reduced by a factor \(\exp(-\mathrm{av\_fac}\times
A_V)\). Finally, the key \code{shield\_fac}, which is optional, can be set
equal to a callable that describes the rate by which the reaction rate
is reduced due to line shielding. This function can take any number of
argument -- see below. Thus the final reaction rate per reactant will
be equal to
\begin{equation*}
\begin{split}\mathrm{reaction\, rate} = \chi \times \mathrm{rate} \times
\mathrm{shield\_fac} \times \exp(-\mathrm{av\_fac}\times A_V)\end{split}
\end{equation*}
As an example, suppose we wished to include the reactions
\begin{equation*}
\begin{split}\gamma + \mathrm{C} & \rightarrow \mathrm{C}^+ + \mathrm{e}^- \\
\gamma + \mathrm{CO} & \rightarrow \mathrm{C} + \mathrm{O} \\\end{split}
\end{equation*}
The first reaction occurs at a rate per C atom \(5.1\times
10^{-10}\,\mathrm{s}^{-1}\) in unshielded space, and the optical depth
to the photons producing the reaction is \(3.0\times A_V\). There
is no significant self-shielding. The second reaction occurs at a rate
per CO molecule \(1.7\times 10^{-10}\,\mathrm{s}^{-1}\) in
unshielded space, with a dust optical depth equal to \(1.7\times
A_V\), and with an additional function describing line shielding called
\code{fShield\_CO}. We could create a \code{photoeactions} object containing
this reaction by doing:

\begin{Verbatim}[commandchars=\\\{\}]
\PYG{k+kn}{from} \PYG{n+nn}{despotic}\PYG{n+nn}{.}\PYG{n+nn}{chemistry}\PYG{n+nn}{.}\PYG{n+nn}{reactions} \PYG{k}{import} \PYG{n}{photoreactions}
\PYG{n}{specList} \PYG{o}{=} \PYG{p}{[}\PYG{l+s+s2}{\PYGZdq{}}\PYG{l+s+s2}{C}\PYG{l+s+s2}{\PYGZdq{}}\PYG{p}{,} \PYG{l+s+s2}{\PYGZdq{}}\PYG{l+s+s2}{C+}\PYG{l+s+s2}{\PYGZdq{}}\PYG{p}{,} \PYG{l+s+s2}{\PYGZdq{}}\PYG{l+s+s2}{e\PYGZhy{}}\PYG{l+s+s2}{\PYGZdq{}}\PYG{p}{,} \PYG{l+s+s2}{\PYGZdq{}}\PYG{l+s+s2}{O}\PYG{l+s+s2}{\PYGZdq{}}\PYG{p}{,} \PYG{l+s+s2}{\PYGZdq{}}\PYG{l+s+s2}{CO}\PYG{l+s+s2}{\PYGZdq{}}\PYG{p}{]}
\PYG{n}{reactions} \PYGZbs{}
   \PYG{o}{=} \PYG{p}{[} \PYG{p}{\PYGZob{}} \PYG{l+s+s2}{\PYGZdq{}}\PYG{l+s+s2}{spec}\PYG{l+s+s2}{\PYGZdq{}} \PYG{p}{:} \PYG{p}{[}\PYG{l+s+s2}{\PYGZdq{}}\PYG{l+s+s2}{C}\PYG{l+s+s2}{\PYGZdq{}}\PYG{p}{,} \PYG{l+s+s2}{\PYGZdq{}}\PYG{l+s+s2}{C+}\PYG{l+s+s2}{\PYGZdq{}}\PYG{p}{,} \PYG{l+s+s2}{\PYGZdq{}}\PYG{l+s+s2}{e\PYGZhy{}}\PYG{l+s+s2}{\PYGZdq{}}\PYG{p}{]}\PYG{p}{,} \PYG{l+s+s2}{\PYGZdq{}}\PYG{l+s+s2}{stoich}\PYG{l+s+s2}{\PYGZdq{}} \PYG{p}{:} \PYG{p}{[}\PYG{o}{\PYGZhy{}}\PYG{l+m+mi}{1}\PYG{p}{,} \PYG{l+m+mi}{1}\PYG{p}{,} \PYG{l+m+mi}{1}\PYG{p}{]}\PYG{p}{,}
         \PYG{l+s+s2}{\PYGZdq{}}\PYG{l+s+s2}{rate}\PYG{l+s+s2}{\PYGZdq{}} \PYG{p}{:} \PYG{l+m+mf}{5.1e\PYGZhy{}10}\PYG{p}{,} \PYG{l+s+s2}{\PYGZdq{}}\PYG{l+s+s2}{av\PYGZus{}fac}\PYG{l+s+s2}{\PYGZdq{}} \PYG{p}{:} \PYG{l+m+mf}{3.0} \PYG{p}{\PYGZcb{}}\PYG{p}{,}
       \PYG{p}{\PYGZob{}} \PYG{l+s+s2}{\PYGZdq{}}\PYG{l+s+s2}{spec}\PYG{l+s+s2}{\PYGZdq{}} \PYG{p}{:} \PYG{p}{[}\PYG{l+s+s2}{\PYGZdq{}}\PYG{l+s+s2}{CO}\PYG{l+s+s2}{\PYGZdq{}}\PYG{p}{,} \PYG{l+s+s2}{\PYGZdq{}}\PYG{l+s+s2}{C}\PYG{l+s+s2}{\PYGZdq{}}\PYG{p}{,} \PYG{l+s+s2}{\PYGZdq{}}\PYG{l+s+s2}{O}\PYG{l+s+s2}{\PYGZdq{}}\PYG{p}{]}\PYG{p}{,} \PYG{l+s+s2}{\PYGZdq{}}\PYG{l+s+s2}{stoich}\PYG{l+s+s2}{\PYGZdq{}} \PYG{p}{:} \PYG{p}{[}\PYG{o}{\PYGZhy{}}\PYG{l+m+mi}{1}\PYG{p}{,} \PYG{l+m+mi}{1}\PYG{p}{,} \PYG{l+m+mi}{1}\PYG{p}{]}\PYG{p}{,}
         \PYG{l+s+s2}{\PYGZdq{}}\PYG{l+s+s2}{rate}\PYG{l+s+s2}{\PYGZdq{}} \PYG{p}{:} \PYG{l+m+mf}{1.7e\PYGZhy{}10}\PYG{p}{,} \PYG{l+s+s2}{\PYGZdq{}}\PYG{l+s+s2}{av\PYGZus{}fac}\PYG{l+s+s2}{\PYGZdq{}} \PYG{p}{:} \PYG{l+m+mf}{1.7}\PYG{p}{,}
         \PYG{l+s+s2}{\PYGZdq{}}\PYG{l+s+s2}{shield\PYGZus{}fac}\PYG{l+s+s2}{\PYGZdq{}} \PYG{p}{:} \PYG{n}{fShield\PYGZus{}CO} \PYG{p}{\PYGZcb{}} \PYG{p}{]}
\PYG{n}{phr} \PYG{o}{=} \PYG{n}{photoreactions}\PYG{p}{(}\PYG{n}{specList}\PYG{p}{,} \PYG{n}{reactions}\PYG{p}{)}
\end{Verbatim}

Once a photoreactions object exists, the rates of change of all
species due to the included photoreactions can be computed using the
\code{photoreactions.dxdt} routine. Usage is as follows:

\begin{Verbatim}[commandchars=\\\{\}]
\PYG{n}{dxdt} \PYG{o}{=} \PYG{n}{phr}\PYG{o}{.}\PYG{n}{dxdt}\PYG{p}{(}\PYG{n}{x}\PYG{p}{,} \PYG{n}{n}\PYG{p}{,} \PYG{n}{chi}\PYG{p}{,} \PYG{n}{AV}\PYG{p}{,}
                \PYG{n}{shield\PYGZus{}args}\PYG{o}{=}\PYG{p}{[}\PYG{p}{[}\PYG{n}{f1\PYGZus{}arg1}\PYG{p}{,} \PYG{n}{f1\PYGZus{}arg2}\PYG{p}{,} \PYG{o}{.}\PYG{o}{.}\PYG{o}{.}\PYG{p}{]}\PYG{p}{,}
                             \PYG{p}{[}\PYG{n}{f2\PYGZus{}arg1}\PYG{p}{,} \PYG{n}{f2\PYGZus{}arg2}\PYG{p}{,} \PYG{o}{.}\PYG{o}{.}\PYG{o}{.}\PYG{p}{]}\PYG{p}{,}
                             \PYG{o}{.}\PYG{o}{.}\PYG{o}{.} \PYG{p}{]}\PYG{p}{,}
                \PYG{n}{shield\PYGZus{}kw\PYGZus{}args} \PYG{o}{=} \PYG{p}{[} \PYG{p}{\PYGZob{}} \PYG{l+s+s2}{\PYGZdq{}}\PYG{l+s+s2}{f1\PYGZus{}kw1}\PYG{l+s+s2}{\PYGZdq{}} \PYG{p}{:} \PYG{n}{val1}\PYG{p}{,}
                                     \PYG{l+s+s2}{\PYGZdq{}}\PYG{l+s+s2}{f1\PYGZus{}kw2}\PYG{l+s+s2}{\PYGZdq{}} \PYG{p}{:} \PYG{n}{val2}\PYG{p}{,} \PYG{o}{.}\PYG{o}{.}\PYG{o}{.} \PYG{p}{\PYGZcb{}}\PYG{p}{,}
                                   \PYG{p}{\PYGZob{}} \PYG{l+s+s2}{\PYGZdq{}}\PYG{l+s+s2}{f2\PYGZus{}kw1}\PYG{l+s+s2}{\PYGZdq{}} \PYG{p}{:} \PYG{n}{val1}\PYG{p}{,}
                                     \PYG{l+s+s2}{\PYGZdq{}}\PYG{l+s+s2}{f2\PYGZus{}kw2}\PYG{l+s+s2}{\PYGZdq{}} \PYG{p}{:} \PYG{n}{val2}\PYG{p}{,} \PYG{o}{.}\PYG{o}{.}\PYG{o}{.} \PYG{p}{\PYGZcb{}}\PYG{p}{,}
                                   \PYG{o}{.}\PYG{o}{.}\PYG{o}{.} \PYG{p}{]}\PYG{p}{)}
\end{Verbatim}

Here \code{x} is a numpy array giving the current abundances, \code{n} is
the number density of H nuclei, \code{chi} is the unshielded ISRF
strength, and \code{AV} is the dust visual extinction. The optional
arguments \code{shield\_args} and \code{shield\_kw\_args} are used to pass
positional arguments and keyword arguments, respectively, to the
shielding functions. Each one is a list. The first entry in the list
for \code{shield\_args} is a list containing the positional arguments to
be passed to the first shielding function provided in the
\code{reactions} dict passed to the constructor. The second entry in
\code{shield\_args} is a list containing positional arguments to be passed
to the second shielding function, and so forth. Similarly, the
first entry in the \code{shield\_kw\_args} list is a dict containing the
keywords and their values to be passed to the first shielding
function, etc. The length of the \code{shield\_args} and
\code{shield\_kw\_args} must be equal to the number of shielding functions
provided in \code{reactions}, but elements of the lists can be set to
\code{None} if a particular shielding function does not take positional
or keyword arguments. Moreover, both \code{shield\_args} and
\code{shield\_kw\_args} are optional. If they are omitted, then all
shielding functions are invoked with no positional or keyword
arguments, respectively.

Thus in the above example, suppose that the function \code{fShield\_CO}
returns the factor by which the reaction rate is reduced by CO line
shielding. It takes two arguments: \code{NH2} and \code{NCO}, giving the
column densities of \(\mathrm{H}_2\) and \(\mathrm{CO}\),
respectively. It also accepts a keyword argument \code{order} that
specifies the order of interpolation to be used in computing the
shielding function from a table. In this case one could compute the
rates of change of the abundances via:

\begin{Verbatim}[commandchars=\\\{\}]
\PYG{n}{dxdt} \PYG{o}{=} \PYG{n}{phr}\PYG{o}{.}\PYG{n}{dxdt}\PYG{p}{(}\PYG{n}{x}\PYG{p}{,} \PYG{n}{n}\PYG{p}{,} \PYG{n}{chi}\PYG{p}{,} \PYG{n}{AV}\PYG{p}{,}
                \PYG{n}{shield\PYGZus{}args} \PYG{o}{=} \PYG{p}{[}\PYG{p}{[}\PYG{n}{NH2}\PYG{p}{,} \PYG{n}{NCO}\PYG{p}{]}\PYG{p}{]}\PYG{p}{,}
                \PYG{n}{shield\PYGZus{}kw\PYGZus{}args} \PYG{o}{=} \PYG{p}{[} \PYG{p}{\PYGZob{}}\PYG{l+s+s2}{\PYGZdq{}}\PYG{l+s+s2}{order}\PYG{l+s+s2}{\PYGZdq{}} \PYG{p}{:} \PYG{l+m+mi}{2}\PYG{p}{\PYGZcb{}} \PYG{p}{]}\PYG{p}{)}
\end{Verbatim}

This would call \code{fShield\_CO} as:

\begin{Verbatim}[commandchars=\\\{\}]
\PYG{n}{fShield\PYGZus{}CO}\PYG{p}{(}\PYG{n}{NH2}\PYG{p}{,} \PYG{n}{NCO}\PYG{p}{,} \PYG{n}{order}\PYG{o}{=}\PYG{l+m+mi}{2}\PYG{p}{)}
\end{Verbatim}


\chapter{Winds}
\label{winds:winds}\label{winds:sec-winds}\label{winds::doc}
The wind modelling capability in DESPOTIC is located in the
\code{despotic.winds} module. The unofficial name for this portion of the
code is ROCKETSTAR -- as named by the primary author's 5-year old.

The physical model used in the \code{despotic.winds} module is described
in Krumholz et al. (2017), and users are encouraged to read that paper
to understand the physical basis of the calculations. Example scripts
using this code can be found in the repository
\url{https://bitbucket.org/krumholz/despotic\_winds/} associated with
this paper.


\section{Compiling}
\label{winds:compiling}
The \code{despotic.winds} class relies on a copmlementary C++ library for
speed. This must be compiled separately, though the procedure should
be automatic for users with standard tools and libraries
installed. See {\hyperref[installation:ssec\string-winds\string-installation]{\crossref{\DUrole{std,std-ref}{Installing the despotic.winds Module}}}} for details.


\section{The \texttt{pwind} Class}
\label{winds:the-pwind-class}\label{winds:ssec-wind-pwind}
The front end to the \code{despotic.winds} is the \code{pwind} class. This
class requires that the user specify the generalized Eddington ratio
\(\Gamma\) and the Mach number \(\mathcal{M}\) for the
wind. By default this will create a spherical, ideal wind, which is
fully defined by these two parameters. In addition, the the user can
specify a number of other proprties:
\begin{itemize}
\item {} 
The rate of cloud expansion, specified by the keyword
\code{expansion}. Valid values are \code{area}, \code{intermediate}, and
\code{solid angle}.

\item {} 
The gravitational potential, specified by the keyword
\code{potential}. Valid values are \code{point} and \code{isothermal}.

\item {} 
The geometry of the wind, specified by the keyword
\code{geometry}. Valid values are \code{sphere}, \code{cone}, and
\code{cone sheath}. For \code{cone} the user must also specify the cone
tilt (via the keyword \code{phi}) and the cone opening angle (via the
keyword \code{theta}). For \code{cone sheath} the user must also specify
the inner opening angle (via the keyword \code{theta\_in}).

\item {} 
The driving mechanism, specified by the keyword \code{driver}. Valid
values are \code{ideal}, \code{radiation}, and \code{hot}. For \code{radiation},
the user must specify the optical depth at the mean surface density
(via the keyword \code{tau0}). For \code{hot}, the user must specify the
dimensionless hot gas velocity (via the keyword \code{uh}).

\end{itemize}

The are a range of other keywords the affect the behavior of \code{pwind}
objects. See {\hyperref[fulldoc:sssec\string-full\string-pwind]{\crossref{\DUrole{std,std-ref}{pwind}}}} for a full listing.

An example is:

\begin{Verbatim}[commandchars=\\\{\}]
\PYG{k+kn}{import} \PYG{n+nn}{numpy} \PYG{k}{as} \PYG{n+nn}{np}
\PYG{k+kn}{from} \PYG{n+nn}{despotic}\PYG{n+nn}{.}\PYG{n+nn}{winds} \PYG{k}{import} \PYG{n}{pwind}

\PYG{n}{Gamma} \PYG{o}{=} \PYG{l+m+mf}{0.2}
\PYG{n}{Mach} \PYG{o}{=} \PYG{l+m+mf}{50.}
\PYG{n}{tau0} \PYG{o}{=} \PYG{l+m+mf}{100.}
\PYG{n}{phi} \PYG{o}{=} \PYG{n}{np}\PYG{o}{.}\PYG{n}{pi}\PYG{o}{/}\PYG{l+m+mf}{4.0}
\PYG{n}{theta} \PYG{o}{=} \PYG{n}{np}\PYG{o}{.}\PYG{n}{pi}\PYG{o}{/}\PYG{l+m+mf}{2.0}

\PYG{n}{pw} \PYG{o}{=} \PYG{n}{pwind}\PYG{p}{(}\PYG{n}{Gamma}\PYG{p}{,} \PYG{n}{Mach}\PYG{p}{,} \PYG{n}{driver}\PYG{o}{=}\PYG{l+s+s1}{\PYGZsq{}}\PYG{l+s+s1}{radiation}\PYG{l+s+s1}{\PYGZsq{}}\PYG{p}{,} \PYG{n}{potential}\PYG{o}{=}\PYG{l+s+s1}{\PYGZsq{}}\PYG{l+s+s1}{isothermal}\PYG{l+s+s1}{\PYGZsq{}}\PYG{p}{,}
           \PYG{n}{expansion}\PYG{o}{=}\PYG{l+s+s1}{\PYGZsq{}}\PYG{l+s+s1}{solid angle}\PYG{l+s+s1}{\PYGZsq{}}\PYG{p}{,} \PYG{n}{geometry}\PYG{o}{=}\PYG{l+s+s1}{\PYGZsq{}}\PYG{l+s+s1}{cone}\PYG{l+s+s1}{\PYGZsq{}}\PYG{p}{,}
           \PYG{n}{tau0}\PYG{o}{=}\PYG{n}{tau0}\PYG{p}{,} \PYG{n}{theta}\PYG{o}{=}\PYG{n}{theta}\PYG{p}{,} \PYG{n}{phi}\PYG{o}{=}\PYG{n}{phi}\PYG{p}{)}
\end{Verbatim}

This creates a \code{pwind} object that represents a radiatively-driven
wind in an isothermal potential, with clouds maintaining constant
solid angle. The wind is confined to a cone that is tipped by
\(45^\circ\) relative to the vertical, with a \(90^\circ\)
opening angle. The wind is characterized by a generalized Eddington
ratio \(\Gamma = 0.2\) and a Mach number \(\mathcal{M} =
50\), and the optical depth at the mean surface density of the launch
region is \(\tau_0 = 100\).


\section{Caculations Using \texttt{pwind}}
\label{winds:caculations-using-pwind}
The \code{pwind} class defines a series of methods that can be used to
compute the observable properties of the specified wind. There are
four basic types of observables that can be computed:
\begin{itemize}
\item {} 
\code{pwind.tau}: this method computes absorption optical depths. The
user must specify the dimensionless velocity or velocities \code{u} at
which to compute the absorption, as well as the dimensionless
transition strength \(t_X/t_w\) for the wind. This can be
specified either directly, via the keyword \code{tXtw}, or computed
from an input oscillator strength \code{Omega}, wavelength
\code{wl}, abundance \code{abd}, and wind mass removal timescale
\code{tw}. The keyword \code{correlated} specifies whether the wind should
be treated as correlated or uncorrelated. The keyword \code{u\_trans}
specifies that the transition in question is a multiplet, with
the individual transitions occurring at dimensionless velocities
given by \code{u\_trans}.

\item {} 
\code{pwind.Xi}: this method returns the optically thin emission line
shape function \(\Xi\), as defined in the Krumholz et al. (2017)
paper. The user must specify the dimensionless velocity or
velocities \code{u}.

\item {} 
\code{pwind.temp\_LTE}: this returns the brightness or antenna
temperature for a species in LTE. The user must provide the
dimensionless velocity or velocities \code{u} and the wind kinetic
temperature \code{T}. In addition, the user must provide the dimensionless
transition strength \(t_X/t_w\) for the wind. This can be
specified either directly, via the keyword \code{tXtw}, or computed
on the fly in one of two ways. First, the user can specify
\code{Omega}, \code{wl}, \code{abd} and \code{tw}, exactly as for
\code{pwind.tau}. Second, the user can provide a DESPOTIC \code{emitter}
object (see {\hyperref[fulldoc:sssec\string-full\string-emitter]{\crossref{\DUrole{std,std-ref}{emitter}}}}) and a wind removal timescale
\code{tw}. Finally, the keyword \code{correlated} specifies whether the
wind should be treated as correlated or uncorrelated.

\item {} 
\code{pwind.intTA\_LTE}: this computes the velocity-integrated antenna
temperature for an emitting species in LTE. The user must provide
the velocity scale \code{v0} for the wind launching region and the wind
kinetic temperature \code{T}. All other parameters are as for
\code{pwind.temp\_LTE}.

\end{itemize}

All of these routines accept the keywords \code{varpi} and \code{varpi\_t}
which specify the dimensionless axial and transverse position of the
line of sight (\(\varpi_a\) and \(\varpi_t\) in the
terminology of Krumholz et al. 2017). In addition, all routines except
\code{pwind.Xi} accept the keywords \code{fj} and \code{boltzfac}, which
specify the fractional level population for the lower level and the
Boltzmann factor between the two levels of the transition,
respectively.

The are a range of other keywords that affect the behavior of these
computation routines. See {\hyperref[fulldoc:sssec\string-full\string-pwind]{\crossref{\DUrole{std,std-ref}{pwind}}}} for a full
listing.


\chapter{Full Documentation of All DESPOTIC Classes and Functions}
\label{fulldoc:sec-fulldoc}\label{fulldoc::doc}\label{fulldoc:full-documentation-of-all-despotic-classes-and-functions}

\section{despotic classes}
\label{fulldoc:despotic-classes}

\subsection{\texttt{cloud}}
\label{fulldoc:sssec-full-cloud}\label{fulldoc:cloud}\index{cloud (class in despotic)}

\begin{fulllineitems}
\phantomsection\label{fulldoc:despotic.cloud}\pysiglinewithargsret{\strong{class }\code{despotic.}\bfcode{cloud}}{\emph{fileName=None}, \emph{verbose=False}}{}
A class describing the properties of an interstellar cloud, and
providing methods to perform calculations using those properties.
\begin{description}
\item[{Parameters}] \leavevmode\begin{description}
\item[{fileName}] \leavevmode{[}string{]}
name of file from which to read cloud description

\item[{verbose}] \leavevmode{[}Boolean{]}
print out information about the cloud as we read it

\end{description}

\item[{Class attributes}] \leavevmode\begin{description}
\item[{nH}] \leavevmode{[}float{]}
number density of H nuclei, in cm\textasciicircum{}-3

\item[{colDen}] \leavevmode{[}float{]}
center-to-edge column density of H nuclei, in cm\textasciicircum{}-2

\item[{sigmaNT}] \leavevmode{[}float{]}
non-thermal velocity dispersion, in cm s\textasciicircum{}-1

\item[{dVdr}] \leavevmode{[}float{]}
radial velocity gradient, in s\textasciicircum{}-1 (or cm s\textasciicircum{}-1 cm\textasciicircum{}-1)

\item[{Tg}] \leavevmode{[}float{]}
gas kinetic temperature, in K

\item[{Td}] \leavevmode{[}float{]}
dust temperature, in K

\item[{comp}] \leavevmode{[}class composition{]}
a class that stores information about the chemical
composition of the cloud

\item[{dust}] \leavevmode{[}class dustProp{]}
a class that stores information about the properties of the
dust in a cloud

\item[{rad}] \leavevmode{[}class radiation{]}
the radiation field impinging on the cloud

\item[{emitters}] \leavevmode{[}dict{]}
keys of the dict are names of emitting species, and values
are objects of class emitter

\item[{chemnetwork}] \leavevmode{[}chemical network class (optional){]}
a chemical network that is to be used to perform
time-dependent chemical evolution calcualtions for this cloud

\end{description}

\end{description}
\index{abundances (despotic.cloud attribute)}

\begin{fulllineitems}
\phantomsection\label{fulldoc:despotic.cloud.abundances}\pysigline{\bfcode{abundances}}
This property contains the abundances of all emitting species,
stored as an abundanceDict

\end{fulllineitems}

\index{addEmitter() (despotic.cloud method)}

\begin{fulllineitems}
\phantomsection\label{fulldoc:despotic.cloud.addEmitter}\pysiglinewithargsret{\bfcode{addEmitter}}{\emph{emitName}, \emph{emitAbundance}, \emph{emitterFile=None}, \emph{emitterURL=None}, \emph{energySkip=False}, \emph{extrap=True}}{}
Method to add an emitting species
\begin{description}
\item[{Pamameters}] \leavevmode\begin{description}
\item[{emitName}] \leavevmode{[}string{]}
name of the emitting species

\item[{emitAbundance}] \leavevmode{[}float{]}
abundance of the emitting species relative to H

\item[{emitterFile}] \leavevmode{[}string{]}
name of LAMDA file containing data on this species; this
option overrides the default

\item[{emitterURL}] \leavevmode{[}string{]}
URL of LAMDA file containing data on this species; this
option overrides the default

\item[{energySkip}] \leavevmode{[}Boolean{]}
if set to True, this species is ignored when calculating
line cooling rates

\item[{extrap}] \leavevmode{[}Boolean{]}
If set to True, collision rate coefficients for this species
will be extrapolated to temperatures outside the range given
in the LAMDA table. If False, no extrapolation is perfomed,
and providing temperatures outside the range in the table
produces an error

\end{description}

\item[{Returns}] \leavevmode
Nothing

\end{description}

\end{fulllineitems}

\index{chemEvol() (despotic.cloud method)}

\begin{fulllineitems}
\phantomsection\label{fulldoc:despotic.cloud.chemEvol}\pysiglinewithargsret{\bfcode{chemEvol}}{\emph{tFin}, \emph{tInit=0.0}, \emph{nOut=100}, \emph{dt=None}, \emph{tOut=None}, \emph{network=None}, \emph{info=None}, \emph{addEmitters=False}, \emph{evolveTemp='fixed'}, \emph{isobaric=False}, \emph{tempEqParam=None}, \emph{dEdtParam=None}}{}
Evolve the chemical abundances of this cloud in time.
\begin{description}
\item[{Parameters}] \leavevmode\begin{description}
\item[{tFin}] \leavevmode{[}float{]}
end time of integration, in sec

\item[{tInit}] \leavevmode{[}float{]}
start time of integration, in sec

\item[{nOut}] \leavevmode{[}int{]}
number of times at which to report the temperature; this
is ignored if dt or tOut are set

\item[{dt}] \leavevmode{[}float{]}
time interval between outputs, in sec; this is ignored if
tOut is set

\item[{tOut}] \leavevmode{[}array{]}
list of times at which to output the temperature, in sec;
must be sorted in increasing order

\item[{network}] \leavevmode{[}chemical network class{]}
a valid chemical network class; this class must define the
methods \_\_init\_\_, dxdt, and applyAbundances; if None, the
existing chemical network for the cloud is used

\item[{info}] \leavevmode{[}dict{]}
a dict of additional initialization information to be passed
to the chemical network class when it is instantiated

\item[{addEmitters}] \leavevmode{[}Boolean{]}
if True, emitters that are included in the chemical
network but not in the cloud's existing emitter list will
be added; if False, abundances of emitters already in the
emitter list will be updated, but new emiters will not be
added to the cloud

\item[{evolveTemp}] \leavevmode{[}`fixed' \textbar{} `gasEq' \textbar{} `fullEq' \textbar{} `evol'{]}
how to treat the temperature evolution during the
chemical evolution; `fixed' = treat tempeature as fixed;
`gasEq' = hold dust temperature fixed, set gas temperature
to instantaneous equilibrium value; `fullEq' = set gas and
dust temperatures to instantaneous equilibrium values;
`evol' = evolve gas temperature in time along with the
chemistry, assuming the dust is always in instantaneous
equilibrium

\item[{isobaric}] \leavevmode{[}Boolean{]}
if set to True, the gas is assumed to be isobaric during
the evolution (constant pressure); otherwise it is assumed
to be isochoric; note that (since chemistry networks at
present are not allowed to change the mean molecular
weight), this option has no effect if evolveTemp is `fixed'

\item[{tempEqParam}] \leavevmode{[}None \textbar{} dict{]}
if this is not None, then it must be a dict of values that
will be passed as keyword arguments to the cloud.setTempEq,
cloud.setGasTempEq, or cloud.setDustTempEq routines; only
used if evolveTemp is not `fixed'

\item[{dEdtParam}] \leavevmode{[}None \textbar{} dict{]}
if this is not None, then it must be a dict of values that
will be passed as keyword arguments to the cloud.dEdt
routine; only used if evolveTemp is `evol'

\end{description}

\item[{Returns}] \leavevmode\begin{description}
\item[{time}] \leavevmode{[}array of floats{]}
array of output times, in sec

\item[{abundances}] \leavevmode{[}class abundanceDict{]}
an abundanceDict giving the abundances as a function of time

\item[{Tg}] \leavevmode{[}array{]}
gas temperature as a function of time; returned only if
evolveTemp is not `fixed'

\item[{Td}] \leavevmode{[}array{]}
dust temperature as a function of time; returned only if
evolveTemp is not `fixed' or `gasEq'

\end{description}

\item[{Raises}] \leavevmode
despoticError, if network is None and the cloud does not already
have a defined chemical network associated with it

\end{description}

\end{fulllineitems}

\index{chemabundances (despotic.cloud attribute)}

\begin{fulllineitems}
\phantomsection\label{fulldoc:despotic.cloud.chemabundances}\pysigline{\bfcode{chemabundances}}
The property contains the abundances of all species in the
chemical network, stored as an abundanceDict

\end{fulllineitems}

\index{dEdt() (despotic.cloud method)}

\begin{fulllineitems}
\phantomsection\label{fulldoc:despotic.cloud.dEdt}\pysiglinewithargsret{\bfcode{dEdt}}{\emph{c1Grav=0.0}, \emph{thin=False}, \emph{LTE=False}, \emph{fixedLevPop=False}, \emph{noClump=False}, \emph{escapeProbGeom='sphere'}, \emph{PsiUser=None}, \emph{sumOnly=False}, \emph{dustOnly=False}, \emph{gasOnly=False}, \emph{dustCoolOnly=False}, \emph{dampFactor=0.5}, \emph{verbose=False}, \emph{overrideSkip=False}}{}
Return instantaneous values of heating / cooling terms
\begin{description}
\item[{Parameters}] \leavevmode\begin{description}
\item[{c1Grav}] \leavevmode{[}float{]}
if this is non-zero, the cloud is assumed to be
collapsing, and energy is added at a rate
Gamma\_grav = c1 mu\_H m\_H cs\textasciicircum{}2 sqrt(4 pi G rho)

\item[{thin}] \leavevmode{[}Boolean{]}
if set to True, cloud is assumed to be opticall thin

\item[{LTE}] \leavevmode{[}Boolean{]}
if set to True, gas is assumed to be in LTE

\item[{fixedLevPop}] \leavevmode{[}Boolean{]}
if set to True, level populations and escape
probabilities are not recomputed, so the cooling rate is
based on whatever values are stored

\item[{escapeProbGeom}] \leavevmode{[}string, `sphere' or `LVG' or `slab'{]}
specifies the geometry to be assumed in calculating
escape probabilities

\item[{noClump}] \leavevmode{[}Boolean{]}
if set to True, the clumping factor used in estimating
rates for n\textasciicircum{}2 processes is set to unity

\item[{dampFactor}] \leavevmode{[}float{]}
damping factor to use in level population calculations;
see emitter.setLevPopEscapeProb

\item[{PsiUser}] \leavevmode{[}callable{]}
A user-specified function to add additional heating /
cooling terms to the calculation. The function takes the
cloud object as an argument, and must return a two-element
array Psi, where Psi{[}0{]} = gas heating / cooling rate,
Psi{[}1{]} = dust heating / cooling rate. Positive values
indicate heating, negative values cooling, and units are
assumed to be erg s\textasciicircum{}-1 H\textasciicircum{}-1.

\item[{sumOnly}] \leavevmode{[}Boolean{]}
if true, rates contains only four entries: dEdtGas and
dEdtDust give the heating / cooling rates for the
gas and dust summed over all terms, and maxAbsdEdtGas and
maxAbsdEdtDust give the largest of the absolute values of
any of the contributing terms for dust and gas

\item[{gasOnly}] \leavevmode{[}Boolean{]}
if true, the terms GammaISRF, GammaDustLine, LambdaDust,               and PsiUserDust are omitted from rates. If both gasOnly
and sumOnly are true, the dict contains only dEdtGas

\item[{dustOnly}] \leavevmode{[}Boolean{]}
if true, the terms GammaPE, GammaCR, LambdaLine,
GamamDLine, and PsiUserGas are omitted from rates. If both
dustOnly and sumOnly are true, the dict contains only
dEdtDust. Important caveat: the value of dEdtDust returned
in this case will not exactly match that returned if
dustOnly is false, because it will not contain the
contribution from gas line cooling radiation that is
absorbed by the dust

\item[{dustCoolOnly}] \leavevmode{[}Boolean{]}
as dustOnly, but except that now only the terms
LambdaDust, PsiGD, and PsiUserDust are computed

\item[{overrideSkip}] \leavevmode{[}Boolean{]}
if True, energySkip directives are ignored, and cooling
rates are calculated for all species

\end{description}

\item[{Returns}] \leavevmode\begin{description}
\item[{rates}] \leavevmode{[}dict{]}
A dict containing the values of the various heating and
cooling rate terms; all quantities are in units of erg s\textasciicircum{}-1
H\textasciicircum{}-1, and by convention positive = heating, negative =
cooling; for dust-gas exchange, positive indicates heating
of gas, cooling of dust

\end{description}

Elements of the dict are as follows by default, but can be
altered by the additional keywords listed below:
\begin{description}
\item[{GammaPE}] \leavevmode{[}float{]}
photoelectric heating rate

\item[{GammaCR}] \leavevmode{[}float{]}
cosmic ray heating rate

\item[{GammaGrav}] \leavevmode{[}float{]}
gravitational contraction heating rate

\item[{LambdaLine}] \leavevmode{[}dict{]}
cooling rate from lines; dictionary keys correspond to
species in the emitter list, values give line cooling
rate for that species

\item[{LambdaLyA}] \leavevmode{[}float{]}
cooling rate via Lyman alpha emission

\item[{LambdaLyB}] \leavevmode{[}float{]}
cooling rate via Lyman beta emission

\item[{Lambda2p}] \leavevmode{[}float{]}
cooling rate via 2 photon emission

\item[{PsiGD}] \leavevmode{[}float{]}
dust-gas energy exchange rate

\item[{GammaDustISRF}] \leavevmode{[}float{]}
dust heating rate due to the ISRF

\item[{GammaDustCMB}] \leavevmode{[}float{]}
dust heating rate due to CMB

\item[{GammaDustIR}] \leavevmode{[}float{]}
dust heating rate due to IR field

\item[{GammaDustLine}] \leavevmode{[}float{]}
dust heating rate due to absorption of line radiation

\item[{LambdaDust}] \leavevmode{[}float{]}
dust cooling rate due to thermal emission

\item[{PsiUserGas}] \leavevmode{[}float{]}
gas heating / cooling rate from user-specified
function; only included if PsiUser != None

\item[{PsiUserDust}] \leavevmode{[}float{]}
gas heating / cooling rate from user-specified
function; only included is PsiUser != None

\end{description}

\end{description}

\end{fulllineitems}

\index{lineLum() (despotic.cloud method)}

\begin{fulllineitems}
\phantomsection\label{fulldoc:despotic.cloud.lineLum}\pysiglinewithargsret{\bfcode{lineLum}}{\emph{emitName}, \emph{LTE=False}, \emph{noClump=False}, \emph{transition=None}, \emph{thin=False}, \emph{intOnly=False}, \emph{TBOnly=False}, \emph{lumOnly=False}, \emph{escapeProbGeom='sphere'}, \emph{dampFactor=0.5}, \emph{noRecompute=False}, \emph{abstol=1e-08}, \emph{verbose=False}}{}
Return the frequency-integrated intensity of various lines
\begin{description}
\item[{Parameters}] \leavevmode\begin{description}
\item[{emitName}] \leavevmode{[}string{]}
name of the emitter for which the calculation is to be
performed

\item[{LTE}] \leavevmode{[}Boolean{]}
if True, and level populations are unitialized, they will
be initialized to their LTE values; if they are
initialized, this option is ignored

\item[{noClump}] \leavevmode{[}Boolean{]}
if set to True, the clumping factor used in estimating
rates for n\textasciicircum{}2 processes is set to unity

\item[{transition}] \leavevmode{[}list of two arrays{]}
if left as None, luminosity is computed for all
transitions; otherwise only selected transitions are
computed, with transition{[}0{]} = array of upper states
transition{[}1{]} = array of lower states

\item[{thin}] \leavevmode{[}Boolean{]}
if True, the calculation is done assuming the cloud is
optically thin; if level populations are uninitialized,
and LTE is not set, they will be computed assuming the
cloud is optically thin

\item[{intOnly}] \leavevmode{[}Boolean{]}
if true, the output is simply an array containing the
frequency-integrated intensity of the specified lines;
mutually exclusive with TBOnly and lumOnly

\item[{TBOnly}] \leavevmode{[}Boolean{]}
if True, the output is simply an array containing the
velocity-integrated brightness temperatures of the
specified lines; mutually exclusive with intOnly and
lumOnly

\item[{lumOnly}] \leavevmode{[}Boolean{]}
if True, the output is simply an array containing the
luminosity per H nucleus in each of the specified lines;
mutually eclusive with intOnly and TBOonly

\item[{escapeProbGeom}] \leavevmode{[}`sphere' \textbar{} `LVG' \textbar{} `slab'{]}
sets problem geometry that will be assumed in calculating
escape probabilities; ignored if the escape probabilities
are already initialized

\item[{dampFactor}] \leavevmode{[}float{]}
damping factor to use in level population calculations;
see emitter.setLevPopEscapeProb

\item[{noRecompute}] \leavevmode{[}False{]}
if True, level populations and escape probabilities are
not recomputed; instead, stored values are used

\end{description}

\item[{Returns}] \leavevmode
res : list or array

if intOnly, TBOnly, and lumOnly are all False, each element
of the list is a dict containing the following fields:
\begin{description}
\item[{`freq'}] \leavevmode{[}float{]}
frequency of the line in Hz

\item[{`upper'}] \leavevmode{[}int{]}
index of upper state, with ground state = 0 and states
ordered by energy

\item[{`lower'}] \leavevmode{[}int{]}
index of lower state

\item[{`Tupper'}] \leavevmode{[}float{]}
energy of the upper state in K (i.e. energy over kB)

\item[{`Tex'}] \leavevmode{[}float{]}
excitation temperature relating the upper and lower levels

\item[{`intIntensity'}] \leavevmode{[}float{]}
frequency-integrated intensity of the line, with the CMB
contribution subtracted off; units are erg cm\textasciicircum{}-2 s\textasciicircum{}-1 sr\textasciicircum{}-1

\item[{`intTB'}] \leavevmode{[}float{]}
velocity-integrated brightness temperature of the line,
with the CMB contribution subtracted off; units are K km
s\textasciicircum{}-1

\item[{`lumPerH'}] \leavevmode{[}float{]}
luminosity of the line per H nucleus; units are erg s\textasciicircum{}-1
H\textasciicircum{}-1

\item[{`tau'}] \leavevmode{[}float{]}
optical depth in the line, not including dust

\item[{`tauDust'}] \leavevmode{[}float{]}
dust optical depth in the line

\end{description}

\end{description}

if intOnly, TBOnly, or lumOnly are True: res is an array
containing the intIntensity, TB, or lumPerH fields of the dict
described above

\end{fulllineitems}

\index{read() (despotic.cloud method)}

\begin{fulllineitems}
\phantomsection\label{fulldoc:despotic.cloud.read}\pysiglinewithargsret{\bfcode{read}}{\emph{fileName}, \emph{verbose=False}}{}
Read the composition from a file
\begin{description}
\item[{Pamameters}] \leavevmode\begin{description}
\item[{fileName}] \leavevmode{[}string{]}
string giving the name of the composition file

\item[{verbose}] \leavevmode{[}Boolean{]}
print out information about the cloud as it is read

\end{description}

\item[{Returns}] \leavevmode
Nothing

\item[{Remarks}] \leavevmode
For the format of cloud files, see the documentation

\end{description}

\end{fulllineitems}

\index{setChemEq() (despotic.cloud method)}

\begin{fulllineitems}
\phantomsection\label{fulldoc:despotic.cloud.setChemEq}\pysiglinewithargsret{\bfcode{setChemEq}}{\emph{tEqGuess=None}, \emph{network=None}, \emph{info=None}, \emph{addEmitters=False}, \emph{tol=1e-06}, \emph{maxTime=1e+16}, \emph{verbose=False}, \emph{smallabd=1e-15}, \emph{convList=None}, \emph{evolveTemp='fixed'}, \emph{isobaric=False}, \emph{tempEqParam=None}, \emph{dEdtParam=None}, \emph{maxTempIter=50}}{}
Set the chemical abundances for a cloud to their equilibrium
values, computed using a specified chemical network.
\begin{description}
\item[{Parameters}] \leavevmode\begin{description}
\item[{tEqGuess}] \leavevmode{[}float{]}
a guess at the timescale over which equilibrium will be
achieved; if left unspecified, the code will attempt to
estimate this time scale on its own

\item[{network}] \leavevmode{[}chemNetwork object{]}
the chemNetwork object to use; if None, the existing
chemnetwork member of the class (if it exists) is used

\item[{info}] \leavevmode{[}dict{]}
a dict of additional initialization information to be passed
to the chemical network class when it is instantiated

\item[{addEmitters}] \leavevmode{[}Boolean{]}
if True, emitters that are included in the chemical
network but not in the cloud's existing emitter list will
be added; if False, abundances of emitters already in the
emitter list will be updated, but new emiters will not be
added to the cloud

\item[{evolveTemp}] \leavevmode{[}`fixed' \textbar{} `iterate' \textbar{} `iterateDust' \textbar{} `gasEq' \textbar{} `fullEq' \textbar{} `evol'{]}
how to treat the temperature evolution during the chemical
evolution; `fixed' = treat tempeature as fixed; `iterate' =
iterate between setting the gas temperature and chemistry to
equilibrium; `iterateDust' = iterate between setting the gas
and dust temperatures and the chemistry to equilibrium;
`gasEq' = hold dust temperature fixed, set gas temperature to
instantaneous equilibrium value as the chemistry evolves;
`fullEq' = set gas and dust temperatures to instantaneous
equilibrium values while evolving the chemistry network;
`evol' = evolve gas temperature in time along with the
chemistry, assuming the dust is always in instantaneous
equilibrium

\item[{isobaric}] \leavevmode{[}Boolean{]}
if set to True, the gas is assumed to be isobaric during
the evolution (constant pressure); otherwise it is assumed
to be isochoric; note that (since chemistry networks at
present are not allowed to change the mean molecular
weight), this option has no effect if evolveTemp is `fixed'

\item[{tempEqParam}] \leavevmode{[}None \textbar{} dict{]}
if this is not None, then it must be a dict of values that
will be passed as keyword arguments to the cloud.setTempEq,
cloud.setGasTempEq, or cloud.setDustTempEq routines; only
used if evolveTemp is not `fixed'

\item[{dEdtParam}] \leavevmode{[}None \textbar{} dict{]}
if this is not None, then it must be a dict of values that
will be passed as keyword arguments to the cloud.dEdt
routine; only used if evolveTemp is `evol'

\item[{tol}] \leavevmode{[}float{]}
tolerance requirement on the equilibrium solution

\item[{convList}] \leavevmode{[}list{]}
list of species to include when calculating tolerances to
decide if network is converged; species not listed are not
considered. If this is None, then all species are considered
in deciding if the calculation is converged.

\item[{smallabd}] \leavevmode{[}float{]}
abundances below smallabd are not considered when checking for
convergence; set to 0 or a negative value to consider all
abundances, but beware that this may result in false
non-convergence due to roundoff error in very small abundances

\item[{maxTempIter}] \leavevmode{[}int{]}
maximum number of iterations when iterating between chemistry
and temperature; only used if evolveTemp is `iterate' or
`iterateDust'

\item[{verbose}] \leavevmode{[}Boolean{]}
if True, diagnostic information is printed as the calculation
proceeds

\end{description}

\item[{Returns}] \leavevmode\begin{description}
\item[{converged}] \leavevmode{[}Boolean{]}
True if the calculation converged, False if not

\end{description}

\item[{Raises}] \leavevmode
despoticError, if network is None and the cloud does not
already have a defined chemical network associated with it

\item[{Remarks}] \leavevmode
The final abundances are written to the cloud whether or
not the calculation converges.

\end{description}

\end{fulllineitems}

\index{setDustTempEq() (despotic.cloud method)}

\begin{fulllineitems}
\phantomsection\label{fulldoc:despotic.cloud.setDustTempEq}\pysiglinewithargsret{\bfcode{setDustTempEq}}{\emph{PsiUser=None}, \emph{Tdinit=None}, \emph{noLines=False}, \emph{noClump=False}, \emph{verbose=False}, \emph{dampFactor=0.5}}{}
Set Td to equilibrium dust temperature at fixed Tg
\begin{description}
\item[{Parameters}] \leavevmode\begin{description}
\item[{Tdinit}] \leavevmode{[}float{]}
initial guess for gas temperature

\item[{PsiUser}] \leavevmode{[}callable{]}
A user-specified function to add additional heating /
cooling terms to the calculation. The function takes the
cloud object as an argument, and must return a two-element
array Psi, where Psi{[}0{]} = gas heating / cooling rate,
Psi{[}1{]} = dust heating / cooling rate. Positive values
indicate heating, negative values cooling, and units are
assumed to be erg s\textasciicircum{}-1 H\textasciicircum{}-1.

\item[{noLines}] \leavevmode{[}Boolean{]}
If True, line heating of the dust is ignored. This can
make the calculation significantly faster.

\item[{noClump}] \leavevmode{[}Boolean{]}
if set to True, the clumping factor used in estimating
rates for n\textasciicircum{}2 processes is set to unity

\item[{dampFactor}] \leavevmode{[}float{]}
damping factor to use in level population calculations;
see emitter.setLevPopEscapeProb

\item[{verbose}] \leavevmode{[}Boolean{]}
if True, diagnostic information is printed

\end{description}

\item[{Returns}] \leavevmode\begin{description}
\item[{success}] \leavevmode{[}Boolean{]}
True if dust temperature calculation converged, False if
not

\end{description}

\end{description}

\end{fulllineitems}

\index{setGasTempEq() (despotic.cloud method)}

\begin{fulllineitems}
\phantomsection\label{fulldoc:despotic.cloud.setGasTempEq}\pysiglinewithargsret{\bfcode{setGasTempEq}}{\emph{c1Grav=0.0}, \emph{thin=False}, \emph{noClump=False}, \emph{LTE=False}, \emph{Tginit=None}, \emph{fixedLevPop=False}, \emph{escapeProbGeom='sphere'}, \emph{PsiUser=None}, \emph{verbose=False}}{}
Set Tg to equilibrium gas temperature at fixed Td
\begin{description}
\item[{Parameters}] \leavevmode\begin{description}
\item[{c1Grav}] \leavevmode{[}float{]}
if this is non-zero, the cloud is assumed to be
collapsing, and energy is added at a rate
Gamma\_grav = c1 mu\_H m\_H cs\textasciicircum{}2 sqrt(4 pi G rho)

\item[{thin}] \leavevmode{[}Boolean{]}
if set to True, cloud is assumed to be opticall thin

\item[{LTE}] \leavevmode{[}Boolean{]}
if set to True, gas is assumed to be in LTE

\item[{Tginit}] \leavevmode{[}float{]}
initial guess for gas temperature

\item[{fixedLevPop}] \leavevmode{[}Boolean{]}
if set to True, level populations are held fixed
at the starting value, rather than caclculated
self-consistently from the temperature

\item[{escapeProbGeom}] \leavevmode{[}`sphere' \textbar{} `LVG' \textbar{} `slab'{]}
specifies the geometry to be assumed in computing escape
probabilities

\item[{noClump}] \leavevmode{[}Boolean{]}
if set to True, the clumping factor used in estimating
rates for n\textasciicircum{}2 processes is set to unity

\item[{PsiUser}] \leavevmode{[}callable{]}
A user-specified function to add additional heating /
cooling terms to the calculation. The function takes the
cloud object as an argument, and must return a two-element
array Psi, where Psi{[}0{]} = gas heating / cooling rate,
Psi{[}1{]} = dust heating / cooling rate. Positive values
indicate heating, negative values cooling, and units are
assumed to be erg s\textasciicircum{}-1 H\textasciicircum{}-1.

\item[{verbose}] \leavevmode{[}Boolean{]}
if True, print status messages while running

\end{description}

\item[{Returns}] \leavevmode\begin{description}
\item[{success}] \leavevmode{[}Boolean{]}
True if the calculation converges, False if it does not

\end{description}

\end{description}

\end{fulllineitems}

\index{setTempEq() (despotic.cloud method)}

\begin{fulllineitems}
\phantomsection\label{fulldoc:despotic.cloud.setTempEq}\pysiglinewithargsret{\bfcode{setTempEq}}{\emph{c1Grav=0.0}, \emph{thin=False}, \emph{noClump=False}, \emph{LTE=False}, \emph{Tinit=None}, \emph{fixedLevPop=False}, \emph{escapeProbGeom='sphere'}, \emph{PsiUser=None}, \emph{verbose=False}, \emph{tol=0.0001}}{}
Set Tg and Td to equilibrium gas and dust temperatures
\begin{description}
\item[{Parameters}] \leavevmode\begin{description}
\item[{c1Grav}] \leavevmode{[}float{]}
coefficient for rate of gas gravitational heating

\item[{thin}] \leavevmode{[}Boolean{]}
if set to True, cloud is assumed to be opticall thin

\item[{LTE}] \leavevmode{[}Boolean{]}
if set to True, gas is assumed to be in LTE

\item[{Tinit}] \leavevmode{[}array(2){]}
initial guess for gas and dust temperature

\item[{noClump}] \leavevmode{[}Boolean{]}
if set to True, the clumping factor used in estimating
rates for n\textasciicircum{}2 processes is set to unity

\item[{fixedLevPop}] \leavevmode{[}Boolean{]}
if set to true, level populations are held fixed
at the starting value, rather than caclculated
self-consistently from the temperature

\item[{escapeProbGeom}] \leavevmode{[}`sphere' \textbar{} `LVG' \textbar{} `slab'{]}
specifies the geometry to be assumed in computing escape
probabilities

\item[{PsiUser}] \leavevmode{[}callable{]}
A user-specified function to add additional heating /
cooling terms to the calculation. The function takes the
cloud object as an argument, and must return a two-element
array Psi, where Psi{[}0{]} = gas heating / cooling rate,
Psi{[}1{]} = dust heating / cooling rate. Positive values
indicate heating, negative values cooling, and units are
assumed to be erg s\textasciicircum{}-1 H\textasciicircum{}-1.

\item[{verbose}] \leavevmode{[}Boolean{]}
if True, the code prints diagnostic information as it runs

\end{description}

\item[{Returns}] \leavevmode\begin{description}
\item[{success}] \leavevmode{[}Boolean{]}
True if the iteration converges, False if it does not

\end{description}

\end{description}

\end{fulllineitems}

\index{setVirial() (despotic.cloud method)}

\begin{fulllineitems}
\phantomsection\label{fulldoc:despotic.cloud.setVirial}\pysiglinewithargsret{\bfcode{setVirial}}{\emph{alphaVir=1.0}, \emph{setColDen=False}, \emph{setnH=False}, \emph{NTonly=False}}{}
Set sigmaNT, colDen, or nH to the value required to give a
virial ratio of unity
\begin{description}
\item[{Parameters}] \leavevmode\begin{description}
\item[{alphaVir}] \leavevmode{[}float{]}
virial ratio to be used in computation; defaults to 1

\item[{setColDen}] \leavevmode{[}Boolean{]}
if True, sigmaNT and nH are fixed, and colDen is
altered to give the desired virial ratio

\item[{setnH}] \leavevmode{[}Boolean{]}
if True, sigmaNT and colDen are fixed, and nH is altered
to give the desired virial ratio

\item[{NTonly}] \leavevmode{[}Boolean{]}
if True, the virial ratio is computed considering only the
non-thermal component of the velocity dispersion

\end{description}

\item[{Returns}] \leavevmode
Nothing

\item[{Remarks}] \leavevmode
By default the routine fixes nH and colDen and computes
sigmaNT, but this can be overridden by specifying either
setColDen or setnH. It is an error to set both of these to
True.

\end{description}

\end{fulllineitems}

\index{tempEvol() (despotic.cloud method)}

\begin{fulllineitems}
\phantomsection\label{fulldoc:despotic.cloud.tempEvol}\pysiglinewithargsret{\bfcode{tempEvol}}{\emph{tFin}, \emph{tInit=0.0}, \emph{c1Grav=0.0}, \emph{noClump=False}, \emph{thin=False}, \emph{LTE=False}, \emph{fixedLevPop=False}, \emph{escapeProbGeom='sphere'}, \emph{nOut=100}, \emph{dt=None}, \emph{tOut=None}, \emph{PsiUser=None}, \emph{isobaric=False}, \emph{fullOutput=False}, \emph{dampFactor=0.5}, \emph{verbose=False}}{}
Calculate the evolution of the gas temperature in time
\begin{description}
\item[{Parameters}] \leavevmode\begin{description}
\item[{tFin}] \leavevmode{[}float{]}
end time of integration, in sec

\item[{tInit}] \leavevmode{[}float{]}
start time of integration, in sec

\item[{c1Grav}] \leavevmode{[}float{]}
coefficient for rate of gas gravitational heating

\item[{thin}] \leavevmode{[}Boolean{]}
if set to True, cloud is assumed to be opticall thin

\item[{LTE}] \leavevmode{[}Boolean{]}
if set to True, gas is assumed to be in LTE

\item[{isobaric}] \leavevmode{[}Boolean{]}
if set to True, cooling is calculated isobarically;
otherwise (default behavior) it is computed
isochorically

\item[{fixedLevPop}] \leavevmode{[}Boolean{]}
if set to true, level populations are held fixed
at the starting value, rather than caclculated
self-consistently from the temperature

\item[{noClump}] \leavevmode{[}Boolean{]}
if set to True, the clumping factor used in estimating
rates for n\textasciicircum{}2 processes is set to unity

\item[{escapeProbGeom}] \leavevmode{[}string, `sphere' or `LVG' or `slab'{]}
specifies the geometry to be assumed in computing escape
probabilities

\item[{nOut}] \leavevmode{[}int{]}
number of times at which to report the temperature; this
is ignored if dt or tOut are set

\item[{dt}] \leavevmode{[}float{]}
time interval between outputs, in s; this is ignored if
tOut is set

\item[{tOut}] \leavevmode{[}array{]}
list of times at which to output the temperature, in s;
must be sorted in increasing order

\item[{PsiUser}] \leavevmode{[}callable{]}
A user-specified function to add additional heating /
cooling terms to the calculation. The function takes the
cloud object as an argument, and must return a two-element
array Psi, where Psi{[}0{]} = gas heating / cooling rate,
Psi{[}1{]} = dust heating / cooling rate. Positive values
indicate heating, negative values cooling, and units are
assumed to be erg s\textasciicircum{}-1 H\textasciicircum{}-1.

\item[{fullOutput}] \leavevmode{[}Boolean{]}
if True, the full cloud state is returned at every time,
as opposed to simply the gas temperature

\item[{dampFactor}] \leavevmode{[}float{]}
damping factor to use in calculating level populations
(see emitter for details)

\end{description}

\item[{Returns}] \leavevmode
if fullOutput == False:
\begin{description}
\item[{Tg}] \leavevmode{[}array{]}
array of gas temperatures at specified times, in K

\item[{time}] \leavevmode{[}array{]}
array of output times, in sec

\end{description}

if fullOutput == True:
\begin{description}
\item[{cloudState}] \leavevmode{[}list{]}
each element of the list is a deepcopy of the cloud
state at the corresponding time; there is one list
element per output time

\item[{time}] \leavevmode{[}array of floats{]}
array of output times, in sec

\end{description}

\item[{Remarks}] \leavevmode
If the settings for nOut, dt, or tOut are such that some of
the output times requested are past tEvol, the cloud will only
be evolved up to time tEvol. Similarly, if the last output
time is less than tEvol, the cloud will still be evolved up to
time tEvol, and the gas temperature Tg will be set to its
value at that time.

\end{description}

\end{fulllineitems}


\end{fulllineitems}



\subsection{\texttt{collPartner}}
\label{fulldoc:collpartner}\index{collPartner (class in despotic)}

\begin{fulllineitems}
\phantomsection\label{fulldoc:despotic.collPartner}\pysiglinewithargsret{\strong{class }\code{despotic.}\bfcode{collPartner}}{\emph{fp}, \emph{nlev}, \emph{extrap=True}}{}
A utility class to store information about a particular collision
partner for a given species, and to interpolate collision rates
from those data
\begin{description}
\item[{Parameters}] \leavevmode\begin{description}
\item[{fp}] \leavevmode{[}file{]}
a file object that points to the start of the collision
rate data for one species in a LAMDA file

\item[{nlev}] \leavevmode{[}int{]}
number of levels for the emitting species

\item[{extrap}] \leavevmode{[}Boolean{]}
if True, then computing the collision rate with a
temperature that is outside the table will result in the
maximum or minimum value in the table being returned; if
False, it will raise an error

\end{description}

\item[{Class attributes}] \leavevmode\begin{description}
\item[{nlev}] \leavevmode{[}int{]}
number of energy levels for the emitting species

\item[{ntrans}] \leavevmode{[}int{]}
number of collisional transitions in the data table

\item[{ntemp}] \leavevmode{[}int{]}
number of temperatures in the data table

\item[{tempTable}] \leavevmode{[}array(ntemp){]}
list of temperatues at which collision rate coefficients are
given

\item[{colUpper}] \leavevmode{[}int array(ntrans){]}
list of upper states for collisions

\item[{colLower}] \leavevmode{[}int array(ntrans){]}
list of lower states for collisions

\item[{colRate}] \leavevmode{[}array(ntrans, ntemp){]}
table of downward collision rate coefficients, in cm\textasciicircum{}3 s\textasciicircum{}-1

\item[{colRateInterp}] \leavevmode{[}list(ntrans) of functions{]}
each function in the list takes one variable, the temperature,
as an argument, and returns the collision rate coefficient for
the corresponding transition at the given temperature; only
downard transitions are included

\end{description}

\end{description}
\index{colRateMatrix() (despotic.collPartner method)}

\begin{fulllineitems}
\phantomsection\label{fulldoc:despotic.collPartner.colRateMatrix}\pysiglinewithargsret{\bfcode{colRateMatrix}}{\emph{temp}, \emph{levWgt}, \emph{levTemp}}{}
Return interpolated collision rates for all transitions at a
given temperature, stored as an nlev x nlev matrix.
\begin{description}
\item[{Parameters}] \leavevmode\begin{description}
\item[{temp}] \leavevmode{[}float{]}
temperature at which collision rates are computed, in K

\item[{levWgt}] \leavevmode{[}array{]}
array of statistical weights for each level

\item[{levTemp}] \leavevmode{[}array{]}
array of level energies, measured in K

\end{description}

\item[{Returns}] \leavevmode\begin{description}
\item[{k}] \leavevmode{[}array, shape (nlev, nlev){]}
collision rates at the specified temperature; element i,j
of the matrix gives the collision rate from state i to
state j

\end{description}

\end{description}

\end{fulllineitems}

\index{colRates() (despotic.collPartner method)}

\begin{fulllineitems}
\phantomsection\label{fulldoc:despotic.collPartner.colRates}\pysiglinewithargsret{\bfcode{colRates}}{\emph{temp}, \emph{transition=None}}{}
Return interpolated collision rates for all transitions at a
given temperature or list of temperatures
\begin{description}
\item[{Parameters}] \leavevmode\begin{description}
\item[{temp}] \leavevmode{[}float \textbar{} array{]}
temperature(s) at which collision rates are computed, in K

\item[{transition}] \leavevmode{[}array of int, shape (2, N){]}
list of upper and lower states for which collision rates
are to be computed; default behavior is to computer for
all known transitions

\end{description}

\item[{Returns}] \leavevmode\begin{description}
\item[{rates}] \leavevmode{[}array, shape (ntrans) \textbar{} array, shape (ntrans, ntemp){]}
collision rates at the specified temperature(s)

\end{description}

\end{description}

\end{fulllineitems}


\end{fulllineitems}



\subsection{\texttt{composition}}
\label{fulldoc:composition}\index{composition (class in despotic)}

\begin{fulllineitems}
\phantomsection\label{fulldoc:despotic.composition}\pysigline{\strong{class }\code{despotic.}\bfcode{composition}}
A class describing the chemical composition of the dominant
components (hydrogen, helium, electrons) of an interstellar cloud,
and for computing various quantities from them.
\begin{description}
\item[{Parameters}] \leavevmode
None

\item[{Class attributes}] \leavevmode\begin{description}
\item[{xHI}] \leavevmode{[}float{]}
abundance of HI per H nucleus

\item[{xoH2}] \leavevmode{[}float{]}
abundance of ortho-H2 per H nucleus (note that the maximum
possible value of xoH2 is 0.5, since it is per H nucleus)

\item[{xpH2}] \leavevmode{[}float{]}
abundance of para-H2 per H nucleus  (note that the maximum
possible value of xoH2 is 0.5, since it is per H nucleus)

\item[{xH2}] \leavevmode{[}float{]}
sum of xoH2 and xpH2

\item[{xHe}] \leavevmode{[}float{]}
abundance of He per H nucleus

\item[{xe}] \leavevmode{[}float{]}
abundance of free electrons per H nucleus

\item[{xHplus}] \leavevmode{[}float{]}
abundance of H+ per H nucleus

\item[{mu}] \leavevmode{[}float{]}
mean mass per free particle, in units of H mass

\item[{muH}] \leavevmode{[}float{]}
mean mass per H nucleus, in units of H mass

\item[{qIon}] \leavevmode{[}float{]}
energy added to the gas per primary CR / x-ray ionization

\item[{cv}] \leavevmode{[}float{]}
dimensionless specific heat per H nucleus at constant volume;
the usual specific heat per unit volume may be obtained by
multiplying this by nH * kB, and the specific heat per unit
mass may be obtained by multiplying by nH * muH * kB

\end{description}

\end{description}
\index{computeCv() (despotic.composition method)}

\begin{fulllineitems}
\phantomsection\label{fulldoc:despotic.composition.computeCv}\pysiglinewithargsret{\bfcode{computeCv}}{\emph{T}, \emph{noSet=False}, \emph{Jmax=40}}{}
Compute the dimensionless specific heat per H nucleus; the
dimensional specific heat per H nucleus is this value
multiplied by kB, the dimensional specific heat per unit
volume is this value multiplied by kB * nH, and the
dimensional specific heat per unit mass is this value
multiplied by kB * nH * muH
\begin{description}
\item[{Parameters}] \leavevmode\begin{description}
\item[{T}] \leavevmode{[}float \textbar{} array{]}
temperature in K

\item[{noSet}] \leavevmode{[}Boolean{]}
if True, the value of cv stored in the class is not
altered, but the calculated cv is still returned

\item[{Jmax}] \leavevmode{[}int{]}
maximum J to be used in evaluating the rotational
partition function; should be set to a value such that T
\textless{}\textless{} J(J+1) * thetaRot, there thetaRot = 85.3 K. Defaults to
40.

\end{description}

\item[{Returns}] \leavevmode\begin{description}
\item[{cv}] \leavevmode{[}float \textbar{} array{]}
value of cv

\end{description}

\end{description}

\end{fulllineitems}

\index{computeDerived() (despotic.composition method)}

\begin{fulllineitems}
\phantomsection\label{fulldoc:despotic.composition.computeDerived}\pysiglinewithargsret{\bfcode{computeDerived}}{\emph{nH}}{}
Compute the derived quantities mu, muH, qIon
\begin{description}
\item[{Parameters}] \leavevmode\begin{description}
\item[{nH}] \leavevmode{[}float{]}
volume density in H cm\textasciicircum{}-3

\end{description}

\item[{Returns}] \leavevmode
Nothing

\item[{Remarks}] \leavevmode
For the purposes of this procedure, we treat electrons as
massless.

\end{description}

\end{fulllineitems}

\index{computeEint() (despotic.composition method)}

\begin{fulllineitems}
\phantomsection\label{fulldoc:despotic.composition.computeEint}\pysiglinewithargsret{\bfcode{computeEint}}{\emph{T}, \emph{Jmax=40}}{}
Compute the dimensionless internal energy per H nucleus; the internal
energy per H nucleus in K is this value multiplied by T, and
the internal energy per H nucleus in erg is this value
multiplied by kB * T
\begin{description}
\item[{Parameters}] \leavevmode\begin{description}
\item[{T}] \leavevmode{[}float \textbar{} array{]}
temperature in K

\item[{Jmax}] \leavevmode{[}int{]}
maximum J to be used in evaluating the rotational
partition function; should be set to a value such that T
\textless{}\textless{} J(J+1) * thetaRot, there thetaRot = 85.3 K. Defaults to
40.

\end{description}

\item[{Returns}] \leavevmode\begin{description}
\item[{Eint}] \leavevmode{[}float \textbar{} array{]}
value of Eint

\end{description}

\end{description}

\end{fulllineitems}


\end{fulllineitems}



\subsection{\texttt{despoticError}}
\label{fulldoc:despoticerror}\index{despoticError (class in despotic)}

\begin{fulllineitems}
\phantomsection\label{fulldoc:despotic.despoticError}\pysiglinewithargsret{\strong{class }\code{despotic.}\bfcode{despoticError}}{\emph{message}}{}
Class derived from Exception to handle exceptions raised by
DESPOTIC-specific errors.
\begin{description}
\item[{Parameters}] \leavevmode\begin{description}
\item[{message}] \leavevmode{[}string{]}
the error message

\end{description}

\end{description}

\end{fulllineitems}



\subsection{\texttt{dustProp}}
\label{fulldoc:dustprop}\index{dustProp (class in despotic)}

\begin{fulllineitems}
\phantomsection\label{fulldoc:despotic.dustProp}\pysigline{\strong{class }\code{despotic.}\bfcode{dustProp}}
A class to hold paremeters describing the properties of dust grains.
\begin{description}
\item[{Parameters}] \leavevmode
None

\item[{Class attributes}] \leavevmode\begin{description}
\item[{sigma10}] \leavevmode{[}float{]}
dust opacity to thermal radiation at 10 K, in cm\textasciicircum{}2 H\textasciicircum{}-1

\item[{sigmaPE}] \leavevmode{[}float{]}
dust opacity averaged over 8 - 13.6 eV, the range that
dominates grain photoelectric heatin

\item[{sigmaISRF}] \leavevmode{[}float{]}
dust opacity averaged the range that dominates grain starlight
heating

\item[{Zd}] \leavevmode{[}float{]}
dust abundance normalized to solar neighborhood value

\item[{beta}] \leavevmode{[}float{]}
dust spectral index in the mm, sigma \textasciitilde{} nu\textasciicircum{}beta

\item[{alphaGD}] \leavevmode{[}float{]}
grain-gas coupling coefficient

\end{description}

\end{description}

\end{fulllineitems}



\subsection{\texttt{emitter}}
\label{fulldoc:emitter}\label{fulldoc:sssec-full-emitter}\index{emitter (class in despotic)}

\begin{fulllineitems}
\phantomsection\label{fulldoc:despotic.emitter}\pysiglinewithargsret{\strong{class }\code{despotic.}\bfcode{emitter}}{\emph{emitName}, \emph{emitAbundance}, \emph{extrap=True}, \emph{energySkip=False}, \emph{emitterFile=None}, \emph{emitterURL=None}}{}
Class to store the properties of a single emitting species, and
preform computations on those properties. Note that all quantities
are stored in cgs units.
\begin{quote}
\begin{description}
\item[{Parameters}] \leavevmode\begin{description}
\item[{emitName}] \leavevmode{[}string{]}
name of this species

\item[{emitAbundance}] \leavevmode{[}float{]}
abundance of species relative to H

\item[{emitterFile}] \leavevmode{[}string{]}
name of LAMDA file containing data on this species; this
option overrides the default

\item[{emitterURL}] \leavevmode{[}string{]}
URL of LAMDA file containing data on this species; this
option overrides the default

\item[{energySkip}] \leavevmode{[}Boolean{]}
if True, this species is skipped when computing line
cooling rates

\item[{extrap}] \leavevmode{[}Boolean{]}
if True, collision rate coefficients for this species will
be extrapolated to temperatures outside the range given in
the LAMDA tables

\end{description}

\item[{Returns}] \leavevmode
Nothing

\end{description}
\end{quote}
\begin{description}
\item[{Class attributes}] \leavevmode\begin{description}
\item[{name}] \leavevmode{[}string{]}
name of emitting species

\item[{abundance}] \leavevmode{[}float{]}
abundance of emitting species relative to H nuclei

\item[{data}] \leavevmode{[}class emitterData{]}
physical data for this emitter

\item[{levPop}] \leavevmode{[}array, shape(nlev){]}
array giving populations of each level

\item[{levPopInitialized}] \leavevmode{[}Boolean{]}
flag for whether levPop is initialized or uninitialized

\item[{escapeProb}] \leavevmode{[}array, shape (nlev, nlev){]}
2d array giving escape probability for photons emitted in a
line connecting the given level pair

\item[{escapeProbInitialized}] \leavevmode{[}Boolean{]}
flag for whether escapeProb is initialized or uninitialized

\item[{energySkip}] \leavevmode{[}Boolean{]}
flag that this species should be skipped when computing line
cooling rates

\end{description}

\end{description}
\index{Xthin() (despotic.emitter method)}

\begin{fulllineitems}
\phantomsection\label{fulldoc:despotic.emitter.Xthin}\pysiglinewithargsret{\bfcode{Xthin}}{\emph{trans=None}}{}
Returns the Xthin parameter of Krumholz+ (2016); this function
is just a thin wrapper around emitterData.Xthin
\begin{description}
\item[{Parameters}] \leavevmode\begin{description}
\item[{trans}] \leavevmode{[}int, array, or None{]}
if set, Xthin is returned only for the specified
transitions in the transition list; default is that it
is returned for all transitions

\end{description}

\item[{Returns}] \leavevmode\begin{description}
\item[{Xthin}] \leavevmode{[}float or array{]}
Xthin parameter for the specified transitions, in cm\textasciicircum{}-2
/ (K km s\textasciicircum{}-1)

\end{description}

\end{description}

\end{fulllineitems}

\index{luminosityPerH() (despotic.emitter method)}

\begin{fulllineitems}
\phantomsection\label{fulldoc:despotic.emitter.luminosityPerH}\pysiglinewithargsret{\bfcode{luminosityPerH}}{\emph{rad}, \emph{transition=None}, \emph{total=False}, \emph{thin=False}}{}
Return the luminosities of various lines, computed from the
stored level populations and escape probabilities.
\begin{description}
\item[{Parameters}] \leavevmode\begin{description}
\item[{rad}] \leavevmode{[}class radiation{]}
radiation field impinging on the cloud

\item[{transition}] \leavevmode{[}list{]}
A list containing two 1D arrays of integer type;
transition{[}0{]} = array of upper states, transition{[}1{]} =
array of lower states

\item[{total}] \leavevmode{[}Boolean{]}
if True, the routine returns a single float rather than an
array; this float is the sum of the luminosities per H
nucleus of all lines

\end{description}

\item[{Returns}] \leavevmode\begin{description}
\item[{lum}] \leavevmode{[}array{]}
luminosities per H in specified lines; by default

\end{description}

\item[{Raises}] \leavevmode
despoticError, if the level populations are not initialized,
or if the escape probabilities are not initialized and thin is
not True

\end{description}

\end{fulllineitems}

\index{opticalDepth() (despotic.emitter method)}

\begin{fulllineitems}
\phantomsection\label{fulldoc:despotic.emitter.opticalDepth}\pysiglinewithargsret{\bfcode{opticalDepth}}{\emph{transition=None}, \emph{escapeProbGeom='sphere'}}{}
Return the optical depths of various lines, computed from the
stored escape probabilities.
\begin{description}
\item[{Parameters}] \leavevmode\begin{description}
\item[{transition}] \leavevmode{[}list{]}
A list containing two 1D arrays of integer type;
transition{[}0{]} = array of upper states, transition{[}1{]} =
array of lower states

\item[{escapeProbGeom}] \leavevmode{[}`sphere' \textbar{} `LVG' \textbar{} `slab'{]}
sets problem geometry that will be assumed in calculating
escape probabilities

\end{description}

\item[{Returns}] \leavevmode\begin{description}
\item[{tau}] \leavevmode{[}array{]}
optical depths in specified lines; by default

\end{description}

\end{description}

\end{fulllineitems}

\index{setEscapeProb() (despotic.emitter method)}

\begin{fulllineitems}
\phantomsection\label{fulldoc:despotic.emitter.setEscapeProb}\pysiglinewithargsret{\bfcode{setEscapeProb}}{\emph{thisCloud}, \emph{transition=None}, \emph{escapeProbGeom='sphere'}}{}
Compute escape probabilities from stored level populations
\begin{description}
\item[{Parameters}] \leavevmode\begin{description}
\item[{thisCloud}] \leavevmode{[}class cloud{]}
a cloud object containing the physical and chemical
properties of this cloud

\item[{transition}] \leavevmode{[}list{]}
list of transition for which to set escape probability;
transition{[}0{]} = array of upper states, transition{[}1{]} =
array of lower states

\item[{escapeProbGeom}] \leavevmode{[}`sphere' \textbar{} `LVG' \textbar{} `slab'{]}
sets problem geometry that will be assumed in calculating
escape probabilities

\end{description}

\item[{Returns}] \leavevmode
Nothing

\end{description}

\end{fulllineitems}

\index{setLevPop() (despotic.emitter method)}

\begin{fulllineitems}
\phantomsection\label{fulldoc:despotic.emitter.setLevPop}\pysiglinewithargsret{\bfcode{setLevPop}}{\emph{thisCloud}, \emph{thin=False}, \emph{noClump=False}, \emph{diagOnly=False}}{}
Compute the level populations for this species using the
stored escape probabilities
\begin{description}
\item[{Parameters}] \leavevmode\begin{description}
\item[{thisCloud}] \leavevmode{[}class cloud{]}
a cloud object containing the physical and chemical
properties of this cloud

\item[{thin}] \leavevmode{[}Boolean{]}
if True, the stored escape probabilities are ignored, and
the cloud is assumed to be optically thin (equivalent to
assuming all escape probabilities are 1)

\item[{noClump}] \leavevmode{[}Boolean{]}
if set to True, the clumping factor used in estimating
rates for n\textasciicircum{}2 processes is set to unity

\item[{diagOnly}] \leavevmode{[}Boolean{]}
if true, diagnostic information is returned, but no
attempt is made to solve the equations or calculate the
level popuplations (useful for debugging)

\end{description}

\item[{Returns}] \leavevmode\begin{description}
\item[{infoDict}] \leavevmode{[}dict{]}
A dictionary containing a variety of diagnostic
information

\end{description}

\end{description}

\end{fulllineitems}

\index{setLevPopEscapeProb() (despotic.emitter method)}

\begin{fulllineitems}
\phantomsection\label{fulldoc:despotic.emitter.setLevPopEscapeProb}\pysiglinewithargsret{\bfcode{setLevPopEscapeProb}}{\emph{thisCloud}, \emph{escapeProbGeom='sphere'}, \emph{noClump=False}, \emph{verbose=False}, \emph{reltol=1e-06}, \emph{abstol=1e-08}, \emph{maxiter=200}, \emph{veryverbose=False}, \emph{dampFactor=0.5}}{}
Compute escape probabilities and level populations
simultaneously.
\begin{description}
\item[{Parameters}] \leavevmode\begin{description}
\item[{thisCloud}] \leavevmode{[}class cloud{]}
a cloud object containing the physical and chemical
properties of this cloud

\item[{escapeProbGeom}] \leavevmode{[}`sphere' \textbar{} `LVG' \textbar{} `slab'{]}
sets problem geometry that will be assumed in calculating
escape probabilities

\item[{noClump}] \leavevmode{[}Boolean{]}
if set to True, the clumping factor used in estimating
rates for n\textasciicircum{}2 processes is set to unity

\end{description}

\item[{Returns}] \leavevmode\begin{description}
\item[{success: Boolean}] \leavevmode
True if iteration converges, False if it does not

\end{description}

\item[{Additional Parameters}] \leavevmode\begin{description}
\item[{verbose}] \leavevmode{[}Boolean{]}
if True, diagnostic information is printed

\item[{veryverbose}] \leavevmode{[}Boolean{]}
if True, a very large amount of diagnostic information is
printed; probably useful only for debugging

\item[{reltol}] \leavevmode{[}float{]}
relative tolerance; convergence is considered to have
occured when the absolute value of the difference
between two iterations, divided by the larger of the two
results being differences, is less than reltol

\item[{abstol}] \leavevmode{[}float{]}
absolute tolerance; convergence is considered to have
occured when the absolute value of the difference
between two iterations is less than abstol

\item[{maxiter}] \leavevmode{[}int{]}
maximum number of iterations to allow

\item[{dampFactor}] \leavevmode{[}float{]}
a number in the range (0, 1{]} that damps out changes in level
populations at each iteration. A value of 1 means no
damping, while a value of 0 means the level populations
never change.

\end{description}

\item[{Remarks}] \leavevmode
Convergence occurs when either the relative or the absolute
tolerance condition is satisfied. To disable either relative
or absolute tolerance checking, just set the appropriate
tolerance \textless{}= 0. However, be warned that in many circumstances
disabling absolute tolerances will gaurantee non-convergence,
because truncation errors tend to produce large relative
residuals for high energy states whose populations are very
low, and no amount of iterating will reduce these errors
substantially.

\end{description}

\end{fulllineitems}

\index{setLevPopLTE() (despotic.emitter method)}

\begin{fulllineitems}
\phantomsection\label{fulldoc:despotic.emitter.setLevPopLTE}\pysiglinewithargsret{\bfcode{setLevPopLTE}}{\emph{temp}}{}
Set the level populations of this species to their LTE values
\begin{description}
\item[{Parameters}] \leavevmode\begin{description}
\item[{temp}] \leavevmode{[}float{]}
temperature in K

\end{description}

\item[{Returns}] \leavevmode
Nothing

\end{description}

\end{fulllineitems}

\index{setThin() (despotic.emitter method)}

\begin{fulllineitems}
\phantomsection\label{fulldoc:despotic.emitter.setThin}\pysiglinewithargsret{\bfcode{setThin}}{}{}
Set the escape probabilities for this species to unity
\begin{description}
\item[{Parameters}] \leavevmode
None

\item[{Returns}] \leavevmode
Nothing

\end{description}

\end{fulllineitems}

\index{tX() (despotic.emitter method)}

\begin{fulllineitems}
\phantomsection\label{fulldoc:despotic.emitter.tX}\pysiglinewithargsret{\bfcode{tX}}{\emph{mX}, \emph{trans=None}}{}
Returns the tX line strength parameter of Krumholz+ (2016);
this is just a thin wrapper around emitterData.tX
\begin{description}
\item[{Parameters}] \leavevmode\begin{description}
\item[{mX}] \leavevmode{[}float{]}
total mass per particle of this species, in g

\item[{trans}] \leavevmode{[}int, array, or None{]}
if set, tX is returned only for the specified
transitions in the transition list; default is that it
is returned for all transitions

\end{description}

\item[{Returns}] \leavevmode\begin{description}
\item[{tX}] \leavevmode{[}float or array{]}
transition strength parameter for the specified
transitions, in seconds

\end{description}

\end{description}

\end{fulllineitems}


\end{fulllineitems}



\subsection{\texttt{emitterData}}
\label{fulldoc:emitterdata}\label{fulldoc:sssec-full-emitterdata}\index{emitterData (class in despotic)}

\begin{fulllineitems}
\phantomsection\label{fulldoc:despotic.emitterData}\pysiglinewithargsret{\strong{class }\code{despotic.}\bfcode{emitterData}}{\emph{emitName}, \emph{emitterFile=None}, \emph{emitterURL=None}, \emph{extrap=True}, \emph{noRefresh=False}}{}
Class to store the physical properties of a single emitting
species, and preform computations on those properties. Note that
all quantities are stored in cgs units.
\begin{description}
\item[{Parameters}] \leavevmode\begin{description}
\item[{emitName}] \leavevmode{[}string{]}
name of this species

\item[{emitterFile}] \leavevmode{[}string{]}
name of LAMDA file containing data on this species; this
option overrides the default

\item[{emitterURL}] \leavevmode{[}string{]}
URL of LAMDA file containing data on this species; this
option overrides the default

\item[{extrap}] \leavevmode{[}Boolean{]}
if True, collision rate coefficients for this species will
be extrapolated to temperatures outside the range given in
the LAMDA tables

\item[{noRefresh}] \leavevmode{[}Boolean{]}
if True, the routine will not attempt to automatically
fetch updated versions of files from the web

\end{description}

\item[{Class attributes}] \leavevmode\begin{description}
\item[{name}] \leavevmode{[}string{]}
name of emitting species

\item[{lamdaFile}] \leavevmode{[}string{]}
name of file from which species was read

\item[{molWgt}] \leavevmode{[}float{]}
molecular weight of species, in units of H masses

\item[{nlev}] \leavevmode{[}int{]}
number of energy levels of the species

\item[{levEnergy}] \leavevmode{[}array, shape(nlev){]}
energies of levels

\item[{levTemp}] \leavevmode{[}array, shape(nlev){]}
energies of levels in K (i.e. levEnergy / kB)

\item[{levWgt}] \leavevmode{[}array shape(nlev){]}
degeneracies (statistical weights) of levels

\item[{nrad}] \leavevmode{[}int{]}
number of radiative transition this species has

\item[{radUpper}] \leavevmode{[}integer array, shape (nrad){]}
array containing upper states for radiative transitions

\item[{radLower}] \leavevmode{[}integer array, shape (nrad){]}
array containing lower states for radiative transitions

\item[{radFreq}] \leavevmode{[}array, shape (nrad){]}
array of frequencies of radiative transitions

\item[{radTemp}] \leavevmode{[}array, shape (nrad){]}
same as radFreq, but multiplied by h/kB to give units of K

\item[{radTUpper}] \leavevmode{[}array, shape (nrad){]}
array of temperatures (E/kB) of upper radiative states

\item[{radA}] \leavevmode{[}array, shape (nrad){]}
array of Einstin A coefficients of radiative transitions

\item[{partners}] \leavevmode{[}dict{]}
listing collision partners; keys are partner names, values are
objects of class collPartner

\item[{EinsteinA}] \leavevmode{[}array, shape (nlev, nlev){]}
2d array of nlev x nlev giving Einstein A's for radiative
transitions connecting each level pair

\item[{freq}] \leavevmode{[}array, shape (nlev, nlev){]}
2d array of nlev x nlev giving frequency of radiative
transitions connecting each level pair

\item[{dT}] \leavevmode{[}array, shape (nlev, nlev){]}
2d array of nlev x nlev giving energy difference between each
level pair in K; it is positive for i \textgreater{} j, and negative for i
\textless{} j

\item[{wgtRatio}] \leavevmode{[}array, shape (nlev, nlev){]}
2d array giving ratio of statistical weights of each level
pair

\item[{extrap}] \leavevmode{[}Boolean{]}
if True, collision rate coefficients for this emitter are
allowed to be extrapolated off the data table

\end{description}

\end{description}
\index{Xthin() (despotic.emitterData method)}

\begin{fulllineitems}
\phantomsection\label{fulldoc:despotic.emitterData.Xthin}\pysiglinewithargsret{\bfcode{Xthin}}{\emph{abd}, \emph{trans=None}}{}
Returns the Xthin parameter of Krumholz+ (2016)
\begin{description}
\item[{Parameters}] \leavevmode\begin{description}
\item[{abd}] \leavevmode{[}float{]}
abundance of the species, relative to H

\item[{trans}] \leavevmode{[}int, array, or None{]}
if set, Xthin is returned only for the specified
transitions in the transition list; default is that it
is returned for all transitions

\end{description}

\item[{Returns}] \leavevmode\begin{description}
\item[{Xthin}] \leavevmode{[}float or array{]}
Xthin parameter for the specified transitions, in cm\textasciicircum{}-2
/ (K km s\textasciicircum{}-1)

\end{description}

\end{description}

\end{fulllineitems}

\index{alphathin() (despotic.emitterData method)}

\begin{fulllineitems}
\phantomsection\label{fulldoc:despotic.emitterData.alphathin}\pysiglinewithargsret{\bfcode{alphathin}}{\emph{mX}, \emph{trans=None}}{}
Returns the Xthin parameter of Krumholz+ (2016)
\begin{description}
\item[{Parameters}] \leavevmode\begin{description}
\item[{mX}] \leavevmode{[}float{]}
total mass per particle of this species, in g

\item[{trans}] \leavevmode{[}int, array, or None{]}
if set, Xthin is returned only for the specified
transitions in the transition list; default is that it
is returned for all transitions

\end{description}

\item[{Returns}] \leavevmode\begin{description}
\item[{Xthin}] \leavevmode{[}float or array{]}
Xthin parameter for the specified transitions, in cm\textasciicircum{}-2
/ (K km s\textasciicircum{}-1)

\end{description}

\end{description}

\end{fulllineitems}

\index{collRateMatrix() (despotic.emitterData method)}

\begin{fulllineitems}
\phantomsection\label{fulldoc:despotic.emitterData.collRateMatrix}\pysiglinewithargsret{\bfcode{collRateMatrix}}{\emph{nH}, \emph{comp}, \emph{temp}}{}
This routine computes the matrix of collision rates (not rate
coefficients) between every pair of levels.
\begin{description}
\item[{Parameters}] \leavevmode\begin{description}
\item[{nH}] \leavevmode{[}float{]}
number density of H nuclei

\item[{comp}] \leavevmode{[}class composition{]}
bulk composition of the gas

\item[{temp}] \leavevmode{[}float{]}
gas kinetic temperature

\end{description}

\item[{Returns}] \leavevmode\begin{description}
\item[{q}] \leavevmode{[}array, shape (nlev, nlev){]}
array in which element ij is the rate of collisional
transitions from state i to state j, in s\textasciicircum{}-1

\end{description}

\end{description}

\end{fulllineitems}

\index{partFunc() (despotic.emitterData method)}

\begin{fulllineitems}
\phantomsection\label{fulldoc:despotic.emitterData.partFunc}\pysiglinewithargsret{\bfcode{partFunc}}{\emph{temp}}{}
Compute the partition function for this species at the given
temperature.
\begin{description}
\item[{Parameters}] \leavevmode\begin{description}
\item[{temp}] \leavevmode{[}float \textbar{} array{]}
gas kinetic temperature

\end{description}

\item[{Returns}] \leavevmode\begin{description}
\item[{Z}] \leavevmode{[}float \textbar{} array{]}
the partition function Z(T) for this species

\end{description}

\end{description}

\end{fulllineitems}

\index{readLamda() (despotic.emitterData method)}

\begin{fulllineitems}
\phantomsection\label{fulldoc:despotic.emitterData.readLamda}\pysiglinewithargsret{\bfcode{readLamda}}{\emph{fp}, \emph{extrap=False}}{}
Read a LAMDA-formatted file
\begin{description}
\item[{Parameters}] \leavevmode\begin{description}
\item[{fp}] \leavevmode{[}file{]}
pointer to the start of a LAMDA-formatted file

\item[{extrap}] \leavevmode{[}Boolean{]}
if True, collision rate coefficients for this species will
be extrapolated to temperatures outside the range given in
the LAMDA tables

\end{description}

\item[{Returns}] \leavevmode
Nothing

\end{description}

\end{fulllineitems}

\index{tX() (despotic.emitterData method)}

\begin{fulllineitems}
\phantomsection\label{fulldoc:despotic.emitterData.tX}\pysiglinewithargsret{\bfcode{tX}}{\emph{mX}, \emph{trans=None}}{}
Returns the tX line strength parameter of Krumholz+ (2016)
\begin{description}
\item[{Parameters}] \leavevmode\begin{description}
\item[{mX}] \leavevmode{[}float{]}
total mass per particle of this species, in g

\item[{trans}] \leavevmode{[}int, array, or None{]}
if set, tX is returned only for the specified
transitions in the transition list; default is that it
is returned for all transitions

\end{description}

\item[{Returns}] \leavevmode\begin{description}
\item[{tX}] \leavevmode{[}float or array{]}
transition strength parameter for the specified
transitions, in seconds

\end{description}

\end{description}

\end{fulllineitems}


\end{fulllineitems}



\subsection{\texttt{radiation}}
\label{fulldoc:radiation}\index{radiation (class in despotic)}

\begin{fulllineitems}
\phantomsection\label{fulldoc:despotic.radiation}\pysigline{\strong{class }\code{despotic.}\bfcode{radiation}}
A class describing the radiation field affecting a cloud.
\begin{description}
\item[{Parameters}] \leavevmode
None

\item[{Class attributes}] \leavevmode\begin{description}
\item[{TCMB}] \leavevmode{[}float{]}
the temperature of the CMB, in K

\item[{TradDust}] \leavevmode{[}float{]}
the temperature of the dust IR background field, in K

\item[{ionRate}] \leavevmode{[}float{]}
the primary ionization rate due to cosmic rays and x-rays, in
s\textasciicircum{}-1 H\textasciicircum{}-1

\item[{chi}] \leavevmode{[}float{]}
strength of the ISRF, normalized to the solar neighborhood
value

\item[{fdDilute}] \leavevmode{[}float{]}
dilution factor for the dust radiation field; a value of 0
means infinite dilution, so the dust does not contribute to the
photon occupation number, while 1 means zero dilution, so
the dust contributes as a full blackbody radiation field at
temperature TradDust

\end{description}

\end{description}
\index{ngamma() (despotic.radiation method)}

\begin{fulllineitems}
\phantomsection\label{fulldoc:despotic.radiation.ngamma}\pysiglinewithargsret{\bfcode{ngamma}}{\emph{Tnu}}{}
Return the photon occupation number from the CMB and dust
radiation fields, equal to 
1 / {[}exp(-h nu / k T\_CMB) - 1{]} + 
fdDilute * 1 / {[}exp(-h nu / k T\_radDust) - 1{]}
\begin{description}
\item[{Parameters}] \leavevmode\begin{description}
\item[{Tnu}] \leavevmode{[}float \textbar{} array{]}
frequency translated into K, i.e. frequency times h/kB

\end{description}

\item[{Returns}] \leavevmode\begin{description}
\item[{ngamma}] \leavevmode{[}float \textbar{} array{]}
photon occupation number

\end{description}

\end{description}

\end{fulllineitems}


\end{fulllineitems}



\subsection{\texttt{zonedcloud}}
\label{fulldoc:sssec-full-zonedcloud}\label{fulldoc:zonedcloud}\index{zonedcloud (class in despotic)}

\begin{fulllineitems}
\phantomsection\label{fulldoc:despotic.zonedcloud}\pysiglinewithargsret{\strong{class }\code{despotic.}\bfcode{zonedcloud}}{\emph{fileName=None}, \emph{colDen=None}, \emph{AV=None}, \emph{nZone=16}, \emph{geometry='sphere'}, \emph{verbose=False}}{}
A class consisting of an interstellar cloud divided into different
column density / extinction zones.
\begin{description}
\item[{Parameters}] \leavevmode\begin{description}
\item[{fileName}] \leavevmode{[}string{]}
name of file from which to read cloud description

\item[{colDen}] \leavevmode{[}array{]}
Array of column densities marking zone centers

\item[{AV}] \leavevmode{[}array{]}
Array of visual extinction values (in mag) marking zone
centers; AV is converted to column density using a V-band
cross section equal to 0.4 * sigmaPE; this argument
ignored if colDen is not None

\item[{nZone}] \leavevmode{[}int{]}
Number of zones into which to divide the cloud, from 0
to the maximum column density found in the cloud
description file fileName; ignored if colDen or AV is
not None

\item[{geometry}] \leavevmode{[}`sphere' \textbar{} `slab'{]}
geometry to assume for the cloud, either `sphere'
(onion-like) or `slab' (layer cake-like)

\item[{verbose}] \leavevmode{[}Boolean{]}
print out information about the cloud as we read it

\end{description}

\end{description}
\index{abundances (despotic.zonedcloud attribute)}

\begin{fulllineitems}
\phantomsection\label{fulldoc:despotic.zonedcloud.abundances}\pysigline{\bfcode{abundances}}
Returns abundances of all emitting species, mass-weighted over
cloud

\end{fulllineitems}

\index{abundances\_zone (despotic.zonedcloud attribute)}

\begin{fulllineitems}
\phantomsection\label{fulldoc:despotic.zonedcloud.abundances_zone}\pysigline{\bfcode{abundances\_zone}}
Return abundances of all emitting species in all zones

\end{fulllineitems}

\index{addEmitter() (despotic.zonedcloud method)}

\begin{fulllineitems}
\phantomsection\label{fulldoc:despotic.zonedcloud.addEmitter}\pysiglinewithargsret{\bfcode{addEmitter}}{\emph{emitName}, \emph{emitAbundance}, \emph{emitterFile=None}, \emph{emitterURL=None}, \emph{energySkip=False}, \emph{extrap=True}}{}
Add an emitting species
\begin{description}
\item[{Pamameters}] \leavevmode\begin{description}
\item[{emitName}] \leavevmode{[}string{]}
name of the emitting species

\item[{emitAbundance}] \leavevmode{[}float or listlike{]}
abundance of the emitting species relative to H; if this
is listlike, it must have the same number of elements as
the number of zones

\item[{emitterFile}] \leavevmode{[}string{]}
name of LAMDA file containing data on this species; this
option overrides the default

\item[{emitterURL}] \leavevmode{[}string{]}
URL of LAMDA file containing data on this species; this
option overrides the default

\item[{energySkip}] \leavevmode{[}Boolean{]}
if set to True, this species is ignored when calculating
line cooling rates

\item[{extrap}] \leavevmode{[}Boolean{]}
if set to True, collision rate coefficients for this species
will be extrapolated to temperatures outside the range given
in the LAMDA table. If False, no extrapolation is perfomed,
and providing temperatures outside the range in the table
produces an error

\end{description}

\item[{Returns}] \leavevmode
Nothing

\end{description}

\end{fulllineitems}

\index{chemabundances (despotic.zonedcloud attribute)}

\begin{fulllineitems}
\phantomsection\label{fulldoc:despotic.zonedcloud.chemabundances}\pysigline{\bfcode{chemabundances}}
Returns abundances of all species in the chemical network,
mass-weighted over the zonedcloud

\end{fulllineitems}

\index{chemabundances\_zone (despotic.zonedcloud attribute)}

\begin{fulllineitems}
\phantomsection\label{fulldoc:despotic.zonedcloud.chemabundances_zone}\pysigline{\bfcode{chemabundances\_zone}}
Return abundances of all emitting species in all zones

\end{fulllineitems}

\index{dEdt() (despotic.zonedcloud method)}

\begin{fulllineitems}
\phantomsection\label{fulldoc:despotic.zonedcloud.dEdt}\pysiglinewithargsret{\bfcode{dEdt}}{\emph{c1Grav=0.0}, \emph{thin=False}, \emph{LTE=False}, \emph{fixedLevPop=False}, \emph{noClump=False}, \emph{escapeProbGeom=None}, \emph{PsiUser=None}, \emph{sumOnly=False}, \emph{dustOnly=False}, \emph{gasOnly=False}, \emph{dustCoolOnly=False}, \emph{dampFactor=0.5}, \emph{verbose=False}, \emph{overrideSkip=False}, \emph{zones=False}}{}
Return instantaneous values of heating / cooling terms
\begin{description}
\item[{Parameters}] \leavevmode\begin{description}
\item[{c1Grav}] \leavevmode{[}float{]}
if this is non-zero, the cloud is assumed to be
collapsing, and energy is added at a rate
Gamma\_grav = c1 mu\_H m\_H cs\textasciicircum{}2 sqrt(4 pi G rho)

\item[{thin}] \leavevmode{[}Boolean{]}
if set to True, cloud is assumed to be opticall thin

\item[{LTE}] \leavevmode{[}Boolean{]}
if set to True, gas is assumed to be in LTE

\item[{fixedLevPop}] \leavevmode{[}Boolean{]}
if set to True, level populations and escape
probabilities are not recomputed, so the cooling rate is
based on whatever values are stored

\item[{escapeProbGeom}] \leavevmode{[}`sphere' \textbar{} `slab' \textbar{} `LVG'{]}
sets problem geometry that will be assumed in calculating
escape probabilities; ignored if the escape probabilities
are already initialized; if left as None, escapeProbGeom
= self.geometry

\item[{noClump}] \leavevmode{[}Boolean{]}
if set to True, the clumping factor used in estimating
rates for n\textasciicircum{}2 processes is set to unity

\item[{dampFactor}] \leavevmode{[}float{]}
damping factor to use in level population calculations;
see emitter.setLevPopEscapeProb

\item[{PsiUser}] \leavevmode{[}callable{]}
A user-specified function to add additional heating /
cooling terms to the calculation. The function takes the
cloud object as an argument, and must return a two-element
array Psi, where Psi{[}0{]} = gas heating / cooling rate,
Psi{[}1{]} = dust heating / cooling rate. Positive values
indicate heating, negative values cooling, and units are
assumed to be erg s\textasciicircum{}-1 H\textasciicircum{}-1.

\item[{sumOnly}] \leavevmode{[}Boolean{]}
if true, rates contains only four entries: dEdtGas and
dEdtDust give the heating / cooling rates for the
gas and dust summed over all terms, and maxAbsdEdtGas and
maxAbsdEdtDust give the largest of the absolute values of
any of the contributing terms for dust and gas

\item[{gasOnly}] \leavevmode{[}Boolean{]}
if true, the terms GammaISRF, GammaDustLine, LambdaDust,               and PsiUserDust are omitted from rates. If both gasOnly
and sumOnly are true, the dict contains only dEdtGas

\item[{dustOnly}] \leavevmode{[}Boolean{]}
if true, the terms GammaPE, GammaCR, LambdaLine,
GamamDLine, and PsiUserGas are omitted from rates. If both
dustOnly and sumOnly are true, the dict contains only
dEdtDust. Important caveat: the value of dEdtDust returned
in this case will not exactly match that returned if
dustOnly is false, because it will not contain the
contribution from gas line cooling radiation that is
absorbed by the dust

\item[{dustCoolOnly}] \leavevmode{[}Boolean{]}
as dustOnly, but except that now only the terms
LambdaDust, PsiGD, and PsiUserDust are computed

\item[{overrideSkip}] \leavevmode{[}Boolean{]}
if True, energySkip directives are ignored, and cooling
rates are calculated for all species

\item[{zones}] \leavevmode{[}Boolean{]}
if True, heating and cooling rates are returned for each
zone; if False, the values returned are mass-weighted over
the entire cloud

\end{description}

\item[{Returns}] \leavevmode\begin{description}
\item[{rates}] \leavevmode{[}dict{]}
A dict containing the values of the various heating and
cooling rate terms; all quantities are in units of erg s\textasciicircum{}-1
H\textasciicircum{}-1, and by convention positive = heating, negative =
cooling; for dust-gas exchange, positive indicates heating
of gas, cooling of dust. By default these quantities are
mass-weighted over the entire cloud, but if zones is True
then they are returned as arrays for each zone

\end{description}

Elements of the dict are as follows by default, but can be
altered by the additional keywords listed below:
\begin{description}
\item[{GammaPE}] \leavevmode{[}float or array{]}
photoelectric heating rate

\item[{GammaCR}] \leavevmode{[}float or array{]}
cosmic ray heating rate

\item[{GammaGrav}] \leavevmode{[}float or array{]}
gravitational contraction heating rate

\item[{LambdaLine}] \leavevmode{[}dict or array{]}
cooling rate from lines; dictionary keys correspond to
species in the emitter list, values give line cooling
rate for that species

\item[{PsiGD}] \leavevmode{[}float or array{]}
dust-gas energy exchange rate

\item[{GammaDustISRF}] \leavevmode{[}float or array{]}
dust heating rate due to the ISRF

\item[{GammaDustCMB}] \leavevmode{[}float or array{]}
dust heating rate due to CMB

\item[{GammaDustIR}] \leavevmode{[}float or array{]}
dust heating rate due to IR field

\item[{GammaDustLine}] \leavevmode{[}float or array{]}
dust heating rate due to absorption of line radiation

\item[{LambdaDust}] \leavevmode{[}float or array{]}
dust cooling rate due to thermal emission

\item[{PsiUserGas}] \leavevmode{[}float or array{]}
gas heating / cooling rate from user-specified
function; only included if PsiUser != None

\item[{PsiUserDust}] \leavevmode{[}float or array{]}
gas heating / cooling rate from user-specified
function; only included is PsiUser != None

\end{description}

\end{description}

\end{fulllineitems}

\index{lineLum() (despotic.zonedcloud method)}

\begin{fulllineitems}
\phantomsection\label{fulldoc:despotic.zonedcloud.lineLum}\pysiglinewithargsret{\bfcode{lineLum}}{\emph{emitName}, \emph{LTE=False}, \emph{noClump=False}, \emph{transition=None}, \emph{thin=False}, \emph{intOnly=False}, \emph{TBOnly=False}, \emph{lumOnly=False}, \emph{escapeProbGeom=None}, \emph{dampFactor=0.5}, \emph{noRecompute=False}, \emph{abstol=1e-08}, \emph{verbose=False}}{}
Return the frequency-integrated intensity of various lines,
summed over all zones
\begin{description}
\item[{Parameters}] \leavevmode\begin{description}
\item[{emitName}] \leavevmode{[}string{]}
name of the emitter for which the calculation is to be
performed

\item[{LTE}] \leavevmode{[}Boolean{]}
if True, and level populations are unitialized, they will
be initialized to their LTE values; if they are
initialized, this option is ignored

\item[{noClump}] \leavevmode{[}Boolean{]}
if set to True, the clumping factor used in estimating
rates for n\textasciicircum{}2 processes is set to unity

\item[{transition}] \leavevmode{[}list of two arrays{]}
if left as None, luminosity is computed for all
transitions; otherwise only selected transitions are
computed, with transition{[}0{]} = array of upper states
transition{[}1{]} = array of lower states

\item[{thin}] \leavevmode{[}Boolean{]}
if True, the calculation is done assuming the cloud is
optically thin; if level populations are uninitialized,
and LTE is not set, they will be computed assuming the
cloud is optically thin

\item[{intOnly: Boolean}] \leavevmode
if true, the output is simply an array containing the
frequency-integrated intensity of the specified lines;
mutually exclusive with TBOnly and lumOnly

\item[{TBOnly: Boolean}] \leavevmode
if True, the output is simply an array containing the
velocity-integrated brightness temperatures of the
specified lines; mutually exclusive with intOnly and
lumOnly

\item[{lumOnly: Boolean}] \leavevmode
if True, the output is simply an array containing the
luminosity per H nucleus in each of the specified lines;
mutually eclusive with intOnly and TBOonly

\item[{escapeProbGeom}] \leavevmode{[}`sphere' \textbar{} `slab' \textbar{} `LVG'{]}
sets problem geometry that will be assumed in calculating
escape probabilities; ignored if the escape probabilities
are already initialized; if left as None, escapeProbGeom
= self.geometry

\item[{dampFactor}] \leavevmode{[}float{]}
damping factor to use in level population calculations;
see emitter.setLevPopEscapeProb

\item[{noRecompute}] \leavevmode{[}False{]}
if True, level populations and escape probabilities are
not recomputed; instead, stored values are used

\end{description}

\item[{Returns}] \leavevmode
res : list or array

if intOnly, TBOnly, and lumOnly are all False, each element
of the list is a dict containing the following fields:
\begin{description}
\item[{`freq'}] \leavevmode{[}float{]}
frequency of the line in Hz

\item[{`upper'}] \leavevmode{[}int{]}
index of upper state, with ground state = 0 and states
ordered by energy

\item[{`lower'}] \leavevmode{[}int{]}
index of lower state

\item[{`Tupper'}] \leavevmode{[}float{]}
energy of the upper state in K (i.e. energy over kB)

\item[{`Tex'}] \leavevmode{[}float{]}
excitation temperature relating the upper and lower levels

\item[{`intIntensity'}] \leavevmode{[}float{]}
frequency-integrated intensity of the line, with the CMB
contribution subtracted off; units are erg cm\textasciicircum{}-2 s\textasciicircum{}-1 sr\textasciicircum{}-1

\item[{`intTB'}] \leavevmode{[}float{]}
velocity-integrated brightness temperature of the line,
with the CMB contribution subtracted off; units are K km
s\textasciicircum{}-1

\item[{`lumPerH'}] \leavevmode{[}float{]}
luminosity of the line per H nucleus; units are erg s\textasciicircum{}-1
H\textasciicircum{}-1

\item[{`tau'}] \leavevmode{[}float{]}
optical depth in the line, not including dust

\item[{`tauDust'}] \leavevmode{[}float{]}
dust optical depth in the line

\end{description}

\end{description}

if intOnly, TBOnly, or lumOnly are True: res is an array
containing the intIntensity, TB, or lumPerH fields of the dict
described above

\end{fulllineitems}

\index{mass() (despotic.zonedcloud method)}

\begin{fulllineitems}
\phantomsection\label{fulldoc:despotic.zonedcloud.mass}\pysiglinewithargsret{\bfcode{mass}}{\emph{edge=False}}{}
Returns the mass in each zone.
\begin{description}
\item[{Parameters}] \leavevmode\begin{description}
\item[{edge}] \leavevmode{[}Boolean{]}
if True, the value returned gives the cumulative mass at
each zone edge, starting from the outer edge; otherwise
the value returned is the mass of each zone

\end{description}

\item[{Returns}] \leavevmode\begin{description}
\item[{mass}] \leavevmode{[}array{]}
mass of each zone (if edge is False), or cumulative mass
to each zone edge (if edge is True)

\end{description}

\item[{Remarks}] \leavevmode
if the geometry is `slab', the masses are undefined, and
this return returns None

\end{description}

\end{fulllineitems}

\index{radius() (despotic.zonedcloud method)}

\begin{fulllineitems}
\phantomsection\label{fulldoc:despotic.zonedcloud.radius}\pysiglinewithargsret{\bfcode{radius}}{\emph{edge=False}}{}
Return the radius of each zone.
\begin{description}
\item[{Parameters}] \leavevmode\begin{description}
\item[{edge}] \leavevmode{[}Boolean{]}
if True, the value returned is the radii of the zone
edges; otherwise it is the radii of the zone centers

\end{description}

\item[{Returns}] \leavevmode\begin{description}
\item[{rad}] \leavevmode{[}array{]}
radii of zone centers (default) or edges (if edge is
True)

\end{description}

\item[{Remarks}] \leavevmode
if the geometry is `slab', the values returned are the
depths into the slab rather than the radii

\end{description}

\end{fulllineitems}

\index{setChemEq() (despotic.zonedcloud method)}

\begin{fulllineitems}
\phantomsection\label{fulldoc:despotic.zonedcloud.setChemEq}\pysiglinewithargsret{\bfcode{setChemEq}}{\emph{**kwargs}}{}
Set the chemical abundances for a cloud to their equilibrium
values, computed using a specified chemical network.
\begin{description}
\item[{Parameters}] \leavevmode\begin{description}
\item[{kwargs}] \leavevmode{[}dict{]}
these arguments are passed through to the corresponding
function for each zone

\end{description}

\item[{Returns}] \leavevmode\begin{description}
\item[{success}] \leavevmode{[}Boolean{]}
True if the calculation converges, False if it does not

\end{description}

\item[{Remarks}] \leavevmode
if the key escapeProbGeom is not in kwargs, then
escapeProbGeom will be set to self.geometry

\end{description}

\end{fulllineitems}

\index{setDustTempEq() (despotic.zonedcloud method)}

\begin{fulllineitems}
\phantomsection\label{fulldoc:despotic.zonedcloud.setDustTempEq}\pysiglinewithargsret{\bfcode{setDustTempEq}}{\emph{**kwargs}}{}
Set Td to equilibrium dust temperature at fixed Tg
\begin{description}
\item[{Parameters}] \leavevmode\begin{description}
\item[{kwargs}] \leavevmode{[}dict{]}
these arguments are passed through to the corresponding
function for each zone

\end{description}

\item[{Returns}] \leavevmode\begin{description}
\item[{success}] \leavevmode{[}Boolean{]}
True if dust temperature calculation converged, False if
not

\end{description}

\end{description}

\end{fulllineitems}

\index{setGasTempEq() (despotic.zonedcloud method)}

\begin{fulllineitems}
\phantomsection\label{fulldoc:despotic.zonedcloud.setGasTempEq}\pysiglinewithargsret{\bfcode{setGasTempEq}}{\emph{**kwargs}}{}
Set Tg to equilibrium gas temperature at fixed Td
\begin{description}
\item[{Parameters}] \leavevmode\begin{description}
\item[{kwargs}] \leavevmode{[}dict{]}
these arguments are passed through to the corresponding
function for each zone

\end{description}

\item[{Returns}] \leavevmode\begin{description}
\item[{success}] \leavevmode{[}Boolean{]}
True if the calculation converges, False if it does not

\end{description}

\item[{Remarks}] \leavevmode
if the key escapeProbGeom is not in kwargs, then
escapeProbGeom will be set to self.geometry

\end{description}

\end{fulllineitems}

\index{setTempEq() (despotic.zonedcloud method)}

\begin{fulllineitems}
\phantomsection\label{fulldoc:despotic.zonedcloud.setTempEq}\pysiglinewithargsret{\bfcode{setTempEq}}{\emph{**kwargs}}{}
Set Tg and Td to equilibrium gas and dust temperatures
\begin{description}
\item[{Parameters}] \leavevmode\begin{description}
\item[{kwargs}] \leavevmode{[}dict{]}
these arguments are passed through to the corresponding
function for each zone

\end{description}

\item[{Returns}] \leavevmode\begin{description}
\item[{success}] \leavevmode{[}Boolean{]}
True if the calculation converges, False if it does not

\end{description}

\item[{Remarks}] \leavevmode
if the key escapeProbGeom is not in kwargs, then
escapeProbGeom will be set to self.geometry

\end{description}

\end{fulllineitems}

\index{setVirial() (despotic.zonedcloud method)}

\begin{fulllineitems}
\phantomsection\label{fulldoc:despotic.zonedcloud.setVirial}\pysiglinewithargsret{\bfcode{setVirial}}{\emph{alphaVir=1.0}, \emph{NTonly=False}}{}
This routine sets the velocity dispersion in all zones to the
virial value
\begin{description}
\item[{Parameters}] \leavevmode\begin{description}
\item[{alphaVir}] \leavevmode{[}float{]}
virial ratio to be used in computation; defaults to 1

\item[{NTonly}] \leavevmode{[}Boolean{]}
if True, the virial ratio is computed considering only the
non-thermal component of the velocity dispersion

\end{description}

\item[{Returns}] \leavevmode
Nothing

\end{description}

\end{fulllineitems}


\end{fulllineitems}



\section{despotic functions}
\label{fulldoc:despotic-functions}

\subsection{\texttt{fetchLamda}}
\label{fulldoc:fetchlamda}\index{fetchLamda() (in module despotic)}

\begin{fulllineitems}
\phantomsection\label{fulldoc:despotic.fetchLamda}\pysiglinewithargsret{\code{despotic.}\bfcode{fetchLamda}}{\emph{inputURL}, \emph{path=None}, \emph{fileName=None}}{}
Routine to download LAMDA files from the web.
\begin{description}
\item[{Parameters}] \leavevmode\begin{description}
\item[{inputURL}] \leavevmode{[}string{]}
URL of LAMDA file containing data on this species; if this
does not begin with ``\url{http://}'', indicating it is a URL, then
this is assumed to be a filename within LAMDA, and a default
URL is appended

\item[{path}] \leavevmode{[}string{]}
relative or absolute path at which to store the file; if not
set, the current directory is used; if the specified path does
not exist, it is created

\item[{fileName}] \leavevmode{[}string{]}
name to give to file; if not set, defaults to the same as the
name in LAMDA

\end{description}

\item[{Returns}] \leavevmode\begin{description}
\item[{fname}] \leavevmode{[}string{]}
local file name to which downloaded file was written; if URL
cannot be opened, None is returned instead

\end{description}

\end{description}

\end{fulllineitems}



\subsection{\texttt{lineProfLTE}}
\label{fulldoc:sssec-full-lineproflte}\label{fulldoc:lineproflte}\index{lineProfLTE() (in module despotic)}

\begin{fulllineitems}
\phantomsection\label{fulldoc:despotic.lineProfLTE}\pysiglinewithargsret{\code{despotic.}\bfcode{lineProfLTE}}{\emph{emdat}, \emph{u}, \emph{l}, \emph{R}, \emph{denProf}, \emph{TProf}, \emph{vProf=0.0}, \emph{sigmaProf=0.0}, \emph{offset=0.0}, \emph{TCMB=2.73}, \emph{vOut=None}, \emph{vLim=None}, \emph{nOut=100}, \emph{dv=None}, \emph{mxstep=10000}, \emph{beamdisp=0.0}}{}
Return the brightness temperature versus velocity for a
specified line, assuming the level populations are in LTE. The
calculation is done along an infinitely thin pencil beam.
\begin{description}
\item[{Parameters}] \leavevmode\begin{description}
\item[{em}] \leavevmode{[}class emitterData{]}
emitterData object describing the emitting species for
which the computation is to be made

\item[{u}] \leavevmode{[}int{]}
upper state of line to be computed

\item[{l}] \leavevmode{[}int{]}
lower state of line to be computed

\item[{R}] \leavevmode{[}float{]}
cloud radius in cm

\item[{denProf}] \leavevmode{[}float or callable{]}
If denProf is a float, this give the density in particles
cm\textasciicircum{}-3 of the emitting species, which is taken to be
uniform. denProf can also be a function giving the density
as a function of radius; see remarks below for details.

\item[{TProf}] \leavevmode{[}float \textbar{} callable{]}
same as denProf, but giving the temperature in K

\item[{vProf}] \leavevmode{[}float \textbar{} callable{]}
same as vProf, but giving the bulk radial
velocity in cm/s; if omitted, bulk velocity is set to 0

\item[{sigmaProf}] \leavevmode{[}float \textbar{} callable{]}
same as denProf, but giving the non-thermal
velocity dispersion in cm/s; if omitted, non-thermal
velocity dispersion is set to 0

\item[{offset}] \leavevmode{[}float{]}
fractional distance from cloud center at which
measurement is made; 0 = at cloud center, 1 = at
cloud edge; valid values are 0 - 1

\item[{vOut}] \leavevmode{[}sequence{]}
sequence of velocities (relative to line center at 0) at
which the output is to be returned

\item[{vLim}] \leavevmode{[}sequence (2){]}
maximum and minimum velocities relative to line center at
which to compute TB

\item[{nOut}] \leavevmode{[}int{]}
number of velocities at which to output

\item[{dv}] \leavevmode{[}float{]}
velocity spacing at which to produce output

\item[{TCMB}] \leavevmode{[}float{]}
CMB temperature used as a background to the cloud, in K

\item[{mxstep}] \leavevmode{[}int{]}
maximum number of steps in the ODE solver

\item[{beamdisp}] \leavevmode{[}float{]}
the dispersion of the Gaussian beam, measured in units
of the cloud radius; a value of 0 causes the integration do
be done alone a perfect pencil beam; not currently compatible
with offset != 0

\end{description}

\item[{Returns}] \leavevmode\begin{description}
\item[{TB}] \leavevmode{[}array{]}
brightness temperature as a function of velocity (in K)

\item[{vOut}] \leavevmode{[}array{]}
velocities at which TB is computed (in cm s\textasciicircum{}-1)

\end{description}

\item[{Raises}] \leavevmode
despoticError is the specified upper and lower state have no
radiative transition between them, or if offset is not in the
range 0 - 1

\item[{Remarks}] \leavevmode
The functions denProf, TProf, vProf, and sigmaProf, if
specified, should accept one floating argument, and return one
floating value. The argument r is the radial position within
the cloud in normalized units, so that the center is at r = 0
and the edge at r = 1. The return value should be the density,
temperature, velocity, or non-thermal velocity dispesion at
that position, in cgs units.

\end{description}

\end{fulllineitems}



\subsection{\texttt{refreshLamda}}
\label{fulldoc:refreshlamda}\label{fulldoc:sssec-full-refreshlamda}\index{refreshLamda() (in module despotic)}

\begin{fulllineitems}
\phantomsection\label{fulldoc:despotic.refreshLamda}\pysiglinewithargsret{\code{despotic.}\bfcode{refreshLamda}}{\emph{path=None}, \emph{cutoffDate=None}, \emph{cutoffAge=None}, \emph{LamdaURL=None}}{}
Refreshes LAMDA files by fetching new ones from the web
\begin{description}
\item[{Parameters}] \leavevmode\begin{description}
\item[{path}] \leavevmode{[}string{]}
path to the local LAMDA database; defaults to getting this
information from the environment variable DESPOTIC\_HOME

\item[{cutoffDate}] \leavevmode{[}class datetime.date or class datetime.datetime{]}
a date or datetime specifying the age cutoff for updating
files; files older than cutoffDate are updated, newer ones are
not

\item[{cutoffAge}] \leavevmode{[}class datetime.timedelta{]}
a duration between the present instant and the point in the
past separating files that will be updated from files that
will not be

\item[{LamdaURL}] \leavevmode{[}string{]}
URL where LAMDA is located; defaults to the default value in
fetchLamda

\end{description}

\item[{Returns}] \leavevmode
Nothing

\item[{Remarks}] \leavevmode
If the user sets both a cutoff age and a cutoff date, the date is
used. If neither is set, the default cutoff age is 6 months.

\end{description}

\end{fulllineitems}



\section{despotic.chemistry classes}
\label{fulldoc:despotic-chemistry-classes}

\subsection{\texttt{abundanceDict}}
\label{fulldoc:sssec-full-abundancedict}\label{fulldoc:abundancedict}\index{abundanceDict (class in despotic.chemistry)}

\begin{fulllineitems}
\phantomsection\label{fulldoc:despotic.chemistry.abundanceDict}\pysiglinewithargsret{\strong{class }\code{despotic.chemistry.}\bfcode{abundanceDict}}{\emph{specList}, \emph{x}}{}
An abundanceDict object is a wrapper around a numpy array of
abundances, and maps between human-readable chemical names
(e.g. CO) and numeric indices in the array. Elements can be
queried and addressed using a dict-like interface (e.g. if abd is
an abundanceDict object, one could do abd{[}'CO'{]} = 1.0e-4), but the
underlying data structure can be manipulated with the speed and
flexibility of a numpy array. In particular, one can perform
arithmetic operations such as addition on abundance dicts, and they
are simply applied to the underyling numpy array using the usual
numpy operator rules.

This mapping between species names and array indices is created
when the dict is first initialized, and is immutable
thereafter. Thus operations that modify the keys in a dict are
disallowed for abundanceDict objects.
\begin{description}
\item[{Parameters}] \leavevmode\begin{description}
\item[{specList}] \leavevmode{[}list{]}
list of species names for this abundanceDict; each list
element must be a string

\item[{x}] \leavevmode{[}array of rank 1 or 2{]}
array of abundances; the length of the first dimension of
x must be equal to the length of specList

\end{description}

\end{description}
\index{clear() (despotic.chemistry.abundanceDict method)}

\begin{fulllineitems}
\phantomsection\label{fulldoc:despotic.chemistry.abundanceDict.clear}\pysiglinewithargsret{\bfcode{clear}}{}{}
raises an error, since abundanceDicts are
immutable

\end{fulllineitems}

\index{copy() (despotic.chemistry.abundanceDict method)}

\begin{fulllineitems}
\phantomsection\label{fulldoc:despotic.chemistry.abundanceDict.copy}\pysiglinewithargsret{\bfcode{copy}}{}{}
copy works just as for an ordinary dict

\end{fulllineitems}

\index{has\_key() (despotic.chemistry.abundanceDict method)}

\begin{fulllineitems}
\phantomsection\label{fulldoc:despotic.chemistry.abundanceDict.has_key}\pysiglinewithargsret{\bfcode{has\_key}}{\emph{key}}{}
has\_key works just as for an ordinary dict

\end{fulllineitems}

\index{index() (despotic.chemistry.abundanceDict method)}

\begin{fulllineitems}
\phantomsection\label{fulldoc:despotic.chemistry.abundanceDict.index}\pysiglinewithargsret{\bfcode{index}}{\emph{spec}}{}~\begin{description}
\item[{Parameters}] \leavevmode\begin{description}
\item[{spec}] \leavevmode{[}string \textbar{} iterable{]}
if this is a string, the method returns the index of
that chemical species; if it is an iterable, the
iterable must contain strings, and  the method
returns an array containing the indices of all species in
the iterable

\end{description}

\item[{Returns}] \leavevmode\begin{description}
\item[{index}] \leavevmode{[}int \textbar{} array{]}
indices of the input species; if spec is a string, this is
an int; otherwise it is an array of ints

\end{description}

\item[{Raises}] \leavevmode
KeyError, if spec or any of its elements is not in the species
list

\end{description}

\end{fulllineitems}

\index{keys() (despotic.chemistry.abundanceDict method)}

\begin{fulllineitems}
\phantomsection\label{fulldoc:despotic.chemistry.abundanceDict.keys}\pysiglinewithargsret{\bfcode{keys}}{}{}
keys works just as for an ordinary dict

\end{fulllineitems}

\index{pop() (despotic.chemistry.abundanceDict method)}

\begin{fulllineitems}
\phantomsection\label{fulldoc:despotic.chemistry.abundanceDict.pop}\pysiglinewithargsret{\bfcode{pop}}{\emph{key}}{}
raises an error, since abundanceDicts are
immutable

\end{fulllineitems}

\index{popitem() (despotic.chemistry.abundanceDict method)}

\begin{fulllineitems}
\phantomsection\label{fulldoc:despotic.chemistry.abundanceDict.popitem}\pysiglinewithargsret{\bfcode{popitem}}{}{}
raises an error, since abundanceDicts are
immutable

\end{fulllineitems}

\index{values() (despotic.chemistry.abundanceDict method)}

\begin{fulllineitems}
\phantomsection\label{fulldoc:despotic.chemistry.abundanceDict.values}\pysiglinewithargsret{\bfcode{values}}{}{}
values returns a list of numpy arrays corresponding to the
rows of x

\end{fulllineitems}


\end{fulllineitems}



\subsection{\texttt{chemNetwork}}
\label{fulldoc:chemnetwork}\index{chemNetwork (class in despotic.chemistry)}

\begin{fulllineitems}
\phantomsection\label{fulldoc:despotic.chemistry.chemNetwork}\pysiglinewithargsret{\strong{class }\code{despotic.chemistry.}\bfcode{chemNetwork}}{\emph{cloud=None}, \emph{info=None}}{}
This is a purely abstract class that defines the methods that all
chemistry networks are required to implement. Chemistry networks
should be derived from it, and should override its
methods. Attempting to instantiate this directly will lead to an
error.
\begin{description}
\item[{Parameters}] \leavevmode\begin{description}
\item[{cloud}] \leavevmode{[}class cloud{]}
a cloud object to which this network should be attached

\item[{info}] \leavevmode{[}dict{]}
a dict of additional information to be passed to the network
on instantiation

\end{description}

\item[{Class attributes}] \leavevmode\begin{description}
\item[{specList}] \leavevmode{[}list{]}
list of strings giving the names of the species being
treated in the chemical network

\item[{x}] \leavevmode{[}array{]}
array of abundances of the species in specList

\item[{cloud}] \leavevmode{[}class cloud{]}
a cloud object to which this chemical network is attached;
can be None

\end{description}

\end{description}
\index{abundances (despotic.chemistry.chemNetwork attribute)}

\begin{fulllineitems}
\phantomsection\label{fulldoc:despotic.chemistry.chemNetwork.abundances}\pysigline{\bfcode{abundances}}
The current abundances of every species in the chemical
network, stored as an abundanceDict.

\end{fulllineitems}

\index{applyAbundances() (despotic.chemistry.chemNetwork method)}

\begin{fulllineitems}
\phantomsection\label{fulldoc:despotic.chemistry.chemNetwork.applyAbundances}\pysiglinewithargsret{\bfcode{applyAbundances}}{\emph{addEmitters=False}}{}
This method writes abundances from the chemical network back
to the cloud to which this network is attached.
\begin{description}
\item[{Parameters}] \leavevmode\begin{description}
\item[{addEmitters}] \leavevmode{[}bool{]}
if True, and the network contains emitters that are not
part of the parent cloud, then the network will attempt
to add them using cloud.addEmitter. Otherwise this
routine will change the abundances of whatever emitters
are already attached to the cloud, but will not add new
ones.

\end{description}

\item[{Returns:}] \leavevmode
Nothing

\end{description}

\end{fulllineitems}

\index{dxdt() (despotic.chemistry.chemNetwork method)}

\begin{fulllineitems}
\phantomsection\label{fulldoc:despotic.chemistry.chemNetwork.dxdt}\pysiglinewithargsret{\bfcode{dxdt}}{\emph{xin}, \emph{time}}{}
This routine returns the time rate of change of the abundances
for all species in the network.
\begin{description}
\item[{Parameters}] \leavevmode\begin{description}
\item[{xin}] \leavevmode{[}array{]}
array of starting abundances

\item[{time}] \leavevmode{[}float{]}
current time in sec

\end{description}

\item[{Returns}] \leavevmode\begin{description}
\item[{dxdt}] \leavevmode{[}array{]}
the time derivative of all species abundances

\end{description}

\end{description}

\end{fulllineitems}


\end{fulllineitems}



\subsection{\texttt{cr\_reactions}}
\label{fulldoc:cr-reactions}\index{cr\_reactions (class in despotic.chemistry)}

\begin{fulllineitems}
\phantomsection\label{fulldoc:despotic.chemistry.cr_reactions}\pysiglinewithargsret{\strong{class }\code{despotic.chemistry.}\bfcode{cr\_reactions}}{\emph{specList}, \emph{reactions}, \emph{sparse=False}}{}
The cr\_reactions class is used to handle computation of
cosmic ray-induced reaction rates. In addition to the constructor,
the class has only a single method: dxdt, which returns the
reaction rates.
\begin{description}
\item[{Parameters}] \leavevmode\begin{description}
\item[{specList: listlike of string}] \leavevmode
List of all chemical species in the full reaction network,
including those that are derived from conservation laws
instead of being computed directly

\item[{reactions}] \leavevmode{[}list of dict{]}
A list listing all the reactions to be registered; each
entry in the list must be a dict containing the keys `spec'
`stoich', and `rate', which list the species involved in the
reaction, the stoichiometric factors for each species, and
the reaction rate per primary CR ionization,
respectively. Sign convention is that reactants on the left 
hand side have negative stoichiometric factors, those on the
right hand side have positive factors.

\item[{sparse}] \leavevmode{[}bool{]}
If True, the reaction rate matrix is represented as a
sparse matrix; otherwise it is a dense matrix. This has no
effect on results, but depending on the chemical network it
may lead to improved execution speed and/or reduced memory
usage.

\end{description}

\item[{Examples}] \leavevmode
To list the reaction cr + H -\textgreater{} H+ + e-, the dict entry should be:

\begin{Verbatim}[commandchars=\\\{\}]
\PYG{p}{\PYGZob{}} \PYG{l+s+s1}{\PYGZsq{}}\PYG{l+s+s1}{spec}\PYG{l+s+s1}{\PYGZsq{}} \PYG{p}{:} \PYG{p}{[}\PYG{l+s+s1}{\PYGZsq{}}\PYG{l+s+s1}{H}\PYG{l+s+s1}{\PYGZsq{}}\PYG{p}{,} \PYG{l+s+s1}{\PYGZsq{}}\PYG{l+s+s1}{H+}\PYG{l+s+s1}{\PYGZsq{}}\PYG{p}{,} \PYG{l+s+s1}{\PYGZsq{}}\PYG{l+s+s1}{e\PYGZhy{}}\PYG{l+s+s1}{\PYGZsq{}}\PYG{p}{]}\PYG{p}{,} \PYG{l+s+s1}{\PYGZsq{}}\PYG{l+s+s1}{stoich}\PYG{l+s+s1}{\PYGZsq{}} \PYG{p}{:} \PYG{p}{[}\PYG{o}{\PYGZhy{}}\PYG{l+m+mi}{1}\PYG{p}{,} \PYG{l+m+mi}{1}\PYG{p}{,} \PYG{l+m+mi}{1}\PYG{p}{]}\PYG{p}{,}
  \PYG{l+s+s1}{\PYGZsq{}}\PYG{l+s+s1}{rate}\PYG{l+s+s1}{\PYGZsq{}} \PYG{p}{:} \PYG{l+m+mf}{1.0} \PYG{p}{\PYGZcb{}}
\end{Verbatim}

\end{description}
\index{dxdt() (despotic.chemistry.cr\_reactions method)}

\begin{fulllineitems}
\phantomsection\label{fulldoc:despotic.chemistry.cr_reactions.dxdt}\pysiglinewithargsret{\bfcode{dxdt}}{\emph{x}, \emph{n}, \emph{ionrate}}{}
This function returns the time derivative of the abundances x
for a given cosmic ray ionization rate.
\begin{description}
\item[{Parameters}] \leavevmode\begin{description}
\item[{x}] \leavevmode{[}array(N){]}
array of current species abundances

\item[{n}] \leavevmode{[}float{]}
number density of H nuclei; only used if some reactions
have multiple species on the LHS, otherwise can be set to
any positive value

\item[{ionrate}] \leavevmode{[}float{]}
cosmic ray primary ionization rate

\end{description}

\item[{Returns}] \leavevmode\begin{description}
\item[{dxdt: array(N)}] \leavevmode
rate of change of all species abundances

\end{description}

\end{description}

\end{fulllineitems}


\end{fulllineitems}



\subsection{\texttt{NL99}}
\label{fulldoc:nl99}\index{NL99 (class in despotic.chemistry)}

\begin{fulllineitems}
\phantomsection\label{fulldoc:despotic.chemistry.NL99}\pysiglinewithargsret{\strong{class }\code{despotic.chemistry.}\bfcode{NL99}}{\emph{cloud=None}, \emph{info=None}}{}
This class the implements the chemistry network of Nelson \& Langer
(1999, ApJ, 524, 923).
\begin{description}
\item[{Parameters}] \leavevmode\begin{description}
\item[{cloud}] \leavevmode{[}class cloud{]}
a DESPOTIC cloud object from which initial data are to be
taken

\item[{info}] \leavevmode{[}dict{]}
a dict containing additional parameters

\end{description}

\item[{Remarks}] \leavevmode
The dict info may contain the following key - value pairs:
\begin{description}
\item[{`xC'}] \leavevmode{[}float{]}
the total C abundance per H nucleus; defaults to 2.0e-4

\item[{`xO'}] \leavevmode{[}float{]}
the total H abundance per H nucleus; defaults to 4.0e-4

\item[{`xM'}] \leavevmode{[}float{]}
the total refractory metal abundance per H
nucleus; defaults to 2.0e-7

\item[{`sigmaDustV'}] \leavevmode{[}float{]}
V band dust extinction cross
section per H nucleus; if not set, the default behavior
is to assume that sigmaDustV = 0.4 * cloud.dust.sigmaPE

\item[{`AV'}] \leavevmode{[}float{]}
total visual extinction; ignored if sigmaDustV is set

\item[{`noClump'}] \leavevmode{[}bool{]}
if True, the clumping factor is set to 1.0; defaults to False

\end{description}

\end{description}
\index{AV (despotic.chemistry.NL99 attribute)}

\begin{fulllineitems}
\phantomsection\label{fulldoc:despotic.chemistry.NL99.AV}\pysigline{\bfcode{AV}}
visual extinction in mag

\end{fulllineitems}

\index{NH (despotic.chemistry.NL99 attribute)}

\begin{fulllineitems}
\phantomsection\label{fulldoc:despotic.chemistry.NL99.NH}\pysigline{\bfcode{NH}}
column density of H nuclei

\end{fulllineitems}

\index{abundances (despotic.chemistry.NL99 attribute)}

\begin{fulllineitems}
\phantomsection\label{fulldoc:despotic.chemistry.NL99.abundances}\pysigline{\bfcode{abundances}}
abundances of all species in the chemical network

\end{fulllineitems}

\index{applyAbundances() (despotic.chemistry.NL99 method)}

\begin{fulllineitems}
\phantomsection\label{fulldoc:despotic.chemistry.NL99.applyAbundances}\pysiglinewithargsret{\bfcode{applyAbundances}}{\emph{addEmitters=False}}{}
This method writes the abundances produced by the chemical
network to the cloud's emitter list.
\begin{description}
\item[{Parameters}] \leavevmode\begin{description}
\item[{addEmitters}] \leavevmode{[}Boolean{]}
if True, emitters that are included in the chemical
network but not in the cloud's existing emitter list will
be added; if False, abundances of emitters already in the
emitter list will be updated, but new emiters will not be
added to the cloud

\end{description}

\item[{Returns}] \leavevmode
Nothing

\item[{Remarks}] \leavevmode
If there is no cloud associated with this chemical network,
this routine does nothing and silently returns.

\end{description}

\end{fulllineitems}

\index{cfac (despotic.chemistry.NL99 attribute)}

\begin{fulllineitems}
\phantomsection\label{fulldoc:despotic.chemistry.NL99.cfac}\pysigline{\bfcode{cfac}}
clumping factor; cannot be set directly, calculated from temp
and sigmaNT

\end{fulllineitems}

\index{chi (despotic.chemistry.NL99 attribute)}

\begin{fulllineitems}
\phantomsection\label{fulldoc:despotic.chemistry.NL99.chi}\pysigline{\bfcode{chi}}
ISRF strength, normalized to solar neighborhood value

\end{fulllineitems}

\index{dxdt() (despotic.chemistry.NL99 method)}

\begin{fulllineitems}
\phantomsection\label{fulldoc:despotic.chemistry.NL99.dxdt}\pysiglinewithargsret{\bfcode{dxdt}}{\emph{xin}, \emph{time}}{}
This method returns the time derivative of all abundances for
this chemical network.
\begin{description}
\item[{Parameters}] \leavevmode\begin{description}
\item[{xin}] \leavevmode{[}array(10){]}
current abundances of all species

\item[{time}] \leavevmode{[}float{]}
current time; not actually used, but included as an
argument for compatibility with odeint

\end{description}

\item[{Returns}] \leavevmode\begin{description}
\item[{dxdt}] \leavevmode{[}array(10){]}
time derivative of x

\end{description}

\end{description}

\end{fulllineitems}

\index{extendAbundances() (despotic.chemistry.NL99 method)}

\begin{fulllineitems}
\phantomsection\label{fulldoc:despotic.chemistry.NL99.extendAbundances}\pysiglinewithargsret{\bfcode{extendAbundances}}{\emph{xin=None}}{}
Compute abundances of derived species not directly followed in
the network.
\begin{description}
\item[{Parameters}] \leavevmode\begin{description}
\item[{xin}] \leavevmode{[}array{]}
abundances of species directly tracked in the network;
if left as None, the abundances stored internally to the
network are used

\end{description}

\item[{Returns}] \leavevmode\begin{description}
\item[{x}] \leavevmode{[}array{]}
abundances, including those of derived species

\end{description}

\end{description}

\end{fulllineitems}

\index{ionRate (despotic.chemistry.NL99 attribute)}

\begin{fulllineitems}
\phantomsection\label{fulldoc:despotic.chemistry.NL99.ionRate}\pysigline{\bfcode{ionRate}}
primary ionization rate from cosmic rays and x-rays

\end{fulllineitems}

\index{nH (despotic.chemistry.NL99 attribute)}

\begin{fulllineitems}
\phantomsection\label{fulldoc:despotic.chemistry.NL99.nH}\pysigline{\bfcode{nH}}
volume density of H nuclei

\end{fulllineitems}

\index{sigmaNT (despotic.chemistry.NL99 attribute)}

\begin{fulllineitems}
\phantomsection\label{fulldoc:despotic.chemistry.NL99.sigmaNT}\pysigline{\bfcode{sigmaNT}}
non-thermal velocity dispersion

\end{fulllineitems}

\index{temp (despotic.chemistry.NL99 attribute)}

\begin{fulllineitems}
\phantomsection\label{fulldoc:despotic.chemistry.NL99.temp}\pysigline{\bfcode{temp}}
gas kinetic temperature

\end{fulllineitems}

\index{xHe (despotic.chemistry.NL99 attribute)}

\begin{fulllineitems}
\phantomsection\label{fulldoc:despotic.chemistry.NL99.xHe}\pysigline{\bfcode{xHe}}
He abundance

\end{fulllineitems}


\end{fulllineitems}



\subsection{\texttt{NL99\_GC}}
\label{fulldoc:nl99-gc}\index{NL99\_GC (class in despotic.chemistry)}

\begin{fulllineitems}
\phantomsection\label{fulldoc:despotic.chemistry.NL99_GC}\pysiglinewithargsret{\strong{class }\code{despotic.chemistry.}\bfcode{NL99\_GC}}{\emph{cloud=None}, \emph{info=None}}{}
This class the implements the CO chemistry network of Nelson \& Langer
(1999, ApJ, 524, 923) coupled to the H2 chemistry network of
Glover \& MacLow (2007, ApJS, 169, 239), as combined by Glover \&
Clark (2012, MNRAS, 421, 9).
\begin{description}
\item[{Parameters}] \leavevmode\begin{description}
\item[{cloud}] \leavevmode{[}class cloud{]}
a DESPOTIC cloud object from which initial data are to be
taken

\item[{info}] \leavevmode{[}dict{]}
a dict containing additional parameters

\end{description}

\item[{Remarks}] \leavevmode
The dict info may contain the following key - value pairs:
\begin{description}
\item[{`xC'}] \leavevmode{[}float{]}
the total C abundance per H nucleus; defaults to 2.0e-4

\item[{`xO'}] \leavevmode{[}float{]}
the total H abundance per H nucleus; defaults to 4.0e-4

\item[{`xM'}] \leavevmode{[}float{]}
the total refractory metal abundance per H
nucleus; defaults to 2.0e-7

\item[{`Zd'}] \leavevmode{[}float{]}
dust abundance in solar units; defaults to 1.0

\item[{`sigmaDustV'}] \leavevmode{[}float{]}
V band dust extinction cross
section per H nucleus; if not set, the default behavior
is to assume that sigmaDustV = 0.4 * cloud.dust.sigmaPE

\item[{`AV'}] \leavevmode{[}float{]}
total visual extinction; ignored if sigmaDustV is set

\item[{`noClump'}] \leavevmode{[}bool{]}
if True, the clumping factor is set to 1.0; defaults to False

\item[{`sigmaNT'}] \leavevmode{[}float{]}
non-thermal velocity dispersion

\item[{`temp'}] \leavevmode{[}float{]}
gas kinetic temperature

\end{description}

\end{description}
\index{AV (despotic.chemistry.NL99\_GC attribute)}

\begin{fulllineitems}
\phantomsection\label{fulldoc:despotic.chemistry.NL99_GC.AV}\pysigline{\bfcode{AV}}
visual extinction in mag

\end{fulllineitems}

\index{NH (despotic.chemistry.NL99\_GC attribute)}

\begin{fulllineitems}
\phantomsection\label{fulldoc:despotic.chemistry.NL99_GC.NH}\pysigline{\bfcode{NH}}
column density of H nuclei

\end{fulllineitems}

\index{Zd (despotic.chemistry.NL99\_GC attribute)}

\begin{fulllineitems}
\phantomsection\label{fulldoc:despotic.chemistry.NL99_GC.Zd}\pysigline{\bfcode{Zd}}
dust abundance normalized to the solar neighborhood value

\end{fulllineitems}

\index{abundances (despotic.chemistry.NL99\_GC attribute)}

\begin{fulllineitems}
\phantomsection\label{fulldoc:despotic.chemistry.NL99_GC.abundances}\pysigline{\bfcode{abundances}}
abundances of all species in the chemical network

\end{fulllineitems}

\index{applyAbundances() (despotic.chemistry.NL99\_GC method)}

\begin{fulllineitems}
\phantomsection\label{fulldoc:despotic.chemistry.NL99_GC.applyAbundances}\pysiglinewithargsret{\bfcode{applyAbundances}}{\emph{addEmitters=False}}{}
This method writes the abundances produced by the chemical
network to the cloud's emitter list.
\begin{description}
\item[{Parameters}] \leavevmode\begin{description}
\item[{addEmitters}] \leavevmode{[}Boolean{]}
if True, emitters that are included in the chemical
network but not in the cloud's existing emitter list will
be added; if False, abundances of emitters already in the
emitter list will be updated, but new emiters will not be
added to the cloud

\end{description}

\item[{Returns}] \leavevmode
Nothing

\item[{Remarks}] \leavevmode
If there is no cloud associated with this chemical network,
this routine does nothing and silently returns.

\end{description}

\end{fulllineitems}

\index{cfac (despotic.chemistry.NL99\_GC attribute)}

\begin{fulllineitems}
\phantomsection\label{fulldoc:despotic.chemistry.NL99_GC.cfac}\pysigline{\bfcode{cfac}}
clumping factor; cannot be set directly, calculated from temp
and sigmaNT

\end{fulllineitems}

\index{chi (despotic.chemistry.NL99\_GC attribute)}

\begin{fulllineitems}
\phantomsection\label{fulldoc:despotic.chemistry.NL99_GC.chi}\pysigline{\bfcode{chi}}
ISRF strength, normalized to solar neighborhood value

\end{fulllineitems}

\index{dxdt() (despotic.chemistry.NL99\_GC method)}

\begin{fulllineitems}
\phantomsection\label{fulldoc:despotic.chemistry.NL99_GC.dxdt}\pysiglinewithargsret{\bfcode{dxdt}}{\emph{xin}, \emph{time}}{}
This method returns the time derivative of all abundances for
this chemical network.
\begin{description}
\item[{Parameters}] \leavevmode\begin{description}
\item[{xin}] \leavevmode{[}array(12){]}
current abundances of all species

\item[{time}] \leavevmode{[}float{]}
current time; not actually used, but included as an
argument for compatibility with odeint

\end{description}

\item[{Returns}] \leavevmode\begin{description}
\item[{dxdt}] \leavevmode{[}array(12){]}
time derivative of x

\end{description}

\end{description}

\end{fulllineitems}

\index{extendAbundances() (despotic.chemistry.NL99\_GC method)}

\begin{fulllineitems}
\phantomsection\label{fulldoc:despotic.chemistry.NL99_GC.extendAbundances}\pysiglinewithargsret{\bfcode{extendAbundances}}{\emph{xin=None}, \emph{outdict=False}}{}
Compute abundances of derived species not directly followed in
the network.
\begin{description}
\item[{Parameters}] \leavevmode\begin{description}
\item[{xin}] \leavevmode{[}array{]}
abundances of species directly tracked in the network;
if left as None, the abundances stored internally to the
network are used

\item[{outdict}] \leavevmode{[}bool{]}
if True, the values are returned as an abundanceDict; if
False, they are returned as a pure numpy array

\end{description}

\item[{Returns}] \leavevmode\begin{description}
\item[{x}] \leavevmode{[}array \textbar{} abundanceDict{]}
abundances, including those of derived species

\end{description}

\end{description}

\end{fulllineitems}

\index{ionRate (despotic.chemistry.NL99\_GC attribute)}

\begin{fulllineitems}
\phantomsection\label{fulldoc:despotic.chemistry.NL99_GC.ionRate}\pysigline{\bfcode{ionRate}}
primary ionization rate from cosmic rays and x-rays

\end{fulllineitems}

\index{mu() (despotic.chemistry.NL99\_GC method)}

\begin{fulllineitems}
\phantomsection\label{fulldoc:despotic.chemistry.NL99_GC.mu}\pysiglinewithargsret{\bfcode{mu}}{\emph{xin=None}}{}
Return mean particle mass in units of H mass
\begin{description}
\item[{Parameters}] \leavevmode\begin{description}
\item[{xin}] \leavevmode{[}array{]}
Chemical composition for which computation is to be
done; if left as None, current chemical composition is
used

\end{description}

\item[{Returns}] \leavevmode\begin{description}
\item[{mu}] \leavevmode{[}float{]}
Mean mass per free particle, in units of H mass

\end{description}

\end{description}

\end{fulllineitems}

\index{nH (despotic.chemistry.NL99\_GC attribute)}

\begin{fulllineitems}
\phantomsection\label{fulldoc:despotic.chemistry.NL99_GC.nH}\pysigline{\bfcode{nH}}
volume density of H nuclei

\end{fulllineitems}

\index{sigmaNT (despotic.chemistry.NL99\_GC attribute)}

\begin{fulllineitems}
\phantomsection\label{fulldoc:despotic.chemistry.NL99_GC.sigmaNT}\pysigline{\bfcode{sigmaNT}}
non-thermal velocity dispersion

\end{fulllineitems}

\index{xHe (despotic.chemistry.NL99\_GC attribute)}

\begin{fulllineitems}
\phantomsection\label{fulldoc:despotic.chemistry.NL99_GC.xHe}\pysigline{\bfcode{xHe}}
He abundance

\end{fulllineitems}


\end{fulllineitems}



\subsection{\texttt{photoreactions}}
\label{fulldoc:photoreactions}\index{photoreactions (class in despotic.chemistry)}

\begin{fulllineitems}
\phantomsection\label{fulldoc:despotic.chemistry.photoreactions}\pysiglinewithargsret{\strong{class }\code{despotic.chemistry.}\bfcode{photoreactions}}{\emph{specList}, \emph{reactions}, \emph{sparse=False}}{}
The photoreactions class is used to handle computation of
photon-induced reaction rates. Generally, it returns reaction
rates for any reaction of the form
gamma + ... -\textgreater{} ...
with a rate that scales as the ISRF strength, parameterized in
units of the Habing (1968) field, multiplied by dust and gas
shielding factors. In addition to the constructor, the class has
only a single method: dxdt, which returns the reaction rates.
\begin{description}
\item[{Parameters}] \leavevmode\begin{description}
\item[{specList: listlike of string}] \leavevmode
List of all chemical species in the full reaction network,
including those that are derived from conservation laws
instead of being computed directly

\item[{reactions}] \leavevmode{[}list of dict{]}
A list listing all the reactions to be registered; each
entry in the list must be a dict containing the keys:
\begin{itemize}
\item {} \begin{description}
\item[{`spec'}] \leavevmode{[}list {]}
list of strings giving the species involved in the reaction

\end{description}

\item {} \begin{description}
\item[{`stoich'}] \leavevmode{[}list{]}
list of int stoichiometric factor for each species

\end{description}

\item {} \begin{description}
\item[{`rate'}] \leavevmode{[}float{]}
reaction rate per target in a chi = 1 radiation field

\end{description}

\item {} \begin{description}
\item[{`av\_fac'}] \leavevmode{[}float{]}
optical depth per unit A\_V

\end{description}

\item {} \begin{description}
\item[{`shield\_fac'}] \leavevmode{[}(optional) callable {]}
callable to compute the
shielding factor; see the dxdt method for an
explanation of how to specify its arguments

\end{description}

\end{itemize}
\begin{description}
\item[{Reaction rates per target are given by }] \leavevmode
chi * rate * shield\_fac * exp(-av\_fac * A\_V)

\end{description}

\item[{sparse}] \leavevmode{[}bool{]}
If True, the reaction rate matrix is represented as a
sparse matrix; otherwise it is a dense matrix. This has no
effect on results, but depending on the chemical network it
may lead to improved execution speed and/or reduced memory
usage.

\end{description}

\end{description}

\end{fulllineitems}



\subsection{\texttt{reaction\_matrix}}
\label{fulldoc:reaction-matrix}\index{reaction\_matrix (class in despotic.chemistry)}

\begin{fulllineitems}
\phantomsection\label{fulldoc:despotic.chemistry.reaction_matrix}\pysiglinewithargsret{\strong{class }\code{despotic.chemistry.}\bfcode{reaction\_matrix}}{\emph{specList}, \emph{reactions}, \emph{sparse=False}}{}
This class provides a generic driver for computing rates of change
of chemical species from chemical reactions. This class does the
work of mapping from the reaction rates to rates of change in
species abundances.
\begin{description}
\item[{Parameters}] \leavevmode\begin{description}
\item[{specList: listlike of string}] \leavevmode
List of all chemical species in the full reaction network,
including those that are derived from conservation laws
instead of being computed directly

\item[{reactions}] \leavevmode{[}list of dict{]}
A list listing all the reactions to be registered; each
entry in the list must be a dict containing the keys `spec'
and `stoich', which list the species involved in the
reaction and the stoichiometric factors for each species,
respectively. Sign convention is that reactants on the left
hand side have negative stoichiometric factors, those on the
right hand side have positive factors.

\item[{sparse}] \leavevmode{[}bool{]}
If True, the reaction rate matrix is represented as a
sparse matrix; otherwise it is a dense matrix. This has no
effect on results, but depending on the chemical network it
may lead to improved execution speed and/or reduced memory
usage.

\end{description}

\item[{Examples}] \leavevmode
To describe the reaction C + O -\textgreater{} CO, the correct dict entry is:

\begin{Verbatim}[commandchars=\\\{\}]
\PYG{p}{\PYGZob{}} \PYG{l+s+s1}{\PYGZsq{}}\PYG{l+s+s1}{spec}\PYG{l+s+s1}{\PYGZsq{}} \PYG{p}{:} \PYG{p}{[}\PYG{l+s+s1}{\PYGZsq{}}\PYG{l+s+s1}{C}\PYG{l+s+s1}{\PYGZsq{}}\PYG{p}{,} \PYG{l+s+s1}{\PYGZsq{}}\PYG{l+s+s1}{O}\PYG{l+s+s1}{\PYGZsq{}}\PYG{p}{,} \PYG{l+s+s1}{\PYGZsq{}}\PYG{l+s+s1}{CO}\PYG{l+s+s1}{\PYGZsq{}}\PYG{p}{]}\PYG{p}{,} \PYG{l+s+s1}{\PYGZsq{}}\PYG{l+s+s1}{stoich}\PYG{l+s+s1}{\PYGZsq{}} \PYG{p}{:} \PYG{p}{[}\PYG{o}{\PYGZhy{}}\PYG{l+m+mi}{1}\PYG{p}{,} \PYG{o}{\PYGZhy{}}\PYG{l+m+mi}{1}\PYG{p}{,} \PYG{l+m+mi}{1}\PYG{p}{]} \PYG{p}{\PYGZcb{}}
\end{Verbatim}

To describe the reaction H + H -\textgreater{} H2, the dict is:

\begin{Verbatim}[commandchars=\\\{\}]
\PYG{p}{\PYGZob{}} \PYG{l+s+s1}{\PYGZsq{}}\PYG{l+s+s1}{spec}\PYG{l+s+s1}{\PYGZsq{}} \PYG{p}{:} \PYG{p}{[}\PYG{l+s+s1}{\PYGZsq{}}\PYG{l+s+s1}{H}\PYG{l+s+s1}{\PYGZsq{}}\PYG{p}{,} \PYG{l+s+s1}{\PYGZsq{}}\PYG{l+s+s1}{H2}\PYG{l+s+s1}{\PYGZsq{}}\PYG{p}{]}\PYG{p}{,} \PYG{l+s+s1}{\PYGZsq{}}\PYG{l+s+s1}{stoich}\PYG{l+s+s1}{\PYGZsq{}} \PYG{p}{:} \PYG{p}{[}\PYG{o}{\PYGZhy{}}\PYG{l+m+mi}{2}\PYG{p}{,} \PYG{l+m+mi}{1}\PYG{p}{]} \PYG{p}{\PYGZcb{}}
\end{Verbatim}

\end{description}
\index{dxdt() (despotic.chemistry.reaction\_matrix method)}

\begin{fulllineitems}
\phantomsection\label{fulldoc:despotic.chemistry.reaction_matrix.dxdt}\pysiglinewithargsret{\bfcode{dxdt}}{\emph{x}, \emph{n}, \emph{ratecoef}}{}
This returns the rate of change of species abundances given a
set of rate coefficients.
\begin{description}
\item[{Parameters}] \leavevmode\begin{description}
\item[{x}] \leavevmode{[}array(N\_species){]}
array of current species abundances

\item[{n}] \leavevmode{[}float{]}
number density of H nuclei

\item[{ratecoef}] \leavevmode{[}array(N\_reactions){]}
rate coefficients for each reaction; reaction rate per unit
volume = ratecoef * product of densities of reactants;
dxdt = reaction rate / unit volume / n

\end{description}

\item[{Returns}] \leavevmode\begin{description}
\item[{dxdt}] \leavevmode{[}array(N){]}
rate of change of all species abundances

\end{description}

\end{description}

\end{fulllineitems}


\end{fulllineitems}



\section{despotic.chemistry functions}
\label{fulldoc:despotic-chemistry-functions}

\subsection{\texttt{chemEvol}}
\label{fulldoc:chemevol}\index{chemEvol() (in module despotic.chemistry)}

\begin{fulllineitems}
\phantomsection\label{fulldoc:despotic.chemistry.chemEvol}\pysiglinewithargsret{\code{despotic.chemistry.}\bfcode{chemEvol}}{\emph{cloud}, \emph{tFin}, \emph{tInit=0.0}, \emph{nOut=100}, \emph{dt=None}, \emph{tOut=None}, \emph{network=None}, \emph{info=None}, \emph{addEmitters=False}, \emph{evolveTemp='fixed'}, \emph{isobaric=False}, \emph{tempEqParam=None}, \emph{dEdtParam=None}}{}
Evolve the abundances of a cloud using the specified chemical
network.
\begin{description}
\item[{Parameters}] \leavevmode\begin{description}
\item[{cloud}] \leavevmode{[}class cloud{]}
cloud on which computation is to be performed

\item[{tFin}] \leavevmode{[}float{]}
end time of integration, in sec

\item[{tInit}] \leavevmode{[}float{]}
start time of integration, in sec

\item[{nOut}] \leavevmode{[}int{]}
number of times at which to report the temperature; this
is ignored if dt or tOut are set

\item[{dt}] \leavevmode{[}float{]}
time interval between outputs, in sec; this is ignored if
tOut is set

\item[{tOut}] \leavevmode{[}array{]}
list of times at which to output the temperature, in s;
must be sorted in increasing order

\item[{network}] \leavevmode{[}chemical network class{]}
a valid chemical network class; this class must define the
methods \_\_init\_\_, dxdt, and applyAbundances; if None, the
existing chemical network for the cloud is used

\item[{info}] \leavevmode{[}dict{]}
a dict of additional initialization information to be passed
to the chemical network class when it is instantiated

\item[{addEmitters}] \leavevmode{[}Boolean{]}
if True, emitters that are included in the chemical
network but not in the cloud's existing emitter list will
be added; if False, abundances of emitters already in the
emitter list will be updated, but new emiters will not be
added to the cloud

\item[{evolveTemp}] \leavevmode{[}`fixed' \textbar{} `gasEq' \textbar{} `fullEq' \textbar{} `evol'{]}
how to treat the temperature evolution during the chemical
evolution; `fixed' = treat tempeature as fixed; `gasEq' = hold
dust temperature fixed, set gas temperature to instantaneous
equilibrium value; `fullEq' = set gas and dust temperatures to
instantaneous equilibrium values; `evol' = evolve gas
temperature in time along with the chemistry, assuming the
dust is always in instantaneous equilibrium

\item[{isobaric}] \leavevmode{[}Boolean{]}
if set to True, the gas is assumed to be isobaric during the
evolution (constant pressure); otherwise it is assumed to be
isochoric; note that (since chemistry networks at present are
not allowed to change the mean molecular weight), this option
has no effect if evolveTemp is `fixed'

\item[{tempEqParam}] \leavevmode{[}None \textbar{} dict{]}
if this is not None, then it must be a dict of values that
will be passed as keyword arguments to the cloud.setTempEq,
cloud.setGasTempEq, or cloud.setDustTempEq routines; only used
if evolveTemp is not `fixed'

\item[{dEdtParam}] \leavevmode{[}None \textbar{} dict{]}
if this is not None, then it must be a dict of values that
will be passed as keyword arguments to the cloud.dEdt
routine; only used if evolveTemp is `evol'

\end{description}

\item[{Returns}] \leavevmode\begin{description}
\item[{time}] \leavevmode{[}array{]}
array of output times, in sec

\item[{abundances}] \leavevmode{[}class abundanceDict{]}
an abundanceDict giving the abundances as a function of time

\item[{Tg}] \leavevmode{[}array{]}
gas temperature as a function of time; returned only if
evolveTemp is not `fixed'

\item[{Td}] \leavevmode{[}array{]}
dust temperature as a function of time; returned only if
evolveTemp is not `fixed' or `gasEq'

\end{description}

\item[{Raises}] \leavevmode
despoticError, if network is None and the cloud does not already
have a defined chemical network associated with it

\end{description}

\end{fulllineitems}



\subsection{\texttt{setChemEq}}
\label{fulldoc:setchemeq}\index{setChemEq() (in module despotic.chemistry)}

\begin{fulllineitems}
\phantomsection\label{fulldoc:despotic.chemistry.setChemEq}\pysiglinewithargsret{\code{despotic.chemistry.}\bfcode{setChemEq}}{\emph{cloud}, \emph{tEqGuess=None}, \emph{network=None}, \emph{info=None}, \emph{addEmitters=False}, \emph{tol=1e-06}, \emph{maxTime=1e+16}, \emph{verbose=False}, \emph{smallabd=1e-15}, \emph{convList=None}, \emph{evolveTemp='fixed'}, \emph{isobaric=False}, \emph{tempEqParam=None}, \emph{dEdtParam=None}, \emph{maxTempIter=50}}{}
Set the chemical abundances for a cloud to their equilibrium
values, computed using a specified chemical netowrk.
\begin{description}
\item[{Parameters}] \leavevmode\begin{description}
\item[{cloud}] \leavevmode{[}class cloud{]}
cloud on which computation is to be performed

\item[{tEqGuess}] \leavevmode{[}float{]}
a guess at the timescale over which equilibrium will be
achieved; if left unspecified, the code will attempt to
estimate this time scale on its own

\item[{network}] \leavevmode{[}chemNetwork class{]}
a valid chemNetwork class; this class must define the
methods \_\_init\_\_, dxdt, and applyAbundances; if None, the
existing chemical network for the cloud is used

\item[{info}] \leavevmode{[}dict{]}
a dict of additional initialization information to be passed
to the chemical network class when it is instantiated

\item[{addEmitters}] \leavevmode{[}Boolean{]}
if True, emitters that are included in the chemical
network but not in the cloud's existing emitter list will
be added; if False, abundances of emitters already in the
emitter list will be updated, but new emiters will not be
added to the cloud

\item[{evolveTemp}] \leavevmode{[}`fixed' \textbar{} `iterate' \textbar{} `iterateDust' \textbar{} `gasEq' \textbar{} `fullEq' \textbar{} `evol'{]}
how to treat the temperature evolution during the chemical
evolution:
\begin{itemize}
\item {} 
`fixed' = treat tempeature as fixed

\item {} 
`iterate' = iterate between setting the gas temperature and
chemistry to equilibrium

\item {} 
`iterateDust' = iterate between setting the gas and dust
temperatures and the chemistry to equilibrium

\item {} 
`gasEq' = hold dust temperature fixed, set gas temperature to
instantaneous equilibrium value as the chemistry evolves

\item {} 
`fullEq' = set gas and dust temperatures to instantaneous
equilibrium values while evolving the chemistry network

\item {} 
`evol' = evolve gas temperature in time along with the
chemistry, assuming the dust is always in instantaneous
equilibrium

\end{itemize}

\item[{isobaric}] \leavevmode{[}Boolean{]}
if set to True, the gas is assumed to be isobaric during the
evolution (constant pressure); otherwise it is assumed to be
isochoric; note that (since chemistry networks at present are
not allowed to change the mean molecular weight), this option
has no effect if evolveTemp is `fixed'

\item[{tempEqParam}] \leavevmode{[}None \textbar{} dict{]}
if this is not None, then it must be a dict of values that
will be passed as keyword arguments to the cloud.setTempEq,
cloud.setGasTempEq, or cloud.setDustTempEq routines; only used
if evolveTemp is not `fixed'

\item[{dEdtParam}] \leavevmode{[}None \textbar{} dict{]}
if this is not None, then it must be a dict of values that
will be passed as keyword arguments to the cloud.dEdt
routine; only used if evolveTemp is `evol'

\item[{tol}] \leavevmode{[}float{]}
tolerance requirement on the equilibrium solution

\item[{convList}] \leavevmode{[}list{]}
list of species to include when calculating tolerances to
decide if network is converged; species not listed are not
considered. If this is None, then all species are considered
in deciding if the calculation is converged.

\item[{smallabd}] \leavevmode{[}float{]}
abundances below smallabd are not considered when checking for
convergence; set to 0 or a negative value to consider all
abundances, but beware that this may result in false
non-convergence due to roundoff error in very small abundances

\item[{maxTempIter}] \leavevmode{[}int{]}
maximum number of iterations when iterating between chemistry
and temperature; only used if evolveTemp is `iterate' or
`iterateDust'

\item[{verbose}] \leavevmode{[}Boolean{]}
if True, diagnostic information is printed as the calculation
proceeds

\end{description}

\item[{Returns}] \leavevmode\begin{description}
\item[{converged}] \leavevmode{[}Boolean{]}
True if the calculation converged, False if not

\end{description}

\item[{Raises}] \leavevmode
despoticError, if network is None and the cloud does not already
have a defined chemical network associated with it

\item[{Remarks}] \leavevmode
The final abundances are written to the cloud whether or not the
calculation converges.

\end{description}

\end{fulllineitems}



\subsection{Shielding functions}
\label{fulldoc:shielding-functions}\index{fShield\_H2\_DB() (in module despotic.chemistry.shielding)}

\begin{fulllineitems}
\phantomsection\label{fulldoc:despotic.chemistry.shielding.fShield_H2_DB}\pysiglinewithargsret{\code{despotic.chemistry.shielding.}\bfcode{fShield\_H2\_DB}}{\emph{NH2}, \emph{sigma}}{}
This function returns the shielding factor for H2
photodissociation as a function of the H2 column density and
velocity dispersion, based on the approximation formula of Draine
\& Bertoldi (1996)
\begin{description}
\item[{Parameters}] \leavevmode\begin{description}
\item[{NH2}] \leavevmode{[}float \textbar{} array{]}
H2 column density in cm\textasciicircum{}-2

\item[{sigma}] \leavevmode{[}float \textbar{} array{]}
velocity dispersion in cm s\textasciicircum{}-1

\end{description}

\item[{Returns}] \leavevmode\begin{description}
\item[{fShield}] \leavevmode{[}float \textbar{} array{]}
the shielding factor for the input NH2 and sigma

\end{description}

\end{description}

\end{fulllineitems}

\index{fShield\_CO\_vDB() (in module despotic.chemistry.shielding)}

\begin{fulllineitems}
\phantomsection\label{fulldoc:despotic.chemistry.shielding.fShield_CO_vDB}\pysiglinewithargsret{\code{despotic.chemistry.shielding.}\bfcode{fShield\_CO\_vDB}}{\emph{NCO}, \emph{NH2}, \emph{order=1}}{}
This function returns the shielding factor for CO
photodissociation as a function of CO and H2 column densities,
based on the model of van Dishoeck \& Black (1987)
\begin{description}
\item[{Parameters}] \leavevmode\begin{description}
\item[{NCO}] \leavevmode{[}float \textbar{} array{]}
CO column density in cm\textasciicircum{}-2

\item[{NH2}] \leavevmode{[}float \textbar{} array{]}
H2 column density in cm\textasciicircum{}-2

\item[{order}] \leavevmode{[}1 \textbar{} 2 \textbar{} 3{]}
order of spline interpolation on van Dishoeck \& Black's table;
1 is the safest choice, but 2 and 3 are also provided, and may
be more accurate in some ranges

\end{description}

\item[{Returns}] \leavevmode\begin{description}
\item[{fShield}] \leavevmode{[}array{]}
the shielding factor for the input NCO and NH2 values

\end{description}

\item[{Raises}] \leavevmode
ValueError if order is not 1, 2, or 3

\end{description}

\end{fulllineitems}



\section{despotic.winds classes}
\label{fulldoc:despotic-winds-classes}

\subsection{\texttt{pwind}}
\label{fulldoc:sssec-full-pwind}\label{fulldoc:pwind}\index{pwind (class in despotic.winds)}

\begin{fulllineitems}
\phantomsection\label{fulldoc:despotic.winds.pwind}\pysiglinewithargsret{\strong{class }\code{despotic.winds.}\bfcode{pwind}}{\emph{Gamma}, \emph{mach}, \emph{driver='ideal'}, \emph{potential='point'}, \emph{expansion='intermediate'}, \emph{geometry='sphere'}, \emph{fcrit=1.0}, \emph{epsabs=0.0001}, \emph{epsrel=0.01}, \emph{theta=None}, \emph{phi=None}, \emph{theta\_in=None}, \emph{tau0=None}, \emph{uh=None}, \emph{interpabs=0.01}, \emph{interprel=0.01}}{}~\index{\_\_init\_\_() (despotic.winds.pwind method)}

\begin{fulllineitems}
\phantomsection\label{fulldoc:despotic.winds.pwind.__init__}\pysiglinewithargsret{\bfcode{\_\_init\_\_}}{\emph{Gamma}, \emph{mach}, \emph{driver='ideal'}, \emph{potential='point'}, \emph{expansion='intermediate'}, \emph{geometry='sphere'}, \emph{fcrit=1.0}, \emph{epsabs=0.0001}, \emph{epsrel=0.01}, \emph{theta=None}, \emph{phi=None}, \emph{theta\_in=None}, \emph{tau0=None}, \emph{uh=None}, \emph{interpabs=0.01}, \emph{interprel=0.01}}{}
Creates a generic wind object to compute observable property
of winds
\begin{description}
\item[{Parameters}] \leavevmode\begin{description}
\item[{Gamma: float}] \leavevmode
Eddington factor

\item[{mach: float}] \leavevmode
Mach number

\item[{driver: `ideal' \textbar{} `radiation' \textbar{} `hot'}] \leavevmode
wind driving mechanism; allowed values are `ideal'
(ideal momentum-driven wind), `radiation'
(radiation-driven wind), and `hot' (wind driven by hot
gas entrainment)

\item[{potential: `point' \textbar{} `isothermal'}] \leavevmode
gravitational potential confining the wind; allowed
values are `point' and `isothermal', for point and
isothermal potentials, respectively

\item[{expansion: `area' \textbar{} `intermediate' \textbar{} `solid angle'}] \leavevmode
cloud expansion law: clouds can maintain constant area
(`area'), maintain constant solid angle (`solid angle'),
or have intermediate behavior (`intermediate') where the
area increases with distance as r and the solid angle
decreases as 1/r

\item[{geometry: `sphere' \textbar{} `cone' \textbar{} `cone sheath'}] \leavevmode
geometry of the wind; allowed values are sphere (covers
all space), conical, and conical sheath (meaning that
the wind is bounded by an inner and outer cone)

\item[{fcrit: float}] \leavevmode
material is only considered to be launched into the wind
if x \textless{} f\_crit x\_crit; must be \textless{}= 1.0

\item[{epsabs: float}] \leavevmode
absolute error tolerance for numerical integrations

\item[{epsrel: float}] \leavevmode
relative error tolerance for numerical integrations

\item[{theta: float, in the range (0, pi/2)}] \leavevmode
opening angle of the outer edge of the wind for cone or
cone sheath geometry; ignored for all other geometries

\item[{phi: float, in the range {[}-pi/2, pi/2{]}}] \leavevmode
inclination of the wind cone central axis relative to
the plane of the sky for either cone or cone sheath
geometry; ignored for all other geometries; phi = 0
corresponds to a wind cone in the plane of the sky, phi \textgreater{} 0
corresponds to the varpi \textgreater{} 0 side of the wind pointed
away from the observer

\item[{theta\_in: float, in range (0, theta)}] \leavevmode
opening angle of the inner edge of the wind for cone
sheath geometry, ignored for all other geometries

\item[{tau0: float}] \leavevmode
for radiation-driven winds, the optical depth at the
mean surface density; must be specified for
radiation-driven winds, ignored for all other drivers

\item[{uh: float}] \leavevmode
for hot gas-driven winds, the hot gas speed relative to
the escape speed; must be specified for hot gas-driven
winds, ignored for all other drivers

\item[{interpabs: float}] \leavevmode
absolute error tolerance of the iterpolation tables used
for hot gas-driven winds; ignored for all other drivers;
note that using smaller tolerances quickly becomes very
expensive in memory and computation time, so values much
below 10\textasciicircum{}-2 are not recommended

\item[{interprel: float}] \leavevmode
same as interpabs, but giving relative rather tahn
absolute error tolerance

\end{description}

\end{description}

\end{fulllineitems}

\index{Phi\_c() (despotic.winds.pwind method)}

\begin{fulllineitems}
\phantomsection\label{fulldoc:despotic.winds.pwind.Phi_c}\pysiglinewithargsret{\bfcode{Phi\_c}}{\emph{u}, \emph{fw=None}, \emph{varpi=0.0}, \emph{varpi\_t=0.0}, \emph{a0=1.0}, \emph{a1=1.7976931348623157e+308}}{}
Return the correlated absorption function Phi\_c
\begin{description}
\item[{Parameters:}] \leavevmode\begin{description}
\item[{u: float or arraylike}] \leavevmode
dimensionless line of sight velocity

\item[{fw: float or arraylike}] \leavevmode
covering factor of wind at launch point; if left as
None, defaults to zeta\_A

\item[{varpi: float or arraylike}] \leavevmode
dimensionless impact parameter along the wind axis

\item[{varpi\_t: float or arraylike}] \leavevmode
dimensionless impact parameter transverse to the wind axis

\item[{a0: float or arraylike}] \leavevmode
minimum radius from which to integrate

\item[{a1: float or arraylike}] \leavevmode
maximum radius from which to integrate

\end{description}

\item[{Returns:}] \leavevmode\begin{description}
\item[{Phi\_c: float or array}] \leavevmode
correlated absorption function

\end{description}

\end{description}

\end{fulllineitems}

\index{Phi\_uc() (despotic.winds.pwind method)}

\begin{fulllineitems}
\phantomsection\label{fulldoc:despotic.winds.pwind.Phi_uc}\pysiglinewithargsret{\bfcode{Phi\_uc}}{\emph{u}, \emph{varpi=0.0}, \emph{varpi\_t=0.0}, \emph{a0=1.0}, \emph{a1=1.7976931348623157e+308}}{}
Return the uncorrelated absorption function Phi\_c
\begin{description}
\item[{Parameters:}] \leavevmode\begin{description}
\item[{u: float or arraylike}] \leavevmode
dimensionless line of sight velocity

\item[{varpi: float or arraylike}] \leavevmode
dimensionless impact parameter along the wind axis

\item[{varpi\_t: float or arraylike}] \leavevmode
dimensionless impact parameter transverse to the wind
axis

\item[{a0: float or arraylike}] \leavevmode
minimum radius from which to integrate Phi\_uc

\item[{a1: float or arraylike}] \leavevmode
maximum radius from which to integrate Phi\_uc

\end{description}

\item[{Returns:}] \leavevmode\begin{description}
\item[{Phi\_uc: float or array}] \leavevmode
correlated absorption function

\end{description}

\end{description}

\end{fulllineitems}

\index{Psi() (despotic.winds.pwind method)}

\begin{fulllineitems}
\phantomsection\label{fulldoc:despotic.winds.pwind.Psi}\pysiglinewithargsret{\bfcode{Psi}}{\emph{tXtw}, \emph{fj}, \emph{boltzfac}, \emph{correlated=True}, \emph{fw=None}, \emph{varpi=0.0}, \emph{varpi\_t=0.0}, \emph{thin=False}}{}
Return the integrated intensity integral Psi
\begin{description}
\item[{Parameters:}] \leavevmode\begin{description}
\item[{tXtw: float or array}] \leavevmode
ratio of timescales tX and tw

\item[{fj: float or array}] \leavevmode
fraction of the emitters in the lower state of the
transition

\item[{boltzfac: float or array}] \leavevmode
Boltzmann factor exp(-E\_ij/kB T) for the two states of
the transition

\item[{correlated: bool}] \leavevmode
if True, assume correlated winds; if False, uncorrelated

\item[{fw: float or arraylike}] \leavevmode
covering factor of wind at launch point; if left as
None, defaults to zeta\_A; only used if correlated is True

\item[{varpi: float or arraylike}] \leavevmode
dimensionless impact parameter along the wind axis

\item[{varpi\_t: float or arraylike}] \leavevmode
dimensionless impact parameter transverse to the wind axis

\item[{thin: bool or array of bool}] \leavevmode
if True, the escape probability is set to 1

\end{description}

\item[{Returns:}] \leavevmode\begin{description}
\item[{Psi: float or array}] \leavevmode
value of the integral Psi

\end{description}

\end{description}

\end{fulllineitems}

\index{U() (despotic.winds.pwind method)}

\begin{fulllineitems}
\phantomsection\label{fulldoc:despotic.winds.pwind.U}\pysiglinewithargsret{\bfcode{U}}{\emph{x}, \emph{a}}{}
Return the radial velocity U\_a(x) for this wind
\begin{description}
\item[{Parameters:}] \leavevmode\begin{description}
\item[{x: float or arraylike}] \leavevmode
dimensionless column density

\item[{a: float or arraylike}] \leavevmode
dimensionless radius

\end{description}

\item[{Returns:}] \leavevmode\begin{description}
\item[{U: float or arraylike}] \leavevmode
dimensionless radial velocity

\end{description}

\end{description}

\end{fulllineitems}

\index{U2() (despotic.winds.pwind method)}

\begin{fulllineitems}
\phantomsection\label{fulldoc:despotic.winds.pwind.U2}\pysiglinewithargsret{\bfcode{U2}}{\emph{x}, \emph{a}}{}
Return the square velocity U\_a\textasciicircum{}2(x) for this wind
\begin{description}
\item[{Parameters:}] \leavevmode\begin{description}
\item[{x: float or arraylike}] \leavevmode
dimensionless column density

\item[{a: float or arraylike}] \leavevmode
dimensionless radius

\end{description}

\item[{Returns:}] \leavevmode\begin{description}
\item[{U2: float or arraylike}] \leavevmode
dimensionless square velocity; a value of nan is returned
for any combinations of x and a that are forbidden for
this wind

\end{description}

\end{description}

\end{fulllineitems}

\index{X() (despotic.winds.pwind method)}

\begin{fulllineitems}
\phantomsection\label{fulldoc:despotic.winds.pwind.X}\pysiglinewithargsret{\bfcode{X}}{\emph{ur}, \emph{a}}{}
Return the column density X\_a(ur) for this wind
\begin{description}
\item[{Parameters:}] \leavevmode\begin{description}
\item[{ur: float}] \leavevmode
dimensionless radial velocity

\item[{a: float}] \leavevmode
dimensionless radius

\end{description}

\item[{Returns:}] \leavevmode\begin{description}
\item[{X: array}] \leavevmode
dimensionless column densities; a value of nan is
returned for combinations of ur and a that are forbitten
for this wind

\end{description}

\end{description}

\end{fulllineitems}

\index{Xfac() (despotic.winds.pwind method)}

\begin{fulllineitems}
\phantomsection\label{fulldoc:despotic.winds.pwind.Xfac}\pysiglinewithargsret{\bfcode{Xfac}}{\emph{T}, \emph{emit}, \emph{tw}, \emph{muH=1.4}, \emph{trans=0}, \emph{varpi=0.0}, \emph{varpi\_t=0.0}, \emph{correlated=True}, \emph{fw=None}}{}
Return the X factor -- the conversion between mass and
velocity-integrated antenna temperature.
\begin{description}
\item[{Parameters:}] \leavevmode\begin{description}
\item[{T: float or array}] \leavevmode
wind kinetic temperature, in K

\item[{emit: emitter or sequence of emitter}] \leavevmode
an emitter object for the transition of interest

\item[{tw: float or arraylike}] \leavevmode
mass removal timescale, in sec

\item[{trans: int or array of int}] \leavevmode
transition for which to compute the emission

\item[{varpi: float or arraylike}] \leavevmode
dimensionless impact parameter along the wind axis

\item[{varpi\_t: float or arraylike}] \leavevmode
dimensionless impact parameter transverse to the wind axis

\item[{correlated: bool}] \leavevmode
if True, assume correlated winds; if False, uncorrelated

\item[{fw: float or arraylike}] \leavevmode
covering factor of wind at launch point; if left as

\end{description}

\item[{Returns:}] \leavevmode\begin{description}
\item[{Xfac: float or array}] \leavevmode
X factor, in the same units as Xthin

\end{description}

\end{description}

\end{fulllineitems}

\index{Xi() (despotic.winds.pwind method)}

\begin{fulllineitems}
\phantomsection\label{fulldoc:despotic.winds.pwind.Xi}\pysiglinewithargsret{\bfcode{Xi}}{\emph{u}, \emph{varpi=0.0}, \emph{varpi\_t=0.0}}{}
Return the optically thin emission line shape function
Xi.
\begin{description}
\item[{Parameters:}] \leavevmode\begin{description}
\item[{u: float or arraylike}] \leavevmode
dimensionless line of sight velocity

\item[{varpi: float or arraylike}] \leavevmode
dimensionless impact parameter along the wind axis

\item[{varpi\_t: float or arraylike}] \leavevmode
dimensionless impact parameter transverse to the wind axis

\end{description}

\item[{Returns:}] \leavevmode\begin{description}
\item[{Xi: float or array}] \leavevmode
subcritical line emission shape function

\end{description}

\end{description}

\end{fulllineitems}

\index{a\_crit() (despotic.winds.pwind method)}

\begin{fulllineitems}
\phantomsection\label{fulldoc:despotic.winds.pwind.a_crit}\pysiglinewithargsret{\bfcode{a\_crit}}{\emph{varpi}, \emph{varpi\_t}, \emph{u=0.0}}{}
Return the radii where a given line of sight enters and exits
the wind, or passes through the plane of the sky
\begin{description}
\item[{Parameters:}] \leavevmode\begin{description}
\item[{varpi: float}] \leavevmode
dimensionless impact parameter along the wind axis

\item[{varpi\_t: float}] \leavevmode
dimensionless impact parameter transverse to the wind axis

\item[{u: float}] \leavevmode
velocity of interest; if u \textgreater{} 0, only distances on the
far side wind are returned; if u \textless{} 0, only
distances on the near side are returned; if u == 0.0,
both far and near side distances are returned

\end{description}

\item[{Returns:}] \leavevmode\begin{description}
\item[{a\_crit: array(0), array(2) or array(4)}] \leavevmode
dimensionless radii of wind entry or exit, or
passage through the midplane, ordered from smallest to
largest value of s

\end{description}

\end{description}

\end{fulllineitems}

\index{dU2da() (despotic.winds.pwind method)}

\begin{fulllineitems}
\phantomsection\label{fulldoc:despotic.winds.pwind.dU2da}\pysiglinewithargsret{\bfcode{dU2da}}{\emph{x}, \emph{a}}{}
Return d/da(U\_a\textasciicircum{}2(x)) for this wind
\begin{description}
\item[{Parameters:}] \leavevmode\begin{description}
\item[{x: float or arraylike}] \leavevmode
dimensionless column density

\item[{a: float or arraylike}] \leavevmode
dimensionless radius

\end{description}

\item[{Returns:}] \leavevmode\begin{description}
\item[{dU2da: float or arraylike}] \leavevmode
dimensionless d/da(U\textasciicircum{}2); a value of nan is returned
for any combinations of x and a that are forbidden for
this wind

\end{description}

\end{description}

\end{fulllineitems}

\index{dU2dx() (despotic.winds.pwind method)}

\begin{fulllineitems}
\phantomsection\label{fulldoc:despotic.winds.pwind.dU2dx}\pysiglinewithargsret{\bfcode{dU2dx}}{\emph{x}, \emph{a}}{}
Return d/dx(U\_a\textasciicircum{}2(x)) for this wind
\begin{description}
\item[{Parameters:}] \leavevmode\begin{description}
\item[{x: float or arraylike}] \leavevmode
dimensionless column density

\item[{a: float or arraylike}] \leavevmode
dimensionless radius

\end{description}

\item[{Returns:}] \leavevmode\begin{description}
\item[{dU2dx: float or arraylike}] \leavevmode
dimensionless d/dx(U\textasciicircum{}2); a value of nan is returned
for any combinations of x and a that are forbidden for
this wind

\end{description}

\end{description}

\end{fulllineitems}

\index{dfcda() (despotic.winds.pwind method)}

\begin{fulllineitems}
\phantomsection\label{fulldoc:despotic.winds.pwind.dfcda}\pysiglinewithargsret{\bfcode{dfcda}}{\emph{a}}{}
Return derivative of the covering factor df\_c / da
\begin{description}
\item[{Parameters:}] \leavevmode\begin{description}
\item[{a: float or arraylike}] \leavevmode
dimensionless radius

\end{description}

\item[{Returns:}] \leavevmode\begin{description}
\item[{dfcda: float or arraylike}] \leavevmode
value of df\_c/da

\end{description}

\end{description}

\end{fulllineitems}

\index{dfcdx() (despotic.winds.pwind method)}

\begin{fulllineitems}
\phantomsection\label{fulldoc:despotic.winds.pwind.dfcdx}\pysiglinewithargsret{\bfcode{dfcdx}}{\emph{x}, \emph{a}}{}
Return the differential covering factor dfc/dx as a
function of x and a
\begin{description}
\item[{Parameters:}] \leavevmode\begin{description}
\item[{x: float or arraylike}] \leavevmode
dimensionless column density

\item[{a: float or arraylike}] \leavevmode
dimensionless radius

\end{description}

\item[{Returns:}] \leavevmode\begin{description}
\item[{dfcdx: float or arraylike}] \leavevmode
differential covering factor

\end{description}

\end{description}

\end{fulllineitems}

\index{drhodx() (despotic.winds.pwind method)}

\begin{fulllineitems}
\phantomsection\label{fulldoc:despotic.winds.pwind.drhodx}\pysiglinewithargsret{\bfcode{drhodx}}{\emph{x}, \emph{a}}{}
Return the differential contribution to the mean density,
drho\_mean/dx, from material with initial surface density x at
radius a
\begin{description}
\item[{Parameters:}] \leavevmode\begin{description}
\item[{x: float or arraylike}] \leavevmode
dimensionless column density

\item[{a: float or arraylike}] \leavevmode
dimensionless radius

\end{description}

\item[{Returns:}] \leavevmode\begin{description}
\item[{drhodx: float or arraylike}] \leavevmode
dimensionless differential density

\end{description}

\end{description}

\end{fulllineitems}

\index{dyda() (despotic.winds.pwind method)}

\begin{fulllineitems}
\phantomsection\label{fulldoc:despotic.winds.pwind.dyda}\pysiglinewithargsret{\bfcode{dyda}}{\emph{a}}{}
Return the derivative of the cloud area function dy/da for
this wind
\begin{description}
\item[{Parameters:}] \leavevmode\begin{description}
\item[{a: float or arraylike}] \leavevmode
dimensionless radius

\end{description}

\item[{Returns:}] \leavevmode\begin{description}
\item[{dyda: float or arraylike}] \leavevmode
derivative of cloud area function

\end{description}

\end{description}

\end{fulllineitems}

\index{eta() (despotic.winds.pwind method)}

\begin{fulllineitems}
\phantomsection\label{fulldoc:despotic.winds.pwind.eta}\pysiglinewithargsret{\bfcode{eta}}{\emph{u}, \emph{tXtw=None}, \emph{abd=None}, \emph{Omega=None}, \emph{wl=None}, \emph{muH=1.4}, \emph{tw=None}, \emph{fj=1.0}, \emph{boltzfac=0.0}, \emph{correlated=True}, \emph{fw=None}, \emph{varpi=0.0}, \emph{varpi\_t=0.0}, \emph{thin=False}}{}
Return the LTE emission profile function eta
\begin{description}
\item[{Parameters:}] \leavevmode\begin{description}
\item[{u: float or array}] \leavevmode
dimensionless line of sight velocity

\item[{tXtw: float or array}] \leavevmode
ratio of timescales tX and tw

\item[{abd: float or arraylike}] \leavevmode
abundance of absorbers relative to H

\item[{Omega: float or arraylike}] \leavevmode
oscillator strength for transition

\item[{wl: float or arraylike}] \leavevmode
wavelength for transition, in cm

\item[{muH: float or arraylike}] \leavevmode
gas mass per H nucleus, in units of H masses

\item[{tw: float or arraylike}] \leavevmode
mass removal timescale, in sec

\item[{fj: float or array}] \leavevmode
fraction of the emitters in the lower state of the
transition

\item[{boltzfac: float or array}] \leavevmode
Boltzmann factor exp(-E\_ij/kB T) for the two states of
the transition

\item[{correlated: bool}] \leavevmode
if True, assume correlated winds; if False, uncorrelated

\item[{fw: float or arraylike}] \leavevmode
covering factor of wind at launch point; if left as
None, defaults to zeta\_A; only used if correlated is True

\item[{varpi: float or arraylike}] \leavevmode
dimensionless impact parameter along the wind axis

\item[{varpi\_t: float or arraylike}] \leavevmode
dimensionless impact parameter transverse to the wind axis

\item[{thin: bool or array of bool}] \leavevmode
if True, the escape probability is set to 1

\end{description}

\item[{Returns:}] \leavevmode\begin{description}
\item[{eta: float or array}] \leavevmode
the LTE emission function eta

\end{description}

\item[{Notes:}] \leavevmode
The user can choose to specify either tXtw, or the
combination of parameters (abd, Omega, wl, tw). In the
latter case, the user can also optionally change the gas
mass per H nucleus by setting muH. If tXtw is set, the
other parameters will be ignored. If it is not set, then an
error is raised if any of abd, Omega, wl, or tw is not
specified.

\end{description}

\end{fulllineitems}

\index{f\_area() (despotic.winds.pwind method)}

\begin{fulllineitems}
\phantomsection\label{fulldoc:despotic.winds.pwind.f_area}\pysiglinewithargsret{\bfcode{f\_area}}{}{}
Return fraction of the unit sphere covered by the wind for the
current geometry; distinct from the covering fraction f\_c,
which is the fraction of the area within the wind geometric
region that is covered
\begin{description}
\item[{Parameters:}] \leavevmode
None

\item[{Returns:}] \leavevmode\begin{description}
\item[{f\_area: float}] \leavevmode
area fraction

\end{description}

\end{description}

\end{fulllineitems}

\index{fc() (despotic.winds.pwind method)}

\begin{fulllineitems}
\phantomsection\label{fulldoc:despotic.winds.pwind.fc}\pysiglinewithargsret{\bfcode{fc}}{\emph{a}}{}
Return the covering factor f\_c as a function of radius
\begin{description}
\item[{Parameters:}] \leavevmode\begin{description}
\item[{a: float or arraylike}] \leavevmode
dimensionless radius

\end{description}

\item[{Returns:}] \leavevmode\begin{description}
\item[{fc: float or array}] \leavevmode
covering fraction

\end{description}

\end{description}

\end{fulllineitems}

\index{intTA\_LTE() (despotic.winds.pwind method)}

\begin{fulllineitems}
\phantomsection\label{fulldoc:despotic.winds.pwind.intTA_LTE}\pysiglinewithargsret{\bfcode{intTA\_LTE}}{\emph{v0}, \emph{T}, \emph{emit=None}, \emph{tw=None}, \emph{tXtw=None}, \emph{abd=None}, \emph{Omega=None}, \emph{wl=None}, \emph{muH=1.4}, \emph{fj=1.0}, \emph{boltzfac=0.0}, \emph{trans=0}, \emph{varpi=0.0}, \emph{varpi\_t=0.0}, \emph{correlated=True}, \emph{fw=None}, \emph{thin=False}}{}
Returns the velocity-integrated antenna temperature along 
a particular line of sight
\begin{description}
\item[{Parameters:}] \leavevmode\begin{description}
\item[{v0: float or array}] \leavevmode
escape speed of system; units are arbitrary, and output
integrated antenna temperature will be in the same
velocity units as v0

\item[{T: float or array}] \leavevmode
wind kinetic temperature, in K

\item[{emit: emitter or sequence of emitter}] \leavevmode
an emitter object for the transition of interest

\item[{tw: float or arraylike}] \leavevmode
mass removal timescale, in sec

\item[{tXtw: float or array}] \leavevmode
ratio of timescales tX and tw

\item[{abd: float or arraylike}] \leavevmode
abundance of absorbers relative to H

\item[{Omega: float or arraylike}] \leavevmode
oscillator strength for transition

\item[{wl: float or arraylike}] \leavevmode
wavelength for transition, in cm

\item[{muH: float or arraylike}] \leavevmode
gas mass per H nucleus, in units of H masses

\item[{fj: float or array}] \leavevmode
fraction of the emitters in the lower state of the
transition

\item[{boltzfac: float or array}] \leavevmode
Boltzmann factor exp(-E\_ij/kB T) for the two states of
the transition

\item[{trans: int or array of int}] \leavevmode
transition for which to compute the emission

\item[{varpi: float or arraylike}] \leavevmode
dimensionless impact parameter along the wind axis

\item[{varpi\_t: float or arraylike}] \leavevmode
dimensionless impact parameter transverse to the wind axis

\item[{correlated: bool}] \leavevmode
if True, assume correlated winds; if False, uncorrelated

\item[{fw: float or arraylike}] \leavevmode
covering factor of wind at launch point; if left as

\item[{thin: bool or array of bool}] \leavevmode
if True, the escape probability is set to 1

\end{description}

\item[{Returns:}] \leavevmode\begin{description}
\item[{vT: float or array}] \leavevmode
velocity-integrated antenna temperature, in units of K
times whatever units v0 is in (i.e., if v0 is in km / s,
then the returned value will be in units of K km / s)

\end{description}

\item[{Note:}] \leavevmode
The user can provide the required input in the following
ways, which are checked in order.
\begin{enumerate}
\item {} 
Set emit and tw, and optionally trans; in this case
tXtw, abd, Omega, wl, fj, and boltzfac are all ignored;
the abundance and all properties of the line are taken
from emit, and the Boltzmann factors are computed
directly from the emitter object and the input
temperature.

\item {} 
Set tXtw and wl, and optionally fj and boltzfac; in this
case emit, tw, abd, and Omega are ignored.

\item {} 
Set abd, Omega, wl, and tw, and optionally fj and
boltzfac; in this case tXtw is computed from (abd,
Omega, wl, tw).

\end{enumerate}

It is an error if not enough parameters are specified to
satisfy conditions 1, 2, or 3.

\end{description}

\end{fulllineitems}

\index{m() (despotic.winds.pwind method)}

\begin{fulllineitems}
\phantomsection\label{fulldoc:despotic.winds.pwind.m}\pysiglinewithargsret{\bfcode{m}}{\emph{a}}{}
Return the potential shape function m(a) for this wind
\begin{description}
\item[{Parameters:}] \leavevmode\begin{description}
\item[{a: float or arraylike}] \leavevmode
dimensionless radius

\end{description}

\item[{Returns:}] \leavevmode\begin{description}
\item[{m: float or arraylike}] \leavevmode
potential shape function

\end{description}

\end{description}

\end{fulllineitems}

\index{pdot() (despotic.winds.pwind method)}

\begin{fulllineitems}
\phantomsection\label{fulldoc:despotic.winds.pwind.pdot}\pysiglinewithargsret{\bfcode{pdot}}{\emph{a}, \emph{fg=None}, \emph{tctw=None}}{}
This returns the total momentum flux through the shell at
radius a, normalised to the driving momentum flux.
\begin{description}
\item[{Parameters:}] \leavevmode\begin{description}
\item[{a: float or array}] \leavevmode
dimensionless radius

\item[{fg: float or array}] \leavevmode
gas fraction; must be in the range (0, 1{]}

\item[{tctw: float or array}] \leavevmode
ratio of the crossing time of the launch region to the
wind evacuation time

\end{description}

\item[{Returns:}] \leavevmode\begin{description}
\item[{pdot: float or array}] \leavevmode
momentum flux normalised to driving momentum flux

\end{description}

\item[{Notes:}] \leavevmode
If both fg and tctw are left as None, then the quantity
returned will be the approximate wind momentum flux,
derived taking the mass outflow rate to be Mdot = fg M\_0 /
zeta\_M tc. If both are set, the result is the exact
momentum flux. It is an error if one is None and the other
is not.

\end{description}

\end{fulllineitems}

\index{rho() (despotic.winds.pwind method)}

\begin{fulllineitems}
\phantomsection\label{fulldoc:despotic.winds.pwind.rho}\pysiglinewithargsret{\bfcode{rho}}{\emph{x}, \emph{a}}{}
Return the density divided by Mdot/(4 pi r0\textasciicircum{}2 v0) as a
function of x and a
\begin{description}
\item[{Parameters:}] \leavevmode\begin{description}
\item[{x: float or arraylike}] \leavevmode
dimensionless column density

\item[{a: float or arraylike}] \leavevmode
dimensionless radius

\end{description}

\item[{Returns:}] \leavevmode\begin{description}
\item[{rho: float or arraylike}] \leavevmode
dimensionless density

\end{description}

\end{description}

\end{fulllineitems}

\index{s\_crit() (despotic.winds.pwind method)}

\begin{fulllineitems}
\phantomsection\label{fulldoc:despotic.winds.pwind.s_crit}\pysiglinewithargsret{\bfcode{s\_crit}}{\emph{varpi}, \emph{varpi\_t}, \emph{u=0.0}}{}
Return the distances along of the line of sight where a given
line of sight enters and exits the wind, or passes
through the plane of the sky
\begin{description}
\item[{Parameters:}] \leavevmode\begin{description}
\item[{varpi: float}] \leavevmode
dimensionless impact parameter along the wind axis

\item[{varpi\_t: float}] \leavevmode
dimensionless impact parameter transverse to the wind axis

\item[{u: float}] \leavevmode
velocity of interest; if u \textgreater{} 0, only distances on the
far side wind are returned; if u \textless{} 0, only
distances on the near side are returned; if u == 0.0,
both far and near side distances are returned

\end{description}

\item[{Returns:}] \leavevmode\begin{description}
\item[{s\_crit: array}] \leavevmode
dimensionless locations of wind entry or exit, or
passage through the midplane, ordered from smallest to
largest value of s

\end{description}

\end{description}

\end{fulllineitems}

\index{tau() (despotic.winds.pwind method)}

\begin{fulllineitems}
\phantomsection\label{fulldoc:despotic.winds.pwind.tau}\pysiglinewithargsret{\bfcode{tau}}{\emph{u}, \emph{tXtw=None}, \emph{abd=None}, \emph{Omega=None}, \emph{wl=None}, \emph{muH=1.4}, \emph{tw=None}, \emph{fj=1.0}, \emph{boltzfac=0.0}, \emph{u\_trans=None}, \emph{correlated=True}, \emph{fw=None}, \emph{varpi=0.0}, \emph{varpi\_t=0.0}, \emph{a0=1.0}, \emph{a1=1.7976931348623157e+308}}{}
This returns the optical depth through the wind.
\begin{description}
\item[{Parameters:}] \leavevmode\begin{description}
\item[{u: float or arraylike}] \leavevmode
dimensionless line of sight velocity

\item[{tXtw: float or arraylike}] \leavevmode
ratio tX/tw for the wind

\item[{abd: float or arraylike}] \leavevmode
abundance of absorbers relative to H

\item[{Omega: float or arraylike}] \leavevmode
oscillator strength for transition

\item[{wl: float or arraylike}] \leavevmode
wavelength for transition, in cm

\item[{muH: float or arraylike}] \leavevmode
gas mass per H nucleus, in units of H masses

\item[{tw: float or arraylike}] \leavevmode
mass removal timescale, in sec

\item[{fj: float or array}] \leavevmode
fraction of the emitters in the lower state of the
transition

\item[{boltzfac: float or array}] \leavevmode
Boltzmann factor exp(-E\_ij/kB T\_ex) for the two states of
the transition, where T\_ex = excitation temperature

\item[{u\_trans: arraylike}] \leavevmode
velocity offset between transitions for multiple
transition computations

\item[{correlated: bool}] \leavevmode
if True, assume correlated winds; if False, uncorrelated

\item[{fw: float or arraylike}] \leavevmode
covering factor of wind at launch point; if left as
None, defaults to zeta\_A; only used if correlated is True

\item[{varpi: float or arraylike}] \leavevmode
dimensionless impact parameter along the wind axis

\item[{varpi\_t: float or arraylike}] \leavevmode
dimensionless impact parameter transverse to the wind axis

\item[{a0: float or arraylike}] \leavevmode
minimum radius from which to integrate optical depth

\item[{a1: float or arraylike}] \leavevmode
maximum radius from which to integrate optical depth

\end{description}

\item[{Returns:}] \leavevmode\begin{description}
\item[{tau: float or array}] \leavevmode
optical depth at specified velocity

\end{description}

\item[{Notes:}] \leavevmode
The user can choose to specify either tXtw, or the
combination of parameters (abd, Omega, wl, tw). In the
latter case, the user can also optionally change the gas
mass per H nucleus by setting muH. If tXtw is set, the
other parameters will be ignored. If it is not set, then an
error is raised if any of abd, Omega, wl, or tw is not
specified.

If u\_trans is set, this changes the interpretation of the
other variables. If u\_trans is not None, then the trailing
dimension of either tXtw or (Omega, wl) must have the same
number of elements as u\_trans, and will be interpreted as
giving the value of tXtw, or equivalently the oscillator
strength and wavelength, for each transition.

\end{description}

\end{fulllineitems}

\index{tau\_c() (despotic.winds.pwind method)}

\begin{fulllineitems}
\phantomsection\label{fulldoc:despotic.winds.pwind.tau_c}\pysiglinewithargsret{\bfcode{tau\_c}}{\emph{u}, \emph{tXtw=None}, \emph{abd=None}, \emph{Omega=None}, \emph{wl=None}, \emph{muH=1.4}, \emph{tw=None}, \emph{fw=None}, \emph{fj=1.0}, \emph{boltzfac=0.0}, \emph{u\_trans=None}, \emph{varpi=0.0}, \emph{varpi\_t=0.0}, \emph{a0=1.0}, \emph{a1=1.7976931348623157e+308}}{}
This returns the optical depth assuming that the wind is
correlated.
\begin{description}
\item[{Parameters:}] \leavevmode\begin{description}
\item[{u: float or arraylike}] \leavevmode
dimensionless line of sight velocity

\item[{tXtw: float or arraylike}] \leavevmode
ratio tX/tw for the wind

\item[{abd: float or arraylike}] \leavevmode
abundance of absorbers relative to H

\item[{Omega: float or arraylike}] \leavevmode
oscillator strength for transition

\item[{wl: float or arraylike}] \leavevmode
wavelength for transition, in cm

\item[{muH: float or arraylike}] \leavevmode
gas mass per H nucleus, in units of H masses

\item[{tw: float or arraylike}] \leavevmode
mass removal timescale, in sec

\item[{fw: float or arraylike}] \leavevmode
covering factor of wind at launch point; if left as
None, defaults to zeta\_A

\item[{fj: float or array}] \leavevmode
fraction of the emitters in the lower state of the
transition

\item[{boltzfac: float or array}] \leavevmode
Boltzmann factor exp(-E\_ij/kB T\_ex) for the two states of
the transition, where T\_ex = excitation temperature

\item[{u\_trans: arraylike}] \leavevmode
velocity offset between transitions for multiple
transition computations

\item[{varpi: float or arraylike}] \leavevmode
dimensionless impact parameter along the wind axis

\item[{varpi\_t: float or arraylike}] \leavevmode
dimensionless impact parameter transverse to the wind axis

\item[{a0: float or arraylike}] \leavevmode
minimum radius from which to integrate optical depth

\item[{a1: float or arraylike}] \leavevmode
maximum radius from which to integrate optical depth

\end{description}

\item[{Returns:}] \leavevmode\begin{description}
\item[{tau: float or array}] \leavevmode
optical depth at specified velocity

\end{description}

\item[{Notes:}] \leavevmode
The user can choose to specify either tXtw, or the
combination of parameters (abd, Omega, wl, tw). In the
latter case, the user can also optionally change the gas
mass per H nucleus by setting muH. If tXtw is set, the
other parameters will be ignored. If it is not set, then an
error is raised if any of abd, Omega, wl, or tw is not
specified.

If u\_trans is set, this changes the interpretation of the
other variables. If u\_trans is not None, then the trailing
dimension of either tXtw or (Omega, wl) must have the same
number of elements as u\_trans, and will be interpreted as
giving the value of tXtw, or equivalently the oscillator
strength and wavelength, for each transition.

\end{description}

\end{fulllineitems}

\index{tau\_uc() (despotic.winds.pwind method)}

\begin{fulllineitems}
\phantomsection\label{fulldoc:despotic.winds.pwind.tau_uc}\pysiglinewithargsret{\bfcode{tau\_uc}}{\emph{u}, \emph{tXtw=None}, \emph{abd=None}, \emph{Omega=None}, \emph{wl=None}, \emph{muH=1.4}, \emph{tw=None}, \emph{fj=1.0}, \emph{boltzfac=0.0}, \emph{u\_trans=None}, \emph{varpi=0.0}, \emph{varpi\_t=0.0}, \emph{a0=1.0}, \emph{a1=1.7976931348623157e+308}}{}
This returns the optical depth assuming that the wind is
uncorrelated.
\begin{description}
\item[{Parameters:}] \leavevmode\begin{description}
\item[{u: float or arraylike}] \leavevmode
dimensionless line of sight velocity

\item[{tXtw: float or arraylike}] \leavevmode
ratio tX/tw for the wind

\item[{abd: float or arraylike}] \leavevmode
abundance of absorbers relative to H

\item[{Omega: float or arraylike}] \leavevmode
oscillator strength for transition

\item[{wl: float or arraylike}] \leavevmode
wavelength for transition, in cm

\item[{muH: float or arraylike}] \leavevmode
gas mass per H nucleus, in units of H masses

\item[{tw: float or arraylike}] \leavevmode
mass removal timescale, in sec

\item[{fj: float or array}] \leavevmode
fraction of the emitters in the lower state of the
transition

\item[{boltzfac: float or array}] \leavevmode
Boltzmann factor exp(-E\_ij/kB T\_ex) for the two states of
the transition, where T\_ex = excitation temperature

\item[{u\_trans: arraylike}] \leavevmode
velocity offset between transitions for multiple
transition computations

\item[{varpi: float or arraylike}] \leavevmode
dimensionless impact parameter along the wind axis

\item[{varpi\_t: float or arraylike}] \leavevmode
dimensionless impact parameter transverse to the wind axis

\item[{a0: float or arraylike}] \leavevmode
minimum radius from which to integrate optical depth

\item[{a1: float or arraylike}] \leavevmode
maximum radius from which to integrate optical depth

\end{description}

\item[{Returns:}] \leavevmode\begin{description}
\item[{tau: float or array}] \leavevmode
optical depth at specified velocity

\end{description}

\item[{Notes:}] \leavevmode
The user can choose to specify either tXtw, or the
combination of parameters (abd, Omega, wl, tw). In the
latter case, the user can also optionally change the gas
mass per H nucleus by setting muH. If tXtw is set, the
other parameters will be ignored. If it is not set, then an
error is raised if any of abd, Omega, wl, or tw is not
specified.

If u\_trans is set, this changes the interpretation of the
other variables. If u\_trans is not None, then the trailing
dimension of either tXtw or (Omega, wl) must have the same
number of elements as u\_trans, and will be interpreted as
giving the value of tXtw, or equivalently the oscillator
strength and wavelength, for each transition.

\end{description}

\end{fulllineitems}

\index{temp\_LTE() (despotic.winds.pwind method)}

\begin{fulllineitems}
\phantomsection\label{fulldoc:despotic.winds.pwind.temp_LTE}\pysiglinewithargsret{\bfcode{temp\_LTE}}{\emph{u}, \emph{T}, \emph{emit=None}, \emph{tw=None}, \emph{tXtw=None}, \emph{abd=None}, \emph{Omega=None}, \emph{wl=None}, \emph{muH=1.4}, \emph{fj=1.0}, \emph{boltzfac=0.0}, \emph{trans=0}, \emph{varpi=0.0}, \emph{varpi\_t=0.0}, \emph{correlated=True}, \emph{fw=None}, \emph{thin=False}, \emph{TA=False}}{}
Return the brightness or antenna temperature of line emission
for a species in LTE observed at velocity u
\begin{description}
\item[{Parameters:}] \leavevmode\begin{description}
\item[{u: float or array}] \leavevmode
dimensionless line of sight velocity

\item[{T: float or array}] \leavevmode
wind kinetic temperature, in K

\item[{emit: emitter or sequence of emitter}] \leavevmode
an emitter object for the transition of interest

\item[{tw: float or arraylike}] \leavevmode
mass removal timescale, in sec

\item[{tXtw: float or array}] \leavevmode
ratio of timescales tX and tw

\item[{abd: float or arraylike}] \leavevmode
abundance of absorbers relative to H

\item[{Omega: float or arraylike}] \leavevmode
oscillator strength for transition

\item[{wl: float or arraylike}] \leavevmode
wavelength for transition, in cm

\item[{muH: float or arraylike}] \leavevmode
gas mass per H nucleus, in units of H masses

\item[{fj: float or array}] \leavevmode
fraction of the emitters in the lower state of the
transition

\item[{boltzfac: float or array}] \leavevmode
Boltzmann factor exp(-E\_ij/kB T) for the two states of
the transition

\item[{trans: int or array of int}] \leavevmode
transition for which to compute the emission

\item[{varpi: float or arraylike}] \leavevmode
dimensionless impact parameter along the wind axis

\item[{varpi\_t: float or arraylike}] \leavevmode
dimensionless impact parameter transverse to the wind axis

\item[{correlated: bool}] \leavevmode
if True, assume correlated winds; if False, uncorrelated

\item[{fw: float or arraylike}] \leavevmode
covering factor of wind at launch point; if left as

\item[{thin: bool or array of bool}] \leavevmode
if True, the escape probability is set to 1

\item[{TA: bool}] \leavevmode
if True, quantity returned is antenna temperature;
default is to return brightness temperature

\end{description}

\item[{Returns:}] \leavevmode\begin{description}
\item[{Temp: float or array}] \leavevmode
brightness temperature (or antenna temperature if TA is
set to True) in K as a function of the velocity u

\end{description}

\item[{Note:}] \leavevmode
The user can provide the required input in the following
ways, which are checked in order.
\begin{enumerate}
\item {} 
Set emit and tw, and optionally trans; in this case
tXtw, abd, Omega, wl, fj, and boltzfac are all ignored;
the abundance and all properties of the line are taken
from emit, and the Boltzmann factors are computed
directly from the emitter object and the input
temperature.

\item {} 
Set tXtw and wl, and optionally fj and boltzfac; in this
case emit, tw, abd, and Omega are ignored.

\item {} 
Set abd, Omega, wl, and tw, and optionally fj and
boltzfac; in this case tXtw is computed from (abd,
Omega, wl, tw).

\end{enumerate}

It is an error if not enough parameters are specified to
satisfy conditions 1, 2, or 3.

\end{description}

\end{fulllineitems}

\index{xi() (despotic.winds.pwind method)}

\begin{fulllineitems}
\phantomsection\label{fulldoc:despotic.winds.pwind.xi}\pysiglinewithargsret{\bfcode{xi}}{\emph{varpi=0.0}, \emph{varpi\_t=0.0}}{}
Return the optically thin integrated intensity function xi.
\begin{description}
\item[{Parameters:}] \leavevmode\begin{description}
\item[{varpi: float or arraylike}] \leavevmode
dimensionless impact parameter along the wind axis

\item[{varpi\_t: float or arraylike}] \leavevmode
dimensionless impact parameter transverse to the wind axis

\end{description}

\item[{Returns:}] \leavevmode\begin{description}
\item[{xi: float or array}] \leavevmode
subcritical line emission intensity function

\end{description}

\end{description}

\end{fulllineitems}

\index{y() (despotic.winds.pwind method)}

\begin{fulllineitems}
\phantomsection\label{fulldoc:despotic.winds.pwind.y}\pysiglinewithargsret{\bfcode{y}}{\emph{a}}{}
Return the cloud area function y(a) for this wind
\begin{description}
\item[{Parameters:}] \leavevmode\begin{description}
\item[{a: float or arraylike}] \leavevmode
dimensionless radius

\end{description}

\item[{Returns:}] \leavevmode\begin{description}
\item[{y: float or arraylike}] \leavevmode
cloud area function

\end{description}

\end{description}

\end{fulllineitems}


\end{fulllineitems}



\section{despotic.winds functions}
\label{fulldoc:despotic-winds-functions}

\subsection{\texttt{sxMach}}
\label{fulldoc:sxmach}\index{sxMach() (in module despotic.winds.pwind\_util)}

\begin{fulllineitems}
\phantomsection\label{fulldoc:despotic.winds.pwind_util.sxMach}\pysiglinewithargsret{\code{despotic.winds.pwind\_util.}\bfcode{sxMach}}{\emph{mach}}{}
Returns the dispersion in log column density versus Mach number
\begin{description}
\item[{Parameters:}] \leavevmode\begin{description}
\item[{mach: float or arraylike}] \leavevmode
Mach number

\end{description}

\item[{Returns:}] \leavevmode\begin{description}
\item[{sx: float or array}] \leavevmode
dispersion of log column density

\end{description}

\end{description}

\end{fulllineitems}



\subsection{\texttt{zetaM}}
\label{fulldoc:zetam}\index{zetaM() (in module despotic.winds.pwind\_util)}

\begin{fulllineitems}
\phantomsection\label{fulldoc:despotic.winds.pwind_util.zetaM}\pysiglinewithargsret{\code{despotic.winds.pwind\_util.}\bfcode{zetaM}}{\emph{xcrit}, \emph{sx}}{}
Returns the mass fraction at column density x \textless{} xcrit
\begin{description}
\item[{Parameters:}] \leavevmode\begin{description}
\item[{xcrit: float or arraylike}] \leavevmode
maximum column density

\item[{sx: float or arraylike}] \leavevmode
dispersion of column densities

\end{description}

\item[{Returns:}] \leavevmode\begin{description}
\item[{zetaM: float or array}] \leavevmode
mass fraction with x \textless{} xcrit

\end{description}

\end{description}

\end{fulllineitems}



\subsection{\texttt{zetaA}}
\label{fulldoc:zetaa}\index{zetaA() (in module despotic.winds.pwind\_util)}

\begin{fulllineitems}
\phantomsection\label{fulldoc:despotic.winds.pwind_util.zetaA}\pysiglinewithargsret{\code{despotic.winds.pwind\_util.}\bfcode{zetaA}}{\emph{xcrit}, \emph{sx}}{}
Returns the area fraction at column density x \textless{} xcrit
\begin{description}
\item[{Parameters:}] \leavevmode\begin{description}
\item[{xcrit: float or arraylike}] \leavevmode
maximum column density

\item[{sx: float or arraylike}] \leavevmode
dispersion of column densities

\end{description}

\item[{Returns:}] \leavevmode\begin{description}
\item[{zetaA: float or array}] \leavevmode
area fraction with x \textless{} xcrit

\end{description}

\end{description}

\end{fulllineitems}



\subsection{\texttt{pM}}
\label{fulldoc:pm}\index{pM() (in module despotic.winds.pwind\_util)}

\begin{fulllineitems}
\phantomsection\label{fulldoc:despotic.winds.pwind_util.pM}\pysiglinewithargsret{\code{despotic.winds.pwind\_util.}\bfcode{pM}}{\emph{x}, \emph{sx}}{}
Returns the mass PDF evaluated at column density x
\begin{description}
\item[{Parameters:}] \leavevmode\begin{description}
\item[{x: float or arraylike}] \leavevmode
dimensionless log column density

\item[{sx: float or arraylike}] \leavevmode
dispersion of column densities

\end{description}

\item[{Returns:}] \leavevmode\begin{description}
\item[{pM: float or array}] \leavevmode
mass PDF of column densities evaluated at x

\end{description}

\end{description}

\end{fulllineitems}



\subsection{\texttt{pA}}
\label{fulldoc:pa}\index{pA() (in module despotic.winds.pwind\_util)}

\begin{fulllineitems}
\phantomsection\label{fulldoc:despotic.winds.pwind_util.pA}\pysiglinewithargsret{\code{despotic.winds.pwind\_util.}\bfcode{pA}}{\emph{x}, \emph{sx}}{}
Returns the area PDF evaluated at column density x
\begin{description}
\item[{Parameters:}] \leavevmode\begin{description}
\item[{x: float or arraylike}] \leavevmode
dimensionless log column density

\item[{sx: float or arraylike}] \leavevmode
dispersion of column densities

\end{description}

\item[{Returns:}] \leavevmode\begin{description}
\item[{pA: float or array}] \leavevmode
area PDF of column densities evaluated at x

\end{description}

\end{description}

\end{fulllineitems}



\subsection{\texttt{tX}}
\label{fulldoc:tx}\index{tX() (in module despotic.winds.pwind\_util)}

\begin{fulllineitems}
\phantomsection\label{fulldoc:despotic.winds.pwind_util.tX}\pysiglinewithargsret{\code{despotic.winds.pwind\_util.}\bfcode{tX}}{\emph{abd}, \emph{Omega}, \emph{wl}, \emph{muH=1.4}}{}
Returns the line timescale tX
\begin{description}
\item[{Parameters:}] \leavevmode\begin{description}
\item[{abd: float or array}] \leavevmode
abundance relative to hydrogen

\item[{muH: float or array}] \leavevmode
gas mass per H nucleus, in units of H masses

\item[{Omega: float or array}] \leavevmode
transition oscillator strength

\item[{wl: float or array}] \leavevmode
transition wavelength in cm

\end{description}

\item[{Returns:}] \leavevmode\begin{description}
\item[{tX: float or array}] \leavevmode
transition time parameter tX

\end{description}

\end{description}

\end{fulllineitems}



\chapter{Acknowledgements}
\label{acknowledgements:acknowledgements}\label{acknowledgements::doc}\label{acknowledgements:sec-acknowledgements}
DESPOTIC was primarily written by \href{http://www.mso.anu.edu.au/~krumholz/}{Mark Krumholz}, but it contains or relies on
the contributions of a number of others.
\begin{itemize}
\item {} 
Desika Narayanan has done more than anyone else to field test the
code and discover bugs.

\item {} 
DESPOTIC grew out of a code initially written in collaboration with
Todd Thompson

\item {} 
Adam Ginsburg provided useful advice and aided in early testing.

\item {} 
Floris van der Tak has assisted in field testing the code, and
helped improve the quality of the documentation.

\item {} 
DESPOTIC would not be possible without the \href{http://home.strw.leidenuniv.nl/~moldata/}{Leiden Atomic and
Molecular Database},
which is maintained by Floris van der Tak, Ewine van Dishoeck, and
John Black, and which was initially established by Feredic Schöier

\end{itemize}

Mark Krumholz's work writing DESPOTIC was funded in part by the US
National Science Foundation (grant AST-0955300) and NASA (grant
NNX09AK31G). Continued development is supported by the Australian
Research Council (grant DP160100695).


\chapter{Indices and tables}
\label{index:indices-and-tables}\begin{itemize}
\item {} 
\DUrole{xref,std,std-ref}{genindex}

\item {} 
\DUrole{xref,std,std-ref}{modindex}

\item {} 
\DUrole{xref,std,std-ref}{search}

\end{itemize}



\renewcommand{\indexname}{Index}
\printindex
\end{document}
